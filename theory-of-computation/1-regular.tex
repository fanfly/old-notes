\chapter{Regular Languages}
\section{Languages}
\begin{definition}
  \label{def:alphabet}
  An \emph{alphabet} is a finite nonempty set of symbols.
\end{definition}

\begin{definition}
  \label{def:string}
  Let $\Sigma$ be an alphabet.
  \begin{itemize}
    \item A \emph{string} over $\Sigma$ is a finite sequence of symbols from
    $\Sigma$.
    The collection of all strings over $\Sigma$ is denoted by $\Sigma^*$.
    \item The \emph{length} of a string $w$, denoted by $|w|$, is the number of
    symbols it contains.
    \item The string containing no symbols is called the \emph{empty string},
    denoted by $\epsilon$.
  \end{itemize}
\end{definition}

\begin{definition}
  \label{def:language}
  A subset of $\Sigma^*$ is called a \emph{language} over $\Sigma$.
\end{definition}

\section{Deterministic Finite State Automata}
\begin{definition}
  \label{def:dfa}
  A \emph{deterministic finite state automaton} (DFA) is a tuple
  \begin{equation*}
    A = (Q, \Sigma, \delta, q_0, F),
  \end{equation*}
  where each component is as follows.
  \begin{itemize}
    \item $Q$ is a finite set of \emph{states}.
    \item $\Sigma$ is a finite set of input symbols.
    \item $\delta: Q \times \Sigma \to Q$ is a function, called the
    \emph{transition function}.
    \item $q_0 \in Q$ is called the \emph{start state}.
    \item $F \subseteq Q$ is called the \emph{accepting states}.
  \end{itemize}
\end{definition}

\begin{definition}
  Let $A = (Q, \Sigma, \delta, q_0, F)$ be a DFA.
  \begin{enumerate}
    \item For each symbol $a \in \Sigma$, we define $\delta_a: Q \to Q$ to be
    the function such that $\delta_a(p) = \delta(p, a)$
    for any states $p, q \in Q$.
    \item For each string $w \in \Sigma^*$, we define $\delta_w: Q \to Q$ as
    follows.
    \begin{itemize}
      \item $\delta_\epsilon$ is the identity function.
      \item For any strings $x \in \Sigma^*$ and any symbol $a \in \Sigma$,
      the function $\delta_{xa}$ satisfies
      $\delta_{xa}(p) = \delta_a(\delta_x(p))$ for any $p \in Q$.
    \end{itemize}
  \end{enumerate}
\end{definition}

\begin{definition}
  Let $A = (Q, \Sigma, \delta, q_0, F)$ be a DFA.
  \begin{itemize}
    \item We say that $A$ accepts a string $w \in \Sigma^*$ if
    $\delta_w(q_0) \in F$.
    \item The \emph{language} of $A$, denoted $L(A)$, is defined as the set of
    strings that are accepted by $A$.
  \end{itemize}
\end{definition}

\begin{definition}
  A language $L$ is \emph{regular} if there exists a DFA $A$ such that
  $L(A) = L$.
\end{definition}

\section{Nondeterministic Finite State Automata}
\begin{definition}
  \label{def:nfa}
  A \emph{nondeterministic finite state automaton} (NFA) is a tuple
  \begin{equation*}
    A = (Q, \Sigma, \delta, q_0, F),
  \end{equation*}
  where each component is as follows.
  \begin{itemize}
    \item $Q$ is a finite set of \emph{states}.
    \item $\Sigma$ is a finite set of input symbols.
    \item $\delta: Q \times \Sigma \times Q$ is a relation, called the
    \emph{transition relation}.
    \item $q_0 \in Q$ is called the \emph{start state}.
    \item $F \subseteq Q$ is called the \emph{accepting states}.
  \end{itemize}
\end{definition}

\begin{definition}
  Let $A = (Q, \Sigma, \delta, q_0, F)$ be an NFA.
  \begin{enumerate}
    \item For each symbol $a \in \Sigma$, we define
    $\delta_a \subseteq Q \times Q$ to be the relation such that
    $(p, q) \in \delta_a$ if and only if $(p, a, q) \in \delta$ for any states
    $p, q \in Q$.
    \item For each string $w \in \Sigma^*$, we define
    $\delta_w \subseteq Q \times Q$ as follows.
    \begin{itemize}
      \item $\delta_\epsilon$ is the identity relation.
      \item For any strings $x \in \Sigma^*$, any symbol $a \in \Sigma$ and
      any states $p, q \in Q$,
      \begin{equation*}
        (p, q) \in \delta_{xa}
      \end{equation*}
      if and only if there exists a state $r \in Q$ such that
      \begin{equation*}
        (p, r) \in \delta_x \quad \text{and} \quad (r, q) \in \delta_a.
      \end{equation*}
    \end{itemize}
  \end{enumerate}
\end{definition}

\begin{definition}
  Let $A = (Q, \Sigma, \delta, q_0, F)$ be an NFA.
  \begin{itemize}
    \item We say that $A$ accepts a string $w \in \Sigma^*$ if there exists
    $q \in F$ such that $(q_0, q) \in \delta_w$.
    \item The \emph{language} of $A$, denoted $L(A)$, is defined as the set of
    strings that are accepted by $A$.
  \end{itemize}
\end{definition}

\begin{theorem}
  For every NFA $A$, there is a DFA $A'$ with $L(A') = L(A)$.
\end{theorem}
\begin{proof}
  Let $A = (Q, \Sigma, \delta, q_0, F)$.
  We construct $A' = (\mathcal{P}(Q), \Sigma, \Delta, \{q_0\}, \Phi)$
  as follows.
  \begin{itemize}
    \item $\Delta: \mathcal{P}(Q) \times \Sigma \to \mathcal{P}(Q)$ is the
    function with
    \begin{equation*}
      \Delta_a(P) = \bigcup_{p \in P} \; \{q \in Q: (p, q) \in \delta_a\}
    \end{equation*}
    for any $P \subseteq Q$ and $a \in \Sigma$.
    \item $\Phi = \{P \subseteq Q: P \cap F \neq \varnothing\}$.
  \end{itemize}
  Now we prove that for any $w \in \Sigma^*$, for any $q \in Q$ and for any
  $P \subseteq Q$, we have $q \in \Delta_w(P)$ if and only if
  $(p, q) \in \delta_w$ for some $p \in P$.
  For the induction basis, let $w = \epsilon$, and we have
  \begin{equation*}
    q \in \Delta_\epsilon(P)
    \quad \Leftrightarrow \quad
    q \in P
    \quad \Leftrightarrow \quad
    \text{$(p, q) \in \delta_\epsilon$ for some $p \in P$}.
  \end{equation*}
  For the induction step, let $w = xa$, where $a$ is the last symbol of $w$.
  Note that by the construction of $\Delta$, we have $q \in \Delta_a(P)$ if and
  only if $(p, q) \in \delta_a$ for some $p \in P$.
  Thus, we can conclude that
  \begin{align*}
    q \in \Delta_{xa}(P)
    \quad &\Leftrightarrow \quad
    q \in \Delta_a(\Delta_x(P)) \\[.5em]
    \quad &\Leftrightarrow \quad
    \text{$(r, q) \in \delta_a$ for some $r \in \Delta_x(P)$} \\
    &\hspace*{2.6em} \text{and $(p, r) \in \delta_x$ for some $p \in P$}
    \\[.5em]
    \quad &\Leftrightarrow \quad
    \text{$(p, q) \in \delta_{xa}$ for some $p \in P$}.
  \end{align*}
  Finally we prove that $L(A') = L(A)$, which is given by
  \begin{align*}
    w \in L(A')
    \quad &\Leftrightarrow \quad
    \Delta_w(\{q_0\}) \in \Phi \\[.5em]
    \quad &\Leftrightarrow \quad
    \Delta_w(\{q_0\}) \cap F \neq \varnothing \\[.5em]
    \quad &\Leftrightarrow \quad
    \text{$q \in \Delta_w(\{q_0\})$ for some $q \in F$} \\[.5em]
    \quad &\Leftrightarrow \quad
    \text{$(p, q) \in \delta_w$ for some $q \in F$ and $p \in \{q_0\}$} \\[.5em]
    \quad &\Leftrightarrow \quad
    \text{$(q_0, q) \in \delta_w$ for some $q \in F$} \\[.5em]
    \quad &\Leftrightarrow \quad
    w \in L(A).
    \qedhere
  \end{align*}
\end{proof}

\section{Regular Expressions}
\begin{definition}
  Let $\Sigma$ be an alphabet.
  A \emph{regular expression} over $\Sigma$ is a string in the minimal language
  over $\Sigma \cup \{\varnothing, \epsilon, {}^*, \cup, (, )\}$
  that satisfies the following conditions.
  \begin{enumerate}[1.]
    \item $\varnothing$ is a regular expression.
    \item $\epsilon$ is a regular expression.
    \item If $a \in \Sigma$, then $a$ is a regular expression.
    \item If $e_1$ and $e_2$ are regular expressions, then so is $(e_1e_2)$.
    \item If $e_1$ and $e_2$ are regular expressions, then so is
    $(e_1 + e_2)$.
    \item If $e$ is a regular expression, then so is $(e)^*$.
  \end{enumerate}
\end{definition}

\begin{definition}
  A regular expression $e$ over an alphabet $\Sigma$ defines a language $L(e)$
  as follows.
  \begin{enumerate}[1.]
    \item $L(\varnothing) = \varnothing$.
    \item $L(\epsilon) = \{\epsilon\}$.
    \item $L(a) = \{a\}$ for each $a \in \Sigma$.
    \item $L((e_1e_2)) = L(e_1)L(e_2)$ for each regular expressions $e_1$ and
    $e_2$.
    \item $L((e_1 + e_2)) = L(e_1) \cup L(e_2)$ for each regular
    expressions $e_1$ and $e_2$.
    \item $L((e)^*) = L(e^*)$ for each regular expression $e$.
  \end{enumerate}
\end{definition}

\begin{remark}
  We may omit parentheses if there is no ambiguity.
\end{remark}

\begin{lemma}
  If $L$ is a regular language over an alphabet $\Sigma$, then there is a
  regular expression $e$ over $\Sigma$ such that $L(e) = L$.
\end{lemma}
\begin{proof}
  Since $L$ is regular, there exists a DFA $A = (Q, \Sigma, \delta, q_0, F)$
  with $L(A) = L$.
  Suppose that $Q = \{p_1, p_2, \dots, p_n\}$ with $p_1 = q_0$.
  For any $1 \leq i \leq n$, for any $1 \leq j \leq n$ and for any
  $0 \leq k \leq n$, let $L_{ij}^{(k)}$ be the language of strings $w$
  satisfying the following conditions (a) and (b).
  \begin{enumerate}
    \item $\delta_w(p_i) = p_j$.
    \item For any nonempty prefix $x$ of $w$, $\delta_x(p_i) = p_\ell$ for some
    $\ell \leq k$.
    (A nonempty proper prefix $x$ of $w$ is a string $x$ such that $w = xy$
    with $x \neq \epsilon$ and $y \neq \epsilon$.)
  \end{enumerate}
  We are going to prove that for all $1 \leq i \leq n$, $1 \leq j \leq n$ and
  $0 \leq k \leq n$, there exists a regular expression $e_{ij}^{(k)}$
  such that
  \begin{equation*}
    L\bigl(e_{ij}^{(k)}\bigr) = L_{ij}^{(k)}.
  \end{equation*}

  The proof is by induction on $k$. For the induction basis, let $k = 0$.
  Let $\Pi_{ij} \subseteq \Sigma$ be the collection of symbols $a$ with
  $\delta_a(p_i) = p_j$.
  If $i \neq j$, we have
  \begin{equation*}
    L_{ij}^{(0)} = \bigcup_{a \in \Pi_{ij}} \{a\},
  \end{equation*}
  and thus we can construct $e_{ij}^{(0)}$ by
  \begin{equation*}
    e_{ij}^{(0)} = \sum_{a \in \Pi_{ij}} a.
  \end{equation*}
  (If $\Pi_{ij} = \varnothing$, then the summation is defined as
  $\varnothing$.)
  If $i = j$, we have
  \begin{equation*}
    L_{ii}^{(0)} = \{\epsilon\} \cup \bigcup_{a \in \Pi_{ii}} \{a\},
  \end{equation*}
  and thus we can construct $e_{ii}^{(0)}$ by
  \begin{equation*}
    e_{ii}^{(0)} = \epsilon + \sum_{a \in \Pi_{ii}} a.
  \end{equation*}
  Now for the induction step, let $k \geq 1$.
  Suppose that $w \in L_{ij}^{(k)}$.
  \begin{itemize}
    \item If there is no nonempty proper prefix $x$ of $w$ such that
    $\delta_x(p_i) = p_k$, then we have $w \in L_{ij}^{(k-1)}$.
    \item Otherwise, let $x_0, x_1, \dots, x_\ell$ be all nonempty
    proper prefixes of $w$ such that
    \begin{equation*}
      \delta_{x_0}(p_i)
      = \delta_{x_1}(p_i)
      = \cdots
      = \delta_{x_\ell}(p_i)
      = p_k,
    \end{equation*}
    where $x_{h-1}$ is a proper prefix of $x_h$ for $1 \leq h \leq \ell$.
    Then there exist $u_0, u_1, \dots, u_{\ell+1}$ such that
    $w = u_0u_1 \cdots u_{\ell+1}$, $x_0 = u_0$, and $x_h = x_{h-1}u_h$
    for $1 \leq h \leq \ell$.
    Note that we have $u_0 \in L_{ik}^{(k-1)}$,
    $u_{\ell+1} \in L_{kj}^{(k-1)}$, and
    $u_h \in L_{kk}^{(k-1)}$ for $1 \leq h \leq \ell$.
    Thus, we can conclude that
    \begin{equation*}
      w \in L_{ik}^{(k-1)}\left(L_{kk}^{(k-1)}\right)^*L_{kj}^{(k-1)}.
    \end{equation*}
  \end{itemize}
  As a result, we have
  \begin{equation*}
    L_{ij}^{(k)} \subseteq L_{ij}^{(k-1)} \cup
    L_{ik}^{(k-1)}\left(L_{kk}^{(k-1)}\right)^*L_{kj}^{(k-1)},
  \end{equation*}
  implying
  \begin{equation*}
    L_{ij}^{(k)} = L_{ij}^{(k-1)} \cup
    L_{ik}^{(k-1)}\left(L_{kk}^{(k-1)}\right)^*L_{kj}^{(k-1)}.
  \end{equation*}
  Therefore, we can construct $e_{ij}^{(k)}$ by
  \begin{equation*}
    e_{ij}^{(k)} = e_{ij}^{(k-1)}
    + e_{ik}^{(k-1)}\left(e_{kk}^{(k-1)}\right)^*e_{kj}^{(k-1)}.
  \end{equation*}

  Now we construct the regular expression $e$ with $L(e) = L$.
  Let $\Phi$ be the set of integers $j \in \{1, \dots, n\}$ such that
  $p_j \in F$.
  Note that we have
  \begin{equation*}
    L = \bigcup_{j \in \Phi} L_{1j}^{(n)},
  \end{equation*}
  and thus $e$ can be constructed by
  \begin{equation*}
    e = \sum_{j \in \Phi} e_{1j}^{(n)},
  \end{equation*}
  which completes the proof.
\end{proof}