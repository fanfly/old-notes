\chapter{Regular Languages}
\section{Deterministic Finite State Automata}
\begin{definition}\label{def:string}
  An \emph{alphabet} $\Sigma$ is a finite set of symbols.
  \begin{itemize}
    \item A \emph{string} over $\Sigma$ is a finite sequence of symbols from
      $\Sigma$.
    \item The \emph{length} of a string $w$, denoted by $|w|$, is the number of
      symbols it contains.
    \item The string of length $0$ is called the \emph{empty string}, denoted
      by $\epsilon$.
  \end{itemize}
\end{definition}

\begin{definition}\label{def:language}
  Let $\Sigma$ be an alphabet.
  \begin{itemize}
    \item For any nonnegative integer $n$, $\Sigma^n$ denotes the set of words
      of length $n$.
    \item $\Sigma^*$ denotes the set of all strings over $\Sigma$.
    \item A \emph{language} over $\Sigma$ is a subset of $\Sigma^*$.
  \end{itemize}
\end{definition}

\begin{definition}\label{def:dfa}
  A \emph{deterministic finite state automaton} (DFA) is a system
  $\mathcal{A} = (\Sigma, Q, q_0, F, \delta)$, where each component is as
  follows.
  \begin{itemize}
    \item $\Sigma$ is the alphabet.
    \item $Q$ is a finite set of \emph{states}.
    \item $q_0 \in Q$ is the \emph{initial} state.
    \item $F \subseteq Q$ is the set of \emph{accepting} states.
    \item $\delta$ is the \emph{transition function} from $Q \times \Sigma$ to
      $Q$.
  \end{itemize}
\end{definition}

\begin{definition}\label{def:dfa-run}
  The \emph{run} of DFA $\mathcal{A} = (\Sigma, Q, q_0, F, \delta)$ on an input
  string $w = a_1 \cdots a_n$ over $\Sigma$ is the sequence of states
  \begin{equation*}
    r = (r_0, r_1, \dots, r_n)
  \end{equation*}
  where $r_0 = q_0$ and $\delta(r_{i-1}, a_i) = r_i$ for each
  $i \in \{1, \dots, n\}$.
  \begin{itemize}
    \item $r$ is an \emph{accepting} run if $r_n \in F$.
    \item We say that $\mathcal{A}$ \emph{accepts} $w$ if the run of
      $\mathcal{A}$ on $w$ is an accepting run.
    \item The language of all strings accepted by $\mathcal{A}$ is denoted by
      $L(\mathcal{A})$.
    \item A language $L$ is \emph{regular} if there is a DFA $\mathcal{A}$ with
      $L = L(\mathcal{A})$.
  \end{itemize}
\end{definition}

\begin{remark}
  \leavevmode
  \begin{itemize}
    \item For DFA $\mathcal{A} = (\Sigma, Q, q_0, F, \delta)$, the empty string
      $\epsilon$ is accepted by $\mathcal{A}$ if and only if $q_0 \in F$.
  \end{itemize}
\end{remark}

\section{Nondeterministic Finite State Automata}
\begin{definition}\label{def:nfa}
  A \emph{nondeterministic finite state automaton} (NFA) is a system
  $\mathcal{A} = (\Sigma, Q, q_0, F, \delta)$, where each component is as
  follows.
  \begin{itemize}
    \item $\Sigma$ is the alphabet.
    \item $Q$ is a finite set of \emph{states}.
    \item $q_0 \in Q$ is the \emph{initial} state.
    \item $F \subseteq Q$ is the set of \emph{accepting} states.
    \item $\delta \subseteq Q \times \Sigma \times Q$ is the
      \emph{transition relation}.
  \end{itemize}
\end{definition}

\begin{definition}\label{def:nfa-run}
  A \emph{run} of NFA $\mathcal{A} = (\Sigma, Q, q_0, F, \delta)$ on an input
  string $w = a_1 \cdots a_n$ over $\Sigma$ is the sequence of states
  \begin{equation*}
    r = (r_0, r_1, \dots, r_n)
  \end{equation*}
  where $r_0 = q_0$ and $(r_{i-1}, a_i, r_i) \in \delta$ for each
  $i \in \{1, \dots, n\}$.
  \begin{itemize}
    \item $r$ is an \emph{accepting} run if $r_n \in F$.
    \item We say that $\mathcal{A}$ \emph{accepts} $w$ if there is an accepting
      run of $\mathcal{A}$ on $w$.
    \item The language of all strings accepted by $\mathcal{A}$ is denoted by
      $L(\mathcal{A})$.
  \end{itemize}
\end{definition}

\begin{theorem}\label{thm:dfa-nfa-equivalence}
  For every NFA $\mathcal{A}$, there is a DFA $\widehat{\mathcal{A}}$ with
  $L(\mathcal{A}) = L(\widehat{\mathcal{A}})$.
\end{theorem}
\begin{proof}
  Let $\mathcal{A} = (\Sigma, Q, q_0, F, \delta)$.
  We construct $\widehat{\mathcal{A}} = (\Sigma, \widehat{Q}, \widehat{q_0},
  \widehat{F}, \widehat{\delta})$ as follows.
  \begin{itemize}
    \item $\widehat{Q} = \mathcal{P}(Q)$.
    \item $\widehat{q_0} = \{q_0\}$.
    \item $\widehat{F} = \{\widehat{q} \in \widehat{Q}:
      q \in \widehat{q}\ \text{for some}\ q \in F\}$.
    \item $\widehat{\delta}: \widehat{Q} \times \Sigma \to \widehat{Q}$ is the
      transition function such that
      \begin{equation*}
        \widehat{\delta}(\widehat{q}, a) = \{q \in Q:
          (p, a, q) \in \delta\ \text{for some}\ p \in \widehat{q}\,\}.
      \end{equation*}
      holds for each $\widehat{q} \in \widehat{Q}$ and $a \in \Sigma$.
  \end{itemize}
  Now we prove that $L(\mathcal{A}) = L(\widehat{\mathcal{A}})$.
  For $w \in \Sigma^*$, let
  $\widehat{r} = (\widehat{r}_0, \widehat{r}_1, \dots, \widehat{r}_n)$ be
  the run of $\widehat{\mathcal{A}}$ on $w$.
  \begin{enumerate}[(i)]
    \item Suppose that $r = (r_0, r_1, \dots, r_n)$ is an accepting run of
      $\mathcal{A}$ on $w$, and we prove that $\widehat{r}$ is an accepting run
      on $w$.
      It is obvious that $r_0 \in \widehat{r}_0$.
      If $r_{i-1} \in \widehat{r}_{i-1}$ for some $i \in \{1, \dots, n\}$, then 
      we have $r_i \in \widehat\delta(\widehat{r}_{i-1}, a_i) = \widehat{r}_i$
      since $(r_{i-1}, a_i, r_i) \in \delta$.
      Thus, $r_n \in \widehat{r}_n$, and it follows that
      $\widehat{r}_n \in \widehat{F}$.
      Therefore, we have $L(\mathcal{A}) \subseteq L(\widehat{\mathcal{A}})$.
    \item Suppose that $\widehat{r}$ is an accepting run.
      Then due to the construction of $\widehat{F}$ and $\widehat\delta$, we
      can construct an accepting run $r = (r_0, r_1, \dots, r_n)$ of
      $\mathcal{A}$ on $w$ as follows.
      \begin{itemize}
        \item Let $r_n$ be a state in $\widehat{r}_n \cap F$.
        \item For $i \in \{0, \dots, n-1\}$, let $r_i$ be a state in
          $\widehat{r}_i$ such that $(r_i, a_{i+1}, r_{i+1}) \in \delta$.
      \end{itemize}
      Thus, we have $L(\widehat{\mathcal{A}}) \subseteq L(\mathcal{A})$,
      which completes the proof. \qedhere
  \end{enumerate}
\end{proof}

\section{Regular Expressions}
\begin{definition}
  Let $\Sigma$ be an alphabet.
  A \emph{regular expression} over $\Sigma$ is a string in the minimal language
  over $\Sigma \cup \{\varnothing, \epsilon, {}^*, \cdot, \cup, (, )\}$
  that satisfies the following conditions.
  \begin{enumerate}[1.]
    \item $\varnothing$ is a regular expression.
    \item $\epsilon$ is a regular expression.
    \item If $a \in \Sigma$, then $a$ is a regular expression.
    \item If $e_1$ and $e_2$ are regular expressions, then so is
      $(e_1 \cdot e_2)$.
    \item If $e_1$ and $e_2$ are regular expressions, then so is
      $(e_1 \cup e_2)$.
    \item If $e$ is a regular expression, then so is $(e)^*$.
  \end{enumerate}
\end{definition}

\begin{definition}
  A regular expression $e$ over an alphabet $\Sigma$ defines a language $L(e)$
  as follows.
  \begin{enumerate}[1.]
    \item $L(\varnothing) = \varnothing$.
    \item $L(\epsilon) = \{\epsilon\}$.
    \item $L(a) = \{a\}$ for each $a \in \Sigma$.
    \item $L((e_1 \cdot e_2)) = L(e_1) \cdot L(e_2)$ for each regular
      expressions $e_1$ and $e_2$.
    \item $L((e_1 \cup e_2)) = L(e_1) \cup L(e_2)$ for each regular
      expressions $e_1$ and $e_2$.
    \item $L((e)^*) = L(e^*)$ for each regular expression $e$.
  \end{enumerate}
\end{definition}

\begin{remark}
  We will write $(e_1e_2)$ instead of $(e_1 \cdot e_2)$ for simplification.
  Furthermore, we may omit parentheses if there is no ambiguity.
\end{remark}

\begin{theorem}
  Let $\Sigma$ be an alphabet.
  A language $L$ over $\Sigma$ is regular if and only if there is a regular
  expression $e$ over $\Sigma$ such that $L = L(e)$.
\end{theorem}
\begin{proof}
  $(\Leftarrow)$
  It holds trivially since all finite languages are regular, and regular
  languages are closed under union, cancatenation and star operations.

  $(\Rightarrow)$
  Since $L$ is regular, there is a DFA
  $\mathcal{A} = (\Sigma, Q, q_0, F, \delta)$ with $L = L(\mathcal{A})$.
  For $p, q \in Q$ and $S \subseteq Q$, define $L(p, S, q)$ to be the language
  of strings $w \in \Sigma^*$ such that the run $r$ of $w = a_1 \cdots a_n$
  on $\mathcal{A}$ with
  \begin{equation*}
    r = (r_0, r_1, \dots, r_{n-1}, r_n)
  \end{equation*}
  satisfying $r_0 = p$, $r_n = q$ and $r_i \in S$ for each
  $i \in \{1, \dots, n-1\}$.
  Then it suffices to prove that there exists a regular expression $e$ with
  $L(e) = L(p, S, q)$ for each $p, q \in Q$ and $S \subseteq Q$ since
  \begin{equation*}
    L = \bigcup_{q \in F} L(q_0, Q, q).
  \end{equation*}
  The proof is by induction on $|S|$.
  For the induction basis, let $|S| = 0$, i.e., $S = \varnothing$.
  Then we have
  \begin{equation*}
    L(p, \varnothing, p)
    = \{\epsilon\} \cup \{a \in \Sigma: \delta(p, a) = p\}
  \end{equation*}
  and
  \begin{equation*}
    L(p, \varnothing, q)
    = \{a \in \Sigma: \delta(p, a) = q\}
  \end{equation*}
  for all $p, q \in Q$ with $p \neq q$, and thus the induction basis is proved.
  Now assume the induction hypothesis that the statement holds for $|S| = k$,
  and we prove that it is true for $|S| = k + 1$.
  Let $s$ be an arbitrary state in $S$ and let $S' = S \setminus \{s\}$.
  Then it suffices to prove that
  \begin{equation*}
    L(p, S, q)
      \quad = \quad L(p, S', q)
      \quad \cup \quad
      L(p, S', s) \cdot (L(s, S', s))^* \cdot L(s, S', q).
  \end{equation*}
  since $|S'| = k$.
  \begin{enumerate}[(i)]
    \item It is obvious that 
      \begin{equation*}
        L(p, S, q)
          \quad \supseteq \quad L(p, S', q)
          \quad \cup \quad
          L(p, S', s) \cdot (L(s, S', s))^* \cdot L(s, S', q)
      \end{equation*}
      since $S = S' \cup \{s\}$.
    \item Then we prove that
      \begin{equation*}
        L(p, S, q)
          \quad \subseteq \quad L(p, S', q)
          \quad \cup \quad
          L(p, S', s) \cdot (L(s, S', s))^* \cdot L(s, S', q).
      \end{equation*}
      Suppose that $w \in L(p, S, q)$.
      Let $i_1, i_2, \dots, i_\ell$ be all indices in $\{1, \dots, n - 1\}$
      such that $r_{i_j} = s$ for each $j \in \{1, \dots, \ell\}$,
      where $i_1 < i_2 < \cdots < i_\ell$.

      If $\ell = 0$, then $w \in L(p, S', q)$ since $r_i \neq s$ for all
      $i \in \{1, \dots, n - 1\}$.
      Otherwise, we have
      \begin{align*}
        a_1 \cdots a_{i_1} &\in L(p, S', s) \\
        a_{i_1+1} \cdots a_{i_2} &\in L(s, S', s) \\
        &\vdots \\
        a_{i_{\ell-1}+1} \cdots a_{i_\ell} &\in L(s, S', s) \\
        a_{i_\ell+1} \cdots a_n &\in L(s, S', q),
      \end{align*}
      and thus $w \in L(p, S', s) \cdot (L(s, S', s))^* \cdot L(s, S', q)$.
      This completes the proof. \qedhere
  \end{enumerate}
\end{proof}