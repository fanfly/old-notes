\chapter{Context-Free Languages}
\section{Context-Free Grammars}
\begin{definition}
  A \emph{context-free grammar (CFG)} is a $4$-tuple
  \begin{equation*}
    G = (V, \Sigma, R, S),
  \end{equation*}
  where each component is as follows.
  \begin{itemize}
    \item $V$ is a finite set of symbols, called \emph{variables}.
    \item $\Sigma$ is a finite set of symbols, called \emph{terminals}, and we
    have $V \cap \Sigma = \varnothing$.
    \item $R$ is a finite set of \emph{rules}, where each rule is a pair
    $(A, \gamma)$ with $A \in V$ and $\gamma \in (V \cup \Sigma^*)$.
    \item $S \in V$ is the \emph{start variable}.
  \end{itemize}
\end{definition}

\begin{definition}
  Let $G = (V, \Sigma, R, S)$ be a CFG.
  \begin{itemize}
    \item For any variable $A \in V$ and for any strings
    $\alpha, \beta, \gamma \in (V \cup \Sigma)^*$, we say that $\alpha A \beta$
    \emph{yields} $\alpha\gamma\beta$ under $G$, denoted by
    \begin{equation*}
      \alpha A \beta \mathrel{\mathop\Rightarrow\limits_G} \alpha\gamma\beta,
    \end{equation*}
    if there is a rule $(A, \gamma) \in R$.
    \item For any strings $\alpha, \beta \in (V \cup \Sigma)^*$, we say that
    $\alpha$ \emph{derives} $\beta$ under $G$, denoted by
    \begin{equation*}
      \alpha \mathrel{\mathop\Rightarrow\limits_G^*} \beta,
    \end{equation*}
    if $\alpha = \beta$ or there exists a string $\gamma \in (V \cup \Sigma)^*$
    such that $\alpha \mathrel{\mathop\Rightarrow\limits_G^*} \gamma$ and
    $\gamma \mathrel{\mathop\Rightarrow\limits_G} \beta$.
  \end{itemize}
\end{definition}

\begin{definition}
  Let $G = (V, \Sigma, R, S)$ be a CFG.
  \begin{itemize}
    \item We say that $G$ \emph{accepts} a string $w \in \Sigma^*$ if
    \begin{equation*}
      S \mathrel{\mathop\Rightarrow\limits_G^*} w.
    \end{equation*}
    \item The \emph{language} of $G$, denoted $L(G)$, is the set of strings
    that are accepted by $G$.
  \end{itemize}
\end{definition}

\begin{definition}
  A language $L$ is \emph{context-free} if there is a CFG $G$ with $L(G) = L$.
\end{definition}

\section{Pushdown Automata}
\begin{definition}
  A \emph{pushdown automata (PDA)} is a 7-tuple
  \begin{equation*}
    A = (Q, \Sigma, \Gamma, \delta, q_0, Z, F),
  \end{equation*}
  where each component is as follows.
  \begin{itemize}
    \item $Q$ is a finite set of states.
    \item $\Sigma$ is a finite alphabet, called the input alphabet.
    \item $\Gamma$ is a finite alphabet, called the stack alphabet.
    \item $\delta \subseteq (Q \times \Sigma \times \Gamma) \times
    (Q \times \Gamma^*)$ is the transition relation.
    \item $q_0 \in Q$ is the initial state.
    \item $Z \in \Gamma$ is the initial symbol.
    \item $F \subseteq Q$ is the set of accepting states.
  \end{itemize}
\end{definition}

\begin{definition}
  Let $A = (Q, \Sigma, \Gamma, \delta, q_0, Z, F)$ be a PDA.
  For each string $w \in \Sigma^*$, we define
  $\delta_w \subseteq (Q \times \Gamma^*) \times (Q \times \Gamma^*)$ such that
  the following properties are satisfied for each $a \in \Sigma$ and
  $x \in \Sigma^*$.
  \begin{itemize}
    \item $\delta_\epsilon = \{(p, \alpha, p, \alpha) :
    \text{$p \in Q$ and $\alpha \in \Gamma^*$}\}$.
    \item $\delta_a = \{(p, X\alpha, q, \gamma\alpha) :
    \text{$(p, a, X, q, \gamma) \in \delta$ and $\alpha \in \Gamma^*$}\}$.
    \item $\delta_{xa} = \{(p, \alpha, q, \beta):
    \text{$(p, \alpha, r, \gamma) \in \delta_x$ and
    $(r, \gamma, q, \beta) \in \delta_a$
    for some $(r, \gamma) \in Q \times \Gamma^*$}\}$.
  \end{itemize}
\end{definition}

\begin{definition}
  Let $A = (Q, \Sigma, \Gamma, \delta, q_0, Z, F)$ be a PDA.
  \begin{itemize}
    \item We say that $A$ \emph{accepts} a string $w \in \Sigma^*$ if
    $(q_0, Z, q, \gamma) \in \delta_w$ for some $q \in F$ and
    $\gamma \in \Gamma^*$.
    \item The \emph{language} of $A$, denoted $L(A)$, is the set of strings
    that are accepted by $A$.
  \end{itemize}
\end{definition}