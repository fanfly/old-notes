\chapter{Context-Free Languages}
\section{Context-Free Grammars}
\begin{definition}
  A \emph{context-free grammar (CFG)} is a $4$-tuple
  \begin{equation*}
    G = (V, \Sigma, R, S),
  \end{equation*}
  where each component is as follows.
  \begin{itemize}
    \item $V$ is a finite set of \emph{variables}.
    \item $\Sigma$ is a finite set of \emph{terminals} with $V \cap \Sigma =
    \varnothing$.
    \item $R$ is a finite set of \emph{rules}, where each rule is a pair
    $(A, \gamma)$ with $A \in V$ and $\gamma \in (V \cup \Sigma)^*$.
    \item $S$ is a special variable in $V$, called the \emph{start variable}.
  \end{itemize}
\end{definition}

\begin{definition}
  Let $G = (V, \Sigma, R, S)$ be a CFG.
  \begin{itemize}
    \item For any $A \in V$ and for any $\alpha, \beta, \gamma \in (V \cup
    \Sigma)^*$, we say that $\alpha A \beta$ \emph{yields} $\alpha\gamma\beta$
    under $G$, denoted by
    \begin{equation*}
      \alpha A \beta \mathrel{\mathop\Rightarrow\limits_G} \alpha\gamma\beta,
    \end{equation*}
    if there is a rule $(A, \gamma) \in R$.
    \item For any $\alpha, \beta \in (V \cup \Sigma)^*$, we say that $\alpha$
    \emph{derives} $\beta$ under $G$, denoted by
    \begin{equation*}
      \alpha \mathrel{\mathop\Rightarrow\limits_G^*} \beta,
    \end{equation*}
    if $\alpha = \beta$, or there exists $\gamma \in (V \cup\Sigma)^*$ such
    that $\alpha \mathrel{\mathop\Rightarrow\limits_G^*}\gamma$ and
    $\gamma \mathrel{\mathop\Rightarrow\limits_G} \beta$.
  \end{itemize}
\end{definition}

\begin{definition}
  Let $G = (V, \Sigma, R, S)$ be a CFG.
  For any string $w \in \Sigma^*$, if
  \begin{equation*}
    S \mathrel{\mathop\Rightarrow\limits_G^*} w,
  \end{equation*}
  then we say that $G$ \emph{accepts} $w$.
  The set of string accepted by $G$ is called the \emph{language} of $G$,
  denoted by $L(G)$.
  A language $L$ is \emph{context-free} if $L = L(G)$ for some CFG $G$.
\end{definition}

\section{Pushdown Automata}
\begin{definition}
  A \emph{pushdown automaton (PDA)} is a 7-tuple
  \begin{equation*}
    M = (Q, \Sigma, \Gamma, \delta, s, \mathord\perp, F),
  \end{equation*}
  where each component is as follows.
  \begin{itemize}
    \item $Q$ is a finite set of \emph{states}.
    \item $\Sigma$ is a finite alphabet, called the \emph{input alphabet}.
    \item $\Gamma$ is a finite alphabet, called the \emph{stack alphabet}.
    \item $\delta \subseteq Q \times (\Sigma \cup \{\epsilon\}) \times \Gamma
    \times Q \times \Gamma^*$ is the \emph{transition relation}.
    \item $s \in Q$ is the \emph{initial state}.
    \item $\mathord\perp \in \Gamma$ is the \emph{initial stack symbol}.
    \item $F \subseteq Q$ is the set of \emph{final states}.
  \end{itemize}
\end{definition}

\begin{definition}
  A \emph{configuration} of a PDA $M = (Q, \Sigma, \Gamma, \delta, s,
  \mathord\perp, F)$ is a triple
  \begin{equation*}
    (q, w, \gamma),
  \end{equation*}
  where $q \in Q$ is the current state, $w \in \Sigma^*$ is the unprocessed
  input, and $\gamma \in \Gamma^*$ is the current stack.
  The \emph{initial configuration} of $M$ on input string $w$ is
  $(s, w, \mathord\perp)$.

  We define the single-step relation $\mathop\vdash_M$ such that for
  any $p, q \in Q$, $a \in \Sigma$, $h \in \Gamma$, $w \in \Sigma^*$ and
  $\beta, \gamma \in \Gamma^*$,
  \begin{equation*}
    (p, aw, h\gamma)
    \; \mathop\vdash\nolimits_M \;
    (q, w, \beta\gamma)
  \end{equation*}
  holds if and only if $(p, a, h, q, \beta) \in \delta$.
  The reflexive transitive closure of $\mathop\vdash_M$ is denoted by
  $\mathop\vdash_M^*$.
\end{definition}

\begin{definition}
  Let $M = (Q, \Sigma, \Gamma, \delta, s, \mathord\perp, F)$ be a
  PDA.
  A string $w \in \Sigma^*$ is \emph{accepted} by $M$ if
  \begin{equation*}
    (s, w, \mathord\perp)
    \; \mathop\vdash\nolimits_M^* \;
    (q, \epsilon, \gamma)
  \end{equation*}
  for some $q \in F$ and $\gamma \in \Gamma^*$.
  The \emph{language} $L(M)$ accepted by $M$ is defined as
  the collection of strings that are accepted by $M$.
\end{definition}

\begin{theorem}
  If $L$ is context-free, then there is a PDA $M$ that accepts $L$.
\end{theorem}
