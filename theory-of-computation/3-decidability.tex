\chapter{Decidability}
\section{Turing Machines}
\begin{definition}
  A \emph{Turing machine} is an 8-tuple
  \begin{equation*}
    M = (Q, \Sigma, \Gamma, \delta, q_0, \sqcup, q_\text{acc}, q_\text{rej}),
  \end{equation*}
  where each component is as follows.
  \begin{itemize}
    \item $Q$ is the finite set of \emph{states}.
    \item $\Sigma$ is the finite set of \emph{input symbols}.
    \item $\Gamma$ is the finite set of \emph{tape symbols} with
    $\Sigma \subseteq \Gamma$.
    \item $\delta : (Q \setminus \{q_\text{acc}, q_\text{rej}\}) \times \Gamma
    \to Q \times \Gamma \times \{-1, +1\}$ is the \emph{transition function}.
    \item $q_0 \in Q$ is the \emph{initial state}.
    \item $\sqcup \in \Gamma \setminus \Sigma$ is a special symbol, called the
    \emph{blank symbol}.
    \item $q_\text{acc}$ and $q_\text{rej}$ are different states in $Q$,
    called the \emph{accepting state} and the \emph{rejecting state},
    respectively.
  \end{itemize}
\end{definition}

\begin{definition}
  Let $M = (Q, \Sigma, \Gamma, \delta, q_0, \sqcup, q_\text{acc},
  q_\text{rej})$ be a Turing machine.
  \begin{itemize}
    \item A \emph{configuration} of $M$ is a triple in $Q \times \{1, 2,
    \dots\} \times \Gamma^*$.
    \item We define a binary relation $\vdash_M$ over $Q \times \{1, 2, \dots\}
    \times \Gamma^*$ such that for any $p, q \in Q$, $i, j \in \{1, 2, \dots
    \}$ and $u, v \in \Gamma^*$,
    \begin{equation*}
      (p, i, u)
      \quad \mathop\vdash\limits_{M} \quad
      (q, j, v)
    \end{equation*}
    if and only if the following conditions hold, where
    \begin{equation*}
      u = a_1a_2 \cdots a_n
      \quad \text{and} \quad
      v = a_1'a_2' \cdots a_m'.
    \end{equation*}
    \begin{enumerate}[1.]
      \item $\delta(p, a_i) = (q, a_i', j - i)$.
      \item Two strings $u$ and $v$ matches except for the $i$th symbol.
      That is, for each $k \in \{1, 2, \dots\} \setminus \{i\}$,
      \begin{equation*}
        \begin{array}{ll}
          a_k = a_k' & \text{if $k \leq n$ and $k \leq m$}, \\[.3em]
          a_k = \sqcup & \text{if $m < k \leq n$}, \\[.3em]
          a_k' = \sqcup & \text{if $n < k \leq m$}.
        \end{array}
      \end{equation*}
    \end{enumerate}
    \item Let $\vdash_M^{(n)}$ denote the $n$th power of $\vdash_M$, and let
    $\vdash_M^*$ denote the reflexive transitive closure of $\vdash_M$.
  \end{itemize}
\end{definition}

\begin{definition}
  Let $M = (Q, \Sigma, \Gamma, \delta, q_0, \sqcup, q_\text{acc},
  q_\text{rej})$ be a Turing machine.
  \begin{itemize}
    \item We say that $M$ \emph{accepts} a string $w \in \Sigma^*$ if $(q_0, 1,
    w) \vdash_M^* (q_\text{acc}, j, u)$ for some $j \in \{1, 2, \dots\}$ and
    $u \in \Gamma^*$.
    \item We say that $M$ \emph{rejects} a string $w \in \Sigma^*$ if $(q_0, 1,
    w) \vdash_M^* (q_\text{rej}, j, u)$ for some $j \in \{1, 2, \dots\}$ and
    $u \in \Gamma^*$.
  \end{itemize}
\end{definition}

\begin{definition}
  Let $L$ be a language over $\Sigma$.
  Let $M = (Q, \Sigma, \Gamma, \delta, q_0, \sqcup, q_\text{acc},
  q_\text{rej})$ be a Turing machine.
  \begin{itemize}
    \item We say that $M$ \emph{recognizes} $L$ if for each $w \in \Sigma^*$,
    \begin{equation*}
      w \in L
      \quad \Leftrightarrow \quad
      \text{$M$ accepts $w$}.
    \end{equation*}
    A language is \emph{recursively enumerable} if it is recognized by some
    Turing machine.
    The collection of recursively enumerable languages is denoted by
    $\mathbf{RE}$.
    \item We say that $M$ \emph{decides} $L$ if for each $w \in \Sigma^*$,
    \begin{align*}
      w \in L
      \quad &\Rightarrow \quad
      \text{$M$ accepts $w$} \\
      w \notin L
      \quad &\Rightarrow \quad
      \text{$M$ rejects $w$}.
    \end{align*}
    A language is \emph{recursive} if it is decided by some Turing machine.
    The collection of recursive languages is denoted by $\mathbf{R}$.
  \end{itemize}
\end{definition}
\begin{remark}
  Note that we have $\mathbf{R} \subseteq \mathbf{RE}$.
\end{remark}

\section{BF Programs}
\begin{definition}
  We define the language of \emph{BF programs} over the alphabet
  $\{\leftarrow, \rightarrow, +, -, \langle, \rangle\}$ as follows.
  \begin{itemize}
    \item $\leftarrow$, $\rightarrow$, $+$ and $-$ are BF programs.
    \item If $P_1$ and $P_2$ are BF programs, then so is $P_1P_2$.
    \item If $P$ is a BF program, then so is $\langle P \rangle$.
  \end{itemize}
\end{definition}

\begin{definition}
  A \emph{BF machine} is a 5-tuple
  \begin{equation*}
    M = (P, \Sigma, \Gamma, \sqcup, \sigma),
  \end{equation*}
  where each component is as follows.
  \begin{itemize}
    \item $P$ is a BF program.
    \item $\Sigma$ is the finite set of \emph{input symbols}.
    \item $\Gamma$ is the finite set of \emph{tape symbols} with
    $\Sigma \subseteq \Gamma$.
    \item $\sqcup \in \Gamma \setminus \Sigma$ is the \emph{blank symbol}.
    \item $\sigma: \Gamma \to \Gamma$ is the \emph{sucessor function} such
    that
    \begin{equation*}
      \Gamma = \bigcup_{k \geq 0} \left\{\sigma^k(\sqcup)\right\}.
    \end{equation*}
  \end{itemize}
\end{definition}