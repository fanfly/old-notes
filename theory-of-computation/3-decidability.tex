\chapter{Decidability}
\section{Turing Machines}
\begin{definition}
  A \emph{Turing machine (TM)} is an 8-tuple
  \begin{equation*}
    M = (Q, \Sigma, \Gamma, \delta, q_0, B, q_\text{acc}, q_\text{rej}),
  \end{equation*}
  where each component is as follows.
  \begin{itemize}
    \item $Q$ is the finite set of \emph{states}.
    \item $\Sigma$ is the finite set of \emph{input symbols}.
    \item $\Gamma$ is the finite set of \emph{tape symbols} with
    $\Sigma \subseteq \Gamma$.
    \item $\delta : (Q \setminus \{q_\text{acc}, q_\text{rej}\}) \times \Gamma
    \to Q \times \Gamma \times \{-1, +1\}$ is the \emph{transition function}.
    \item $q_0 \in Q$ is the \emph{initial state}.
    \item $B \in \Gamma \setminus \Sigma$ is a special symbol, called the
    \emph{blank symbol}.
    \item $q_\text{acc}$ and $q_\text{rej}$ are different states in $Q$,
    called the \emph{accepting state} and the \emph{rejecting state},
    respectively.
  \end{itemize}
\end{definition}

\begin{definition}
  Let $M = (Q, \Sigma, \Gamma, \delta, q_0, B, q_\text{acc}, q_\text{rej})$ be
  a TM.
  \begin{itemize}
    \item A \emph{configuration} of $M$ is a triple
    \begin{equation*}
      (p, i, \alpha),
    \end{equation*}
    where $p$ is a state in $Q$, $i$ is a positive integer, and $\alpha$ is
    a string over $\Gamma$.
    \item For each configuration $(p, i, a_1 \cdots a_n)$ with
    $\delta(p, a_i) = (q, c, d)$, its \emph{subsequent configuration} is
    defined as the triple
    \begin{equation*}
      (q, j, b_1 \cdots b_nB),
    \end{equation*}
    where
    \begin{equation*}
      j = \max\{1, i + d\}
    \end{equation*}
    and
    \begin{equation*}
      b_k =
      \begin{cases}
        c, &\text{if $k = i$} \\
        a_k, &\text{otherwise}
      \end{cases}
    \end{equation*}
    for each $1 \leq k \leq n$.
    \item If $(p, i, \alpha)$ is a configuration of $M$ and its subsequent
    configuration is $(q, j, \beta)$, then we write
    \begin{equation*}
      (p, i, \alpha)
      \; \mathop\vdash\limits_M \;
      (q, j, \beta).
    \end{equation*}
    Furthermore, we write
    \begin{equation*}
      (p, i, \alpha)
      \; \mathop\vdash\limits_M^* \;
      (q, j, \beta)
    \end{equation*}
    if $(p, i, \alpha) = (q, j, \beta)$ or there exists a configuration
    $(r, k, \gamma)$ of $M$ such that
    \begin{equation*}
      (p, i, \alpha)
      \; \mathop\vdash\limits_M^* \;
      (r, k, \gamma)
      \quad \text{and} \quad
      (r, k, \gamma)
      \; \mathop\vdash\limits_M \;
      (q, j, \beta).
    \end{equation*}
  \end{itemize}
\end{definition}

\begin{definition}
  Let $M = (Q, \Sigma, \Gamma, \delta, q_0, B, q_\text{acc}, q_\text{rej})$ be
  a TM.
  \begin{itemize}
    \item We say that $M$ \emph{accepts} a string $w \in \Sigma^*$ if
    \begin{equation*}
      (q_0, 1, wB)
      \; \mathop\vdash\limits_M^* \;
      (q_\text{acc}, i, \alpha)
    \end{equation*}
    for some $i \geq 1$ and $\alpha \in \Gamma^*$.
    \item We say that $M$ \emph{rejects} a string $w \in \Sigma^*$ if
    \begin{equation*}
      (q_0, 1, wB)
      \; \mathop\vdash\limits_M^* \;
      (q_\text{rej}, i, \alpha)
    \end{equation*}
    for some $i \geq 1$ and $\alpha \in \Gamma^*$. 
  \end{itemize}
\end{definition}

\begin{definition}
  Let $M = (Q, \Sigma, \Gamma, \delta, q_0, B, q_\text{acc}, q_\text{rej})$ be
  a TM and let $L$ be a language over $\Sigma$.
  \begin{itemize}
    \item We say that $M$ \emph{recognizes} $L$ if $M$ accepts $w$ for each
    $w \in L$, and $M$ does not accept $w$ for each
    $w \in \Sigma^* \setminus L$.
    \item We say that $M$ \emph{decides} $L$ if $M$ accepts $w$ for each
    $w \in L$, and $M$ rejects $w$ for each $w \in \Sigma^* \setminus L$.
  \end{itemize}
\end{definition}

\section{BF Programs}
\begin{definition}
  We define the language of \emph{BF programs} over the alphabet
  $\{L, R, U, D, \langle, \rangle\}$ as follows.
  \begin{itemize}
    \item $L$, $R$, $U$ and $D$ are BF programs.
    \item If $P_1$ and $P_2$ are BF programs, then so is $P_1P_2$.
    \item If $P$ is a BF program, then so is $\langle P \rangle$.
  \end{itemize}
\end{definition}

\begin{definition}
  A \emph{BF machine} is a 5-tuple
  \begin{equation*}
    M = (P, \Sigma, \Gamma, B, \sigma),
  \end{equation*}
  where each component is as follows.
  \begin{itemize}
    \item $P$ is a BF program.
    \item $\Sigma$ is the finite set of \emph{input symbols}.
    \item $\Gamma$ is the finite set of \emph{tape symbols} with
    $\Sigma \subseteq \Gamma$.
    \item $B \in \Gamma \setminus \Sigma$ is the \emph{blank symbol}.
    \item $\sigma: \Gamma \to \Gamma$ is the \emph{sucessor function} such
    that
    \begin{equation*}
      \Gamma = \bigcup_{k \geq 0} \left\{\sigma^k(B)\right\}.
    \end{equation*}
  \end{itemize}
\end{definition}