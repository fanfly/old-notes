\chapter{Decidability}
\section{Turing Machines}
\begin{definition}
  A \emph{Turing machine} is an 8-tuple
  \begin{equation*}
    M = (Q, \Sigma, \Gamma, \delta, q_0, \sqcup, q_\text{acc}, q_\text{rej}),
  \end{equation*}
  where each component is as follows.
  \begin{itemize}
    \item $Q$ is the finite set of \emph{states}.
    \item $\Sigma$ is the finite set of \emph{input symbols}.
    \item $\Gamma$ is the finite set of \emph{tape symbols} with
    $\Sigma \subseteq \Gamma$.
    \item $\delta : (Q \setminus \{q_\text{acc}, q_\text{rej}\}) \times \Gamma
    \to Q \times \Gamma \times \{-1, 0, +1\}$ is the
    \emph{transition function}.
    \item $q_0 \in Q$ is the \emph{initial state}.
    \item $\sqcup \in \Gamma \setminus \Sigma$ is a special symbol, called the
    \emph{blank symbol}.
    \item $q_\text{acc}$ and $q_\text{rej}$ are distinct states in $Q$,
    called the \emph{accepting state} and the \emph{rejecting state},
    respectively.
  \end{itemize}
\end{definition}

\begin{definition}
  Let $M = (Q, \Sigma, \Gamma, \delta, q_0, \sqcup, q_\text{acc},
  q_\text{rej})$ be a Turing machine.
  \begin{itemize}
    \item A \emph{configuration} of $M$ is a triple in
    $Q \times \{1, 2, \dots\} \times \Gamma^*$.
    \item We define a binary relation $\vdash_M$ over $Q \times \{1, 2, \dots\}
    \times \Gamma^*$ such that for any $p, q \in Q$, $i, j \in \{1, 2, \dots
    \}$ and $u, v \in \Gamma^*$,
    \begin{equation*}
      (p, i, u)
      \quad \mathop\vdash\limits_{M} \quad
      (q, j, v)
    \end{equation*}
    if and only if
    \begin{align*}
      u^{(1)} \cdots u^{(i-1)}u^{(i+1)} \cdots u^{(n)} \sqcup \sqcup \cdots
      \quad &= \quad
      v^{(1)} \cdots v^{(i-1)}v^{(i+1)} \cdots v^{(m)} \sqcup \sqcup \cdots \\
      \delta(p, u^{(i)}) \quad &= \quad (q, v^{(i)}, j - i).
    \end{align*}
    Let $\vdash_M^{(n)}$ denote the $n$th power of $\vdash_M$, and let
    $\mathord{\vdash_M^*} = \bigcup_{n \in \mathbb{N}} \vdash_M^{(n)}$.
  \end{itemize}
\end{definition}
\begin{remark}
  $\vdash_M$ is a partial function.
\end{remark}

\begin{definition}
  Let $M = (Q, \Sigma, \Gamma, \delta, q_0, \sqcup, q_\text{acc},
  q_\text{rej})$ be a Turing machine.
  Let $u \in \Sigma^*$.
  \begin{itemize}
    \item We say that $M$ \emph{accepts} $u$ if
    $(q_0, 1, w) \vdash_M^* (q_\text{acc}, j, v)$ for some
    $j \in \{1, 2, \dots\}$ and $v \in \Gamma^*$.
    \item We say that $M$ \emph{rejects} $u$ if
    $(q_0, 1, w) \vdash_M^* (q_\text{rej}, j, v)$ for some
    $j \in \{1, 2, \dots\}$ and $v \in \Gamma^*$.
    \item We say that $M$ \emph{halts} on input $u$ if $M$ either accepts or
    rejects $u$.
  \end{itemize}
  If $M$ halts on $u$, then we have the following definitions.
  \begin{itemize}
    \item The \emph{running time} of $M$ on input $u$ is the integer $t$ such
    that
    \begin{equation*}
      (q_0, 1, u)
      \;\; \mathop\vdash\limits_{M}^{(t)} \;\;
      (q_\text{acc}, j, v)
      \qquad \text{or} \qquad
      (q_0, 1, u)
      \;\; \mathop\vdash\limits_{M}^{(t)} \;\;
      (q_\text{rej}, j, v),
    \end{equation*}
    where $j \in \{1, 2, \dots\}$ and $v \in \Sigma^*$.
    \item The \emph{accessed space} of $M$ on input $u$ is the maximum integer
    $s$ such that
    \begin{equation*}
      (q_0, 1, u)
      \;\; \mathop\vdash\limits_{M}^* \;\;
      (q, s, v),
    \end{equation*}
    where $q \in Q$ and $v \in \Sigma^*$.
  \end{itemize}
\end{definition}

\begin{definition}
  Let $L$ be a language over $\Sigma$.
  Let $M = (Q, \Sigma, \Gamma, \delta, q_0, \sqcup, q_\text{acc},
  q_\text{rej})$ be a Turing machine.
  \begin{itemize}
    \item We say that $M$ \emph{recognizes} $L$ if for each $w \in \Sigma^*$,
    \begin{equation*}
      w \in L
      \quad \Leftrightarrow \quad
      \text{$M$ accepts $w$}.
    \end{equation*}
    A language is \emph{recursively enumerable} if it is recognized by some
    Turing machine.
    The collection of recursively enumerable languages is denoted by
    $\mathbf{RE}$.
    \item We say that $M$ \emph{decides} $L$ if for each $w \in \Sigma^*$,
    \begin{align*}
      w \in L
      \quad &\Rightarrow \quad
      \text{$M$ accepts $w$} \\
      w \notin L
      \quad &\Rightarrow \quad
      \text{$M$ rejects $w$}.
    \end{align*}
    A language is \emph{recursive} if it is decided by some Turing machine.
    The collection of recursive languages is denoted by $\mathbf{R}$.
  \end{itemize}
\end{definition}
\begin{remark}
  If $M$ decides $L$, then $M$ recognizes $L$.
  Thus, $\mathbf{R} \subseteq \mathbf{RE}$.
\end{remark}

\section{Variants of Turing Machines}
\begin{definition}
  A \emph{$k$-tape Turing machine} is
  \begin{equation*}
    M = (Q, \Sigma, \Gamma, \delta, q_0, \sqcup, q_\text{acc}, q_\text{rej}),
  \end{equation*}
  where
  \begin{equation*}
    \delta: (Q \setminus \{q_\text{acc}, q_\text{rej}\}) \times \Gamma^k
    \to Q \times \Gamma^k \times \{-1, 0, +1\}^k
  \end{equation*}
  is the \emph{transition function} and other components are the same as those
  in the definition of Turing machine.
  \begin{itemize}
    \item A \emph{configuration} of $M$ is a triple in
    $Q \times \{1, 2, \dots\}^k \times (\Gamma^*)^k$, and we define the binary
    relation $\vdash_M$ over the configurations of $M$ such that for any
    $p, q \in Q$, $i_1, \dots, i_k, j_1, \dots, j_k \in \{1, 2, \dots\}$ and
    $u_1, \dots, u_k, v_1, \dots, v_k \in \Gamma^*$,
    \begin{equation*}
      (p, (i_1, \dots, i_k), (u_1, \dots, u_k))
      \quad \mathop\vdash\limits_{M} \quad
      (q, (j_1, \dots, j_k), (v_1, \dots, v_k))
    \end{equation*}
    if and only if
    \begin{equation*}
      u_\kappa^{(1)} \cdots u_\kappa^{(i_\kappa-1)}
      u_\kappa^{(i_\kappa+1)} \cdots u_\kappa^{(n_\kappa)} \sqcup \sqcup \cdots
      \quad = \quad
      v_\kappa^{(1)} \cdots v_\kappa^{(i_\kappa-1)}
      v_\kappa^{(i_\kappa+1)} \cdots v_\kappa^{(m_\kappa)} \sqcup \sqcup \cdots
    \end{equation*}
    for all $\kappa \in \{1, \dots, k\}$ and
    \begin{equation*}
      \delta(p, (u_1^{(i_1)}, \dots, u_k^{(i_k)}))
      \quad = \quad
      (q, (v_1^{(i_1)}, \dots, v_k^{(i_k)}), (j_1 - i_1, \dots, j_k - i_k)).
    \end{equation*}
    \item We say that $M$ \emph{accepts} (resp., \emph{rejects})
    $w \in \Sigma^*$ if
    \begin{align*}
      (q_0, (1, 1, \dots, 1), (w, \epsilon, \dots, \epsilon))
      \quad &\mathop\vdash\limits_{M}^* \quad
      (q_\text{acc}, (j_1, j_2, \dots, j_k), (v_1, v_2, \dots, v_k)) \\
      \Bigl(\text{resp.,} \;
      (q_0, (1, 1, \dots, 1), (w, \epsilon, \dots, \epsilon))
      \quad &\mathop\vdash\limits_{M}^* \quad
      (q_\text{rej}, (j_1, j_2, \dots, j_k), (v_1, v_2, \dots, v_k))
      \Bigr)
    \end{align*}
    for some $j_1, j_2, \dots, j_k \in \{1, 2, \dots\}$ and
    $v_1, v_2, \dots, v_k \in \Gamma^*$.
    If $M$ either accepts or rejects $w$, then we say that $M$ \emph{halts} on
    $w$.
    \item If $M$ halts on $w \in \Sigma^*$, then the \emph{running time} of $M$
    on input $w$ is the integer $t$ with
    \begin{equation*}
      (q_0, (1, 1, \dots, 1), (w, \epsilon, \dots, \epsilon))
      \quad \mathop\vdash\limits_{M}^{(t)} \quad
      (q_\text{acc}, (j_1, j_2, \dots, j_k), (v_1, v_2, \dots, v_k))
    \end{equation*}
    or
    \begin{equation*}
      (q_0, (1, 1, \dots, 1), (w, \epsilon, \dots, \epsilon))
      \quad \mathop\vdash\limits_{M}^{(t)} \quad
      (q_\text{rej}, (j_1, j_2, \dots, j_k), (v_1, v_2, \dots, v_k)),
    \end{equation*}
    and the \emph{accessed space} of $M$ on input $w$ is the maximum sum of
    integers $s_1, \dots, s_k$ with
    \begin{equation*}
      (q_0, (1, 1, \dots, 1), (w, \epsilon, \dots, \epsilon))
      \;\; \mathop\vdash\limits_{M}^* \;\;
      (q, (j_1, j_2, \dots, j_k), (v_1, v_2, \dots, v_k)),
    \end{equation*}
    where $q \in Q$, $j_1, j_2, \dots, j_k \in \{1, 2, \dots\}$ and
    $v_1, v_2, \dots, v_k \in \Gamma^*$.
  \end{itemize}
\end{definition}

\begin{theorem}
  Every $k$-tape Turing machine has an equivalent $1$-tape Turing machine.
\end{theorem}
\begin{proof}
  Let $M = (Q, \Sigma, \Gamma, \delta, q_0, \sqcup, q_\text{acc},
  q_\text{rej})$ be a $k$-tape Turing machine.
  We show how to convert $M$ to an equivalent $1$-tape Turing machine
  $M' = (Q', \Sigma, \Gamma', \delta', q_0, \sqcup, q_\text{acc},
  q_\text{rej})$, where we use
  \begin{equation*}
    \Gamma' = \Gamma \times \{\square, \overset\bullet\square\}
    \cup \{\#\}
  \end{equation*}
  as the tape symbols.
  On input $w \in \Sigma^*$, the machine $M'$ performs the following steps:
  \begin{enumerate}[1.]
    \item Format the tape by turning its content from
    \begin{equation*}
      w_1 \cdots w_n
    \end{equation*}
    into
    \begin{equation*}
      \# \overset\bullet{w_1} \cdots w_n \sqcup \# \overset\bullet{\sqcup} \#
      \overset\bullet{\sqcup} \# \cdots \# \overset\bullet{\sqcup} \#
    \end{equation*}
    such that the tape of $M'$ can simulate the $k$ tapes of $M$.
    \item Continue performing steps 3 -- 4 until it halts.
    \item Scan the tape from the first $\#$ to the last $\#$ to determine the
    $k$ symbols under the dots.
    \item Update the tape according to the transition function $\delta$ of $M$.
    If there is any $\overset\bullet\#$, then replace it with a
    $\overset\bullet\sqcup$ and then insert a $\#$ after it.
  \end{enumerate}
  We have an implementation as follows.
  \begin{equation*}
    \delta'(q, a) =
    \begin{cases}
      ((\text{init}, 1, \overset\bullet{a}), \triangleright, +1),
      & \text{if $q = (\text{init}, 1, \triangleright)$} \\
      ((\text{init}, 1, a), b, +1),
      & \text{if $q = (\text{init}, 1, b)$ with $b \in \Gamma \cup
      \overset\bullet\Gamma$ and $a \neq \sqcup$} \\
      ((\text{init}, 1, \#), b, +1),
      & \text{if $q = (\text{init}, 1, b)$ with $b \in \Gamma \cup
      \overset\bullet\Gamma$ and $a = \sqcup$} \\
      ((\text{init}, \kappa + 1, \overset\bullet\sqcup), \#, +1),
      & \text{if $q = (\text{init}, \kappa, \#)$ with
      $1 \leq \kappa \leq k-1$} \\
      ((\text{init}, \text{back}), \#, 0),
      & \text{if $q = (\text{init}, k, \#)$} \\
      ((\text{init}, \kappa, \#), \overset\bullet\sqcup, +1),
      & \text{if $q = (\text{init}, \kappa, \overset\bullet\sqcup)$ with
      $2 \leq \kappa \leq k$} \\
      ((\text{init}, \text{back}), a, -1),
      & \text{if $q = (\text{init}, \text{back})$ and
      $a \neq \triangleright$} \\
      ((\text{scan}, p, \epsilon), \triangleright, +1),
      & \text{if $q = (\text{init}, \text{back})$ and
      $a = \triangleright$} \\
      \cdots & \\
    \end{cases}
  \end{equation*}
  It can be shown that if $M$ runs on $w$ in time $t$, then $M'$ runs on $w$
  in time $O(t^2)$.
\end{proof}

\begin{definition}
  A \emph{nondeterministic Turing machine} is
  \begin{equation*}
    M = (Q, \Sigma, \Gamma, \delta, q_0, \sqcup, q_\text{acc}, q_\text{rej}),
  \end{equation*}
  where
  \begin{equation*}
    \delta \subseteq ((Q \setminus \{q_\text{acc}, q_\text{reg}\}) \times
    \Gamma) \times (Q \times \Gamma \times \{-1, 0, +1\})
  \end{equation*}
  is the \emph{transition relation} and other components are the same as those
  in the definition of Turing machine.
  \begin{itemize}
    \item A \emph{configuration} of $M$ is a triple in
    $Q \times \{1, 2, \dots\} \times \Gamma^*$, and we define the binary
    relation $\vdash_M$ over the configurations of $M$ such that for any
    $p, q \in Q$, $i, j \in \{1, 2, \dots\}$ and $u, v \in \Gamma^*$,
    \begin{equation*}
      (p, i, u)
      \quad \mathop\vdash\limits_{M} \quad
      (q, j, v)
    \end{equation*}
    if and only if
    \begin{align*}
      u^{(1)} \cdots u^{(i-1)}u^{(i+1)} \cdots u^{(n)} \sqcup \sqcup \cdots
      \quad &= \quad
      v^{(1)} \cdots v^{(i-1)}v^{(i+1)} \cdots v^{(m)} \sqcup \sqcup \cdots \\
      ((p, u^{(i)}), (q, v^{(i)}, j - i)) \quad &\in \quad \delta.
    \end{align*}
    \item Let $u \in \Sigma^*$.
    We say that $M$ \emph{diverges} on input $u$
    (i.e., $M$ does not \emph{halt} on $u$) if for any integer $t \geq 1$ there
    exist $q \in Q$, $j \in \{1, 2, \dots\}$ and $v \in \Gamma^*$ such that
    \begin{equation*}
      (q_0, 1, u)
      \quad \mathop\vdash\limits_{M}^{(t)} \quad
      (q, j, v).
    \end{equation*}
    We say that $M$ \emph{accepts} $u$ if
    \begin{equation*}
      (q_0, 1, u)
      \quad \mathop\vdash\limits_{M}^* \quad
      (q_\text{acc}, j, v)
    \end{equation*}
    for some $j \in \{1, 2, \dots\}$ and $v \in \Gamma^*$.
    We say that $M$ \emph{rejects} $u$ if $M$ neither accepts $u$
    nor diverges on $u$.
    \item If $M$ halts on $u \in \Sigma^*$, then the \emph{running time} of $M$
    on input $u$ is the maximum integer $t$ with
    \begin{equation*}
      (q_0, 1, u)
      \quad \mathop\vdash\limits_{M}^{(t)} \quad
      (q_\text{acc}, j, v)
      \qquad \text{or} \qquad
      (q_0, 1, u)
      \quad \mathop\vdash\limits_{M}^{(t)} \quad
      (q_\text{rej}, j, v),
    \end{equation*}
    and the \emph{accessed space} of $M$ on input $u$ is the maximum integer
    $s$ with
    \begin{equation*}
      (q_0, 1, u)
      \quad \mathop\vdash\limits_{M}^* \quad
      (q, s, v)
    \end{equation*}
    where $q \in Q$, $j \in \{1, 2, \dots\}$ and $v \in \Gamma^*$.
  \end{itemize}
\end{definition}
