\chapter{Predicate Logic}
\section{The Language of Predicate Logic}
In this chapter, we reserve a countable set $\mathcal{V}$, whose elements are
called \emph{variables}.

\begin{definition}
  A \emph{vocabulary} is a pair
  \begin{equation*}
    \mathcal{L} = (\mathcal{P}, \mathcal{F}),
  \end{equation*}
  where $\mathcal{P}$ is the set of \emph{predicate symbols} and $\mathcal{F}$
  is the set of \emph{function symbols}.
  Each predicate symbol and each function symbol comes with an arity, the
  number of argument it expects.
\end{definition}

\begin{definition}
  We define \emph{terms} as follows.
  \begin{enumerate}[1.]
    \item Each variable is a term.
    \item If $f$ is an $n$-ary function symbol and $t_1, \dots, t_n$ are terms,
    then $f(t_1, \dots, t_n)$ is a term.
  \end{enumerate}
\end{definition}

\begin{definition}
  We define \emph{formulas} as follows.
  \begin{enumerate}[1.]
    \item If $P$ is an $n$-ary predicate symbol and $t_1, \dots, t_n$ are
    terms, then $P(t_1, \dots, t_n)$ is a formula.
    \item If $\alpha$ is a formula, then $\neg\alpha$ is a formula.
    \item If $\alpha$ and $\beta$ are formulas, then $(\alpha \to \beta)$ is a
    formula.
    \item If $\alpha$ is a formula and $x$ is a variable, then
    $\forall x \alpha$ is a formula.
  \end{enumerate}
\end{definition}

\section{Models}
\begin{definition}
  A \emph{model} of vocabulary $\mathcal{L} = (\mathcal{P}, \mathcal{F})$
  is a triple
  \begin{equation*}
    \mathcal{M} =
    \Bigl(
      M,
      \bigl(P^\mathcal{M}\bigr)_{P \in \mathcal{P}},
      \bigl(f^\mathcal{M}\bigr)_{f \in \mathcal{F}}
    \Bigr),
  \end{equation*}
  where each component is as follows.
  \begin{itemize}
    \item $M$ is a nonempty set called \emph{universe}.
    \item To each $n$-ary relation symbol $P$ an $n$-ary relation
    $P^\mathcal{M} \subseteq M^n$ is assigned.
    \item To each $n$-ary function symbol $f$ an $n$-ary function
    $f^\mathcal{M}: M^n \to M$ is assigned.
  \end{itemize}
\end{definition}

\begin{definition}
  Let $\mathcal{M}$ be a model of vocabulary $\mathcal{L} = (\mathcal{P},
  \mathcal{F})$.
  An \emph{object assignment} is a function $\sigma$ that maps each variable to
  an element in $M$.

  It can be extended to have its domain the set of terms such that for any
  $n$-ary function symbol $f$ and any terms $t_1, \dots, t_n$, we have
  \begin{equation*}
    \sigma(f(t_1, \dots, t_n))
    = f^\mathcal{M}(\sigma(t_1), \sigma(t_2), \dots, \sigma(t_n)).
  \end{equation*}
\end{definition}

\begin{definition}
  For any model $\mathcal{M}$ and any object assignment $\sigma$, we define
  the satisfaction relation $(\mathcal{M}, \sigma) \vDash \phi$ for each
  formula $\phi$ as follows.
  \begin{itemize}
    \item $(\mathcal{M}, \sigma) \vDash P(t_1, \dots, t_n)$ means
    $(t_1, \dots, t_n) \in P^\mathcal{M}$.
    \item $(\mathcal{M}, \sigma) \vDash \neg\alpha$ holds if and only if
    $(\mathcal{M}, \sigma) \nvDash \alpha$.
    \item $(\mathcal{M}, \sigma) \vDash (\alpha \to \beta)$ holds if and only
    if either $(\mathcal{M}, \sigma) \nvDash \alpha$ or
    $(\mathcal{M}, \sigma) \vDash \beta$.
    \item $(\mathcal{M}, \sigma) \vDash \forall x \alpha$ holds if and only if
    $(\mathcal{M}, \sigma[x \mapsto c]) \vDash \alpha$ for all $c \in M$.
  \end{itemize}
\end{definition}