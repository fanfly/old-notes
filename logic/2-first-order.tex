\chapter{First-Order Logic}
\section{The Language of First-Order Logic}
\begin{definition}
  A \emph{language} $L$ is a disjoint union of a set $L_\mathrm{rel}$ of
  \emph{relation symbols} and a set $L_\mathrm{fun}$ of
  \emph{function symbols}, where each symbol has an arity.
\end{definition}

\begin{definition}
  Let $L$ be a language.
  Let $V$ be a countably infinite set, whose elements are called
  \emph{variables}.
  A \emph{term} is a string obtained as follows.
  \begin{enumerate}
    \item Each variable is a term.
    \item If $f$ is an $n$-ary function symbol and $t_1, \dots, t_n$ are terms,
    then $f(t_1, \dots, t_n)$ is a term.
  \end{enumerate}
  A \emph{formula} is a string obtained as follows.
  \begin{enumerate}
    \item If $R$ is an $n$-ary relation symbol and $t_1, \dots, t_n$ are terms,
    then $R(t_1, \dots, t_n)$ is a formula.
    \item If $\alpha$ is a formula, then so is $\neg\alpha$.
    \item If $\alpha$ and $\beta$ are formulas, then so is
    $(\alpha \to \beta)$.
    \item If $\alpha$ is a formula and $x$ is a variable, then
    $\forall x \alpha$ is a formula.
  \end{enumerate}
\end{definition}

\section{Structures}
\begin{definition}
  Let $L$ be a language.
  A \emph{structure} for $L$ is a triple
  \begin{equation*}
    M =
    \left(U, (R^M)_{R \in L_\mathrm{rel}}, (f^M)_{f \in L_\mathrm{fun}}\right),
  \end{equation*}
  where each component is as follows.
  \begin{itemize}
    \item $U$ is a nonempty set called \emph{universe}.
    \item To each $n$-ary relation symbol $R$ an $n$-ary relation
    $R^M \subseteq U^n$ is assigned.
    \item To each $n$-ary function symbol $f$ an $n$-ary function
    $f^M: U^n \to U$ is assigned.
  \end{itemize}
\end{definition}

\begin{definition}
  Let $L$ be a language and let $M$ is a structure for $L$.
  An \emph{object assignment} is a function $\sigma: V \to U$, and it can be
  extended to have its domain the set of terms such that for any
  $n$-ary function symbol $f$ and any terms $t_1, \dots, t_n$, we have
  \begin{equation*}
    \sigma(f(t_1, \dots, t_n))
    = f^M(\sigma(t_1), \sigma(t_2), \dots, \sigma(t_n)).
  \end{equation*}
\end{definition}