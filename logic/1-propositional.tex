\chapter{Propositional Logic}
\section{The Language of Propositional Logic}
\begin{definition}
  An \emph{alphabet} for propositional logic is a pair
  $\mathcal{A} = (\mathcal{V}, \mathcal{C})$, where each component is as
  follows.
  \begin{itemize}
    \item $\mathcal{V}$ is a countably infinite set of
    \emph{propositional variables}.
    \item $\mathcal{C}$ is a finite set of \emph{connectives} with
    \begin{equation*}
      \mathcal{C} = \bigcup_{i \geq 0} \mathcal{C}_i,
    \end{equation*}
    where $\mathcal{C}_i$ is the set of connectives of arity $i$.
  \end{itemize}
\end{definition}

\begin{remark}
  In the default setting, we usually let
  \begin{align*}
    \mathcal{C}_0 &= \{\bot, \top\} \\
    \mathcal{C}_1 &= \{\neg\} \\
    \mathcal{C}_2 &= \{\wedge, \vee, \to, \leftrightarrow\}
  \end{align*}
  and $\mathcal{C}_j = \varnothing$ for $j \geq 3$.
\end{remark}

\begin{definition}
  The language $\mathcal{L}$ of \emph{formulas} over alphabet
  $\mathcal{A} = (\mathcal{V}, \mathcal{C})$ is the minimal set that satisfies
  the following statements.
  \begin{itemize}
    \item Each propositional variable in $\mathcal{V}$ is a formula.
    \item If $\star$ is a connective in $\mathcal{C}_k$ and
    $\alpha_1, \alpha_2, \dots, \alpha_k$ are formulas,
    then $\star\alpha_1\alpha_2\cdots\alpha_k$ is a formula.
  \end{itemize}
\end{definition}

\section{Truth Assignment}
\begin{definition}
  A \emph{truth assignment} is a function $\tau: \mathcal{V} \to \{0, 1\}$.
  It can be extended to $\bar\tau: \mathcal{L} \to \{0, 1\}$ by assigning each
  connective with arity $k$ to a boolean function from $\{0, 1\}^k$ to
  $\{0, 1\}$.
\end{definition}

\begin{remark}
  By convention, we use the truth table as follows.
  \begin{table}[h!]
    \centering
    \begin{tabular}{cc}
      $\bar\tau(\bot)$ & $\bar\tau(\top)$ \\
      \hline
      0 & 1
    \end{tabular}
    \qquad
    \begin{tabular}{c|c}
      $\bar\tau(\alpha)$ & $\bar\tau(\neg\alpha)$ \\
      \hline
      0 & 1 \\
      1 & 0
    \end{tabular}
    \\[1em]
    \begin{tabular}{cc|cccc}
      $\bar\tau(\alpha)$ & $\bar\tau(\beta)$
        & $\bar\tau(\alpha \wedge \beta)$
        & $\bar\tau(\alpha \vee \beta)$
        & $\bar\tau(\alpha \to \beta)$
        & $\bar\tau(\alpha \leftrightarrow \beta)$ \\
      \hline
      0 & 0 & 0 & 0 & 1 & 1 \\
      0 & 1 & 0 & 1 & 1 & 0 \\
      1 & 0 & 0 & 1 & 0 & 0 \\
      1 & 1 & 1 & 1 & 1 & 1
    \end{tabular}
    \caption{Truth Table}
    \label{table:truth-table}
  \end{table}
\end{remark}

\begin{definition}
  We say that a truth assignment $\tau$ \emph{satisfies} a formula $\alpha$
  if $\bar\tau(\alpha) = 1$.
  Also, we say that $\tau$ satisfies a set $\Sigma$ of formulas if it satisfies
  each formula in $\Sigma$.
\end{definition}

\begin{definition}
  Let $\Sigma$ be a set of formulas and let $\alpha$ be a formula.
  We say that $\Sigma$ \emph{tautologically implies} $\alpha$, denoted by
  $\Sigma \models \alpha$, if every truth assignment satisfying $\Sigma$
  also satisfies $\alpha$.
\end{definition}

\section{Proof System}
\begin{definition}
  The collection $\Lambda$ of \emph{axioms} consists of the formulas listed
  below, where $\alpha, \beta, \gamma$ are formulas.
  \begin{enumerate}[leftmargin=3.5em]
    \item[(A1)] $\alpha \to (\beta \to \alpha)$.
    \item[(A2)] $(\alpha \to (\beta \to \gamma)) \to
    ((\alpha \to \beta) \to (\alpha \to \gamma))$.
    \item[(A3)] $(\neg \beta \to \neg \alpha) \to (\alpha \to \beta)$.
  \end{enumerate}
\end{definition}

\begin{definition}
  A \emph{proof} of a formula $\alpha$ from a collection $\Gamma$ of formulas
  is a sequence of formulas
  \begin{equation*}
    (\alpha_1, \alpha_2, \dots, \alpha_n)
  \end{equation*}
  satisfying the following properties.
  \begin{enumerate}
    \item $\alpha_n = \alpha$.
    \item For $k \in \{1, 2, \dots, n\}$, either
    $\alpha_k \in \Lambda \cup \Gamma$ or there exist
    $i, j \in \{1, 2, \dots, k-1\}$ with $\alpha_j = \alpha_i \to \alpha_k$.
  \end{enumerate}
  If there exists a proof of $\varphi$ from $\Gamma$, we write
  $\Gamma \vdash \varphi$.
  If $\varnothing \vdash \varphi$, we write $\vdash \varphi$ for short.
\end{definition}

\begin{theorem}
  \label{thm:identity}
  For any formula $\alpha$, we have $\vdash \alpha \to \alpha$.
\end{theorem}
\begin{proof}
  We have a proof of $\alpha \to \alpha$ as follows.
  \begin{enumerate}[1.]
    \item $(\alpha \to ((\alpha \to \alpha) \to \alpha)) \to
    ((\alpha \to (\alpha \to \alpha)) \to (\alpha \to \alpha))$.
    \hfill (A2)
    \item $\alpha \to ((\alpha \to \alpha) \to \alpha)$. \hfill (A1)
    \item $(\alpha \to (\alpha \to \alpha)) \to (\alpha \to \alpha)$.
    \hfill (1, 2)
    \item $\alpha \to (\alpha \to \alpha)$. \hfill (A1)
    \item $\alpha \to \alpha$. \hfill (3, 4)
  \end{enumerate}
  Thus, we can conclude that $\vdash \alpha \to \alpha$.
\end{proof}

\begin{proposition}
  Let $\Gamma$ and $\Delta$ be sets of formulas and let $\alpha$ be a formula.
  If $\Gamma \vdash \alpha$ and $\Gamma \subseteq \Delta$, then
  $\Delta \vdash \alpha$.
\end{proposition}
\begin{proof}
  To be completed.
\end{proof}

\begin{theorem}[Deduction Theorem]
  \label{thm:deduction}
  Let $\Gamma$ be a set of formulas and let $\alpha$ and $\beta$ be formulas.
  If $\Gamma \cup \{\alpha\} \vdash \beta$, then
  $\Gamma \vdash \alpha \to \beta$.
\end{theorem}
\begin{proof}
  If $\beta \in \Lambda \cup \Gamma$, then we have
  $\Gamma \vdash \alpha \to \beta_k$ since
  $\vdash \beta_k \to (\alpha \to \beta_k)$.
  Furthermore, if $\beta = \alpha$, then we also have
  $\Gamma \vdash \alpha \to \beta$ since $\vdash \beta \to \beta$ by
  \Cref{thm:identity}.
  Thus, one only needs to consider the case that
  $\beta \notin \Lambda \cup \Gamma \cup \{\alpha\}$.

  Suppose that $(\beta_1, \beta_2, \dots, \beta_n)$ is a proof of $\beta$ from
  $\Gamma \cup \{\alpha\}$.
  For $1 \leq k \leq n$, we prove that $\Gamma \vdash \alpha \to \beta_k$ by
  induction on $k$.
  The induction basis holds for $k = 1$ since
  $\beta_1 \in \Lambda \cup \Gamma \cup \{\alpha\}$.
  For the induction step, let $k \geq 2$ and assume that
  $\Gamma \vdash \alpha \to \beta_\ell$ for $1 \leq \ell < k$.
  We have proved for the case that
  $\beta \in \Lambda \cup \Gamma \cup \{\alpha\}$, and thus we assume without
  loss of generality that there exist $1 \leq i < k$ and $1 \leq j < k$
  such that $\beta_j = \beta_i \to \beta_k$.
  Note that $\Gamma \vdash \alpha \to \beta_i$ and
  $\Gamma \vdash \alpha \to (\beta_i \to \beta_k)$
  hold by induction hypothesis.
  Therefore, since
  \begin{equation*}
    \vdash (\alpha \to (\beta_i \to \beta_k)) \to
    ((\alpha \to \beta_i) \to (\alpha \to \beta_k)),
  \end{equation*}
  we can conclude that $\Gamma \vdash \alpha \to \beta_k$,
  which completes the proof.
\end{proof}

\begin{theorem}
  \label{thm:contradiction}
  For any formula $\alpha$ and $\beta$, we have
  $\vdash \neg\alpha \to (\alpha \to \beta)$.
\end{theorem}
\begin{proof}
  We have a proof of $\neg\alpha \to (\alpha \to \beta)$ as follows.
  \begin{enumerate}[1.]
    \item $((\neg\beta \to \neg\alpha) \to (\alpha \to \beta))
    \to (\neg\alpha \to ((\neg\beta \to \neg\alpha) \to (\alpha \to \beta)))$.
    \hfill (A1)
    \item $(\neg\beta \to \neg\alpha) \to (\alpha \to \beta)$. \hfill (A3)
    \item $\neg\alpha \to ((\neg\beta \to \neg\alpha) \to (\alpha \to \beta))$.
    \hfill (1, 2)
    \item $(\neg\alpha \to ((\neg\beta \to \neg\alpha) \to (\alpha \to \beta)))
    \to ((\neg\alpha \to (\neg\beta \to \neg\alpha))
    \to (\neg\alpha \to (\alpha \to \beta)))$. \\ \hbox{} \hfill (A2)
    \item $(\neg\alpha \to (\neg\beta \to \neg\alpha))
    \to (\neg\alpha \to (\alpha \to \beta))$. \hfill (3, 4)
    \item $\neg\alpha \to (\neg\beta \to \neg\alpha)$. \hfill (A1)
    \item $\neg\alpha \to (\alpha \to \beta)$. \hfill (5, 6)
  \end{enumerate}
  Thus, we can conclude that $\vdash \neg\alpha \to (\alpha \to \beta)$.
\end{proof}