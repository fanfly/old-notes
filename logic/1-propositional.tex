\chapter{Propositional Logic}
\section{The Language of Propositional Logic}
In this chapter, we reserve a countable set $\mathcal{A}$, whose elements are
called \emph{propositional variables}.

\begin{definition}
  We define \emph{formulas} as follows.
  \begin{enumerate}[1.]
    \item Each propositional variable is a formula.
    \item If $\alpha$ is a formula, then $\neg\alpha$ is a formula.
    \item If $\alpha$ and $\beta$ are formulas, then $(\alpha \to \beta)$ is a
    formula.
  \end{enumerate}
\end{definition}

\section{Truth Assignments}
\begin{definition}
  A \emph{truth assignment} is a function $\tau: \mathcal{A} \to \{0, 1\}$, and
  it can be extended to have its domain the set of formulas such that
  \begin{equation*}
    \tau(\neg\alpha) =
    \begin{cases}
      0, & \text{if $\tau(\alpha) = 1$} \\
      1, & \text{if $\tau(\alpha) = 0$}
    \end{cases}
    \quad \text{and} \quad
    \tau(\alpha \to \beta) =
    \begin{cases}
      0, & \text{if $\tau(\alpha) = 1$ and $\tau(\beta) = 0$} \\
      1, & \text{otherwise}
    \end{cases}
  \end{equation*}
  for any formula $\alpha$ and $\beta$.

  We say that $\tau$ \emph{satisfies} a formula $\alpha$, denoted by
  $\tau \vDash \alpha$, if $\tau(\alpha) = 1$.
\end{definition}

\begin{definition}
  Let $\Gamma$ be a set of formulas and let $\alpha$ be a formula.
  We say that $\Gamma$ \emph{tautologically implies} $\alpha$, denoted by
  $\Gamma \vDash \alpha$, if every truth assignment satisfying $\Gamma$
  also satisfies $\alpha$.
\end{definition}

\section{The Proof System}
\begin{definition}
  The collection $\Lambda$ of \emph{axioms} consists of the formulas listed
  below, where $\alpha, \beta, \gamma$ are formulas.
  \begin{enumerate}[leftmargin=3.5em]
    \item[(A1)] $\alpha \to (\beta \to \alpha)$.
    \item[(A2)] $(\alpha \to (\beta \to \gamma)) \to
    ((\alpha \to \beta) \to (\alpha \to \gamma))$.
    \item[(A3)] $(\neg \beta \to \neg \alpha) \to (\alpha \to \beta)$.
  \end{enumerate}
\end{definition}

\begin{definition}
  A \emph{proof} of a formula $\alpha$ from a collection $\Gamma$ of formulas
  is a sequence of formulas
  \begin{equation*}
    (\alpha_1, \alpha_2, \dots, \alpha_n)
  \end{equation*}
  satisfying the following properties.
  \begin{enumerate}
    \item $\alpha_n = \alpha$.
    \item For $k \in \{1, 2, \dots, n\}$, either
    $\alpha_k \in \Lambda \cup \Gamma$ or there exist
    $i, j \in \{1, 2, \dots, k-1\}$ with $\alpha_j = \alpha_i \to \alpha_k$.
  \end{enumerate}
  If there exists a proof of $\varphi$ from $\Gamma$, we write
  $\Gamma \vdash \varphi$.
  If $\varnothing \vdash \varphi$, we write $\vdash \varphi$ for short.
\end{definition}

\begin{theorem}[Law of Identity]
  \label{thm:identity}
  For any formula $\alpha$, we have $\vdash \alpha \to \alpha$.
\end{theorem}
\begin{proof}
  We have a proof of $\alpha \to \alpha$ as follows.
  \begin{enumerate}[(1)]
    \item $(\alpha \to ((\alpha \to \alpha) \to \alpha)) \to
    ((\alpha \to (\alpha \to \alpha)) \to (\alpha \to \alpha))$.
    \hfill (A2)
    \item $\alpha \to ((\alpha \to \alpha) \to \alpha)$. \hfill (A1)
    \item $(\alpha \to (\alpha \to \alpha)) \to (\alpha \to \alpha)$.
    \hfill (1, 2)
    \item $\alpha \to (\alpha \to \alpha)$. \hfill (A1)
    \item $\alpha \to \alpha$. \hfill (3, 4)
  \end{enumerate}
  Thus, we can conclude that $\vdash \alpha \to \alpha$.
\end{proof}

\begin{theorem}[Duns Scotus Law]
  \label{thm:contradiction}
  For any formula $\alpha$ and $\beta$, we have
  $\vdash \neg\alpha \to (\alpha \to \beta)$.
\end{theorem}
\begin{proof}
  We have a proof of $\neg\alpha \to (\alpha \to \beta)$ as follows.
  \begin{enumerate}[(1)]
    \item $((\neg\beta \to \neg\alpha) \to (\alpha \to \beta))
    \to (\neg\alpha \to ((\neg\beta \to \neg\alpha) \to (\alpha \to \beta)))$.
    \hfill (A1)
    \item $(\neg\beta \to \neg\alpha) \to (\alpha \to \beta)$. \hfill (A3)
    \item $\neg\alpha \to ((\neg\beta \to \neg\alpha) \to (\alpha \to \beta))$.
    \hfill (1, 2)
    \item $(\neg\alpha \to ((\neg\beta \to \neg\alpha) \to (\alpha \to \beta)))
    \to ((\neg\alpha \to (\neg\beta \to \neg\alpha))
    \to (\neg\alpha \to (\alpha \to \beta)))$. \\ \hbox{} \hfill (A2)
    \item $(\neg\alpha \to (\neg\beta \to \neg\alpha))
    \to (\neg\alpha \to (\alpha \to \beta))$. \hfill (3, 4)
    \item $\neg\alpha \to (\neg\beta \to \neg\alpha)$. \hfill (A1)
    \item $\neg\alpha \to (\alpha \to \beta)$. \hfill (5, 6)
  \end{enumerate}
  Thus, we can conclude that $\vdash \neg\alpha \to (\alpha \to \beta)$.
\end{proof}

\begin{theorem}[Modus Ponens]
  \label{thm:modus-ponens}
  For any formula $\alpha$ and $\beta$, we have
  $\vdash \alpha \to ((\alpha \to \beta) \to \beta)$.
\end{theorem}
\begin{proof}
  We have a proof of $\alpha \to ((\alpha \to \beta) \to \beta)$ as follows.
  \begin{enumerate}[(1)]
    \item $(\alpha \to \beta) \to (\alpha \to \beta)$.
    \hfill (\Cref{thm:identity})
    \item $((\alpha \to \beta) \to (\alpha \to \beta)) \to
    (((\alpha \to \beta) \to \alpha) \to ((\alpha \to \beta) \to \beta))$.
    \hfill (A2)
    \item $((\alpha \to \beta) \to \alpha) \to ((\alpha \to \beta) \to \beta)$.
    \hfill (1, 2)
    \item $(((\alpha \to \beta) \to \alpha) \to ((\alpha \to \beta) \to \beta))
    \to (\alpha \to (((\alpha \to \beta) \to \alpha) \to ((\alpha \to \beta)
    \to \beta)))$. \hfill (A1)
    \item $\alpha \to (((\alpha \to \beta) \to \alpha) \to ((\alpha \to \beta)
    \to \beta))$. \hfill (3, 4)
    \item $(\alpha \to (((\alpha \to \beta) \to \alpha) \to ((\alpha \to \beta)
    \to \beta))) \to ((\alpha \to ((\alpha \to \beta) \to \alpha)) \to (\alpha
    \to ((\alpha \to \beta) \to \beta)))$. \hfill (A2)
    \item $(\alpha \to ((\alpha \to \beta) \to \alpha)) \to (\alpha \to
    ((\alpha \to \beta) \to \beta))$. \hfill (5, 6)
    \item $\alpha \to ((\alpha \to \beta) \to \alpha)$. \hfill (A1)
    \item $\alpha \to ((\alpha \to \beta) \to \beta)$. \hfill (7, 8)
  \end{enumerate}
  Thus, we can conclude that
  $\vdash \alpha \to ((\alpha \to \beta) \to \beta)$.
\end{proof}

\begin{theorem}[Hypothetical Syllogism]
  \label{thm:syl}
  For any formulas $\alpha$, $\beta$ and $\gamma$, we have
  $\vdash (\beta \to \gamma) \to ((\alpha \to \beta) \to (\alpha \to \gamma))$.
\end{theorem}
\begin{proof}
  We have a proof of $(\beta \to \gamma) \to ((\alpha \to \beta) \to (\alpha
  \to \gamma))$ as follows.
  \begin{enumerate}[(1)]
    \item $(\alpha \to (\beta \to \gamma)) \to ((\alpha \to \beta) \to
    (\alpha \to \gamma))$. \hfill (A2)
    \item $((\alpha \to (\beta \to \gamma)) \to ((\alpha \to \beta) \to
    (\alpha \to \gamma))) \to ((\beta \to \gamma) \to ((\alpha \to (\beta \to
    \gamma)) \to ((\alpha \to \beta) \to (\alpha \to \gamma))))$. \hfill (A1)
    \item $(\beta \to \gamma) \to ((\alpha \to (\beta \to \gamma)) \to
    ((\alpha \to \beta) \to (\alpha \to \gamma)))$. \hfill (1, 2)
    \item $((\beta \to \gamma) \to ((\alpha \to (\beta \to \gamma)) \to
    ((\alpha \to \beta) \to (\alpha \to \gamma)))) \to (((\beta \to \gamma) \to
    (\alpha \to (\beta \to \gamma))) \to ((\beta \to \gamma) \to
    ((\alpha \to \beta) \to (\alpha \to \gamma))))$. \hfill (A2)
    \item $((\beta \to \gamma) \to (\alpha \to (\beta \to \gamma))) \to ((\beta
    \to \gamma) \to ((\alpha \to \beta) \to (\alpha \to \gamma)))$.
    \hfill (3, 4)
    \item $(\beta \to \gamma) \to (\alpha \to (\beta \to \gamma))$. \hfill (A1)
    \item $(\beta \to \gamma) \to ((\alpha \to \beta) \to
    (\alpha \to \gamma))$. \hfill (5, 6)
  \end{enumerate}
  Thus, we can conclude that
  $\vdash (\beta \to \gamma) \to ((\alpha \to \beta) \to (\alpha \to \gamma))$.
\end{proof}

\begin{theorem}[Clavius's Law]
  \label{thm:clavius}
  For any formula $\alpha$, we have $\vdash (\neg\alpha \to \alpha) \to
  \alpha$.
\end{theorem}
\begin{proof}
  We have a proof of $(\neg\alpha \to \alpha) \to \alpha$ as follows.
  \begin{enumerate}[(1)]
    \item $(\neg\alpha \to (\alpha \to \neg(\neg\alpha \to \alpha))) \to
    ((\neg\alpha \to \alpha) \to (\neg\alpha \to \neg(\neg\alpha \to
    \alpha)))$. \hfill (A2)
    \item $\neg\alpha \to (\alpha \to \neg(\neg\alpha \to \alpha))$.
    \hfill (\Cref{thm:contradiction})
    \item $(\neg\alpha \to \alpha) \to (\neg\alpha \to \neg(\neg\alpha \to
    \alpha))$. \hfill (1, 2)
    \item $(\neg\alpha \to \neg(\neg\alpha \to \alpha)) \to ((\neg\alpha \to
    \alpha) \to \alpha)$. \hfill (A3)
    \item $((\neg\alpha \to \neg(\neg\alpha \to \alpha)) \to ((\neg\alpha \to
    \alpha) \to \alpha)) \to (((\neg\alpha \to \alpha) \to (\neg\alpha \to
    \neg(\neg\alpha \to \alpha))) \to ((\neg\alpha \to \alpha) \to ((\neg\alpha
    \to \alpha) \to \alpha)))$. \hfill (\Cref{thm:syl})
    \item $((\neg\alpha \to \alpha) \to (\neg\alpha \to \neg(\neg\alpha \to
    \alpha))) \to ((\neg\alpha \to \alpha) \to ((\neg\alpha \to \alpha) \to
    \alpha))$. \hfill (4, 5)
    \item $(\neg\alpha \to \alpha) \to ((\neg\alpha \to \alpha) \to \alpha)$.
    \hfill (3, 6)
    \item $((\neg\alpha \to \alpha) \to ((\neg\alpha \to \alpha) \to \alpha))
    \to (((\neg\alpha \to \alpha) \to (\neg\alpha \to \alpha)) \to
    ((\neg\alpha \to \alpha) \to \alpha))$. \hfill (A2)
    \item $((\neg\alpha \to \alpha) \to (\neg\alpha \to \alpha)) \to
    ((\neg\alpha \to \alpha) \to \alpha)$ \hfill (7, 8)
    \item $(\neg\alpha \to \alpha) \to (\neg\alpha \to \alpha)$.
    \hfill (\Cref{thm:identity})
    \item $(\neg\alpha \to \alpha) \to \alpha$. \hfill (9, 10)
  \end{enumerate}
  Thus, we can conclude that $\vdash (\neg\alpha \to \alpha) \to \alpha$.
\end{proof}

\begin{theorem}[Elimination of Double Negation]
  \label{thm:dn-elim}
  For any formula $\alpha$, we have $\vdash \neg\neg\alpha \to \alpha$.
\end{theorem}
\begin{proof}
  We have a proof of $\neg\neg\alpha \to \alpha$ as follows.
  \begin{enumerate}[(1)]
    \item $((\neg\alpha \to \alpha) \to \alpha) \to ((\neg\neg\alpha \to
    (\neg\alpha \to \alpha)) \to (\neg\neg\alpha \to \alpha))$.
    \hfill (\Cref{thm:syl})
    \item $(\neg\alpha \to \alpha) \to \alpha$. \hfill (\Cref{thm:clavius})
    \item $(\neg\neg\alpha \to (\neg\alpha \to \alpha)) \to (\neg\neg\alpha \to
    \alpha)$. \hfill (1, 2)
    \item $\neg\neg\alpha \to (\neg\alpha \to \alpha)$.
    \hfill (\Cref{thm:contradiction})
    \item $\neg\neg\alpha \to \alpha$. \hfill (3, 4)
  \end{enumerate}
  Thus, we can conclude that $\vdash \neg\neg\alpha \to \alpha$.
\end{proof}

\begin{theorem}[Introduction of Double Negation]
  \label{thm:dn-intro}
  For any formula $\alpha$, we have $\vdash \alpha \to \neg\neg\alpha$.
\end{theorem}
\begin{proof}
  We have a proof of $\alpha \to \neg\neg\alpha$ as follows.
  \begin{enumerate}[(1)]
    \item $(\neg\neg\neg\alpha \to \neg\alpha) \to (\alpha \to
    \neg\neg\alpha)$. \hfill (A3)
    \item $\neg\neg\neg\alpha \to \neg\alpha$. \hfill (\Cref{thm:dn-elim})
    \item $\alpha \to \neg\neg\alpha$. \hfill (1, 2)
  \end{enumerate}
  Thus, we can conclude that $\vdash \alpha \to \neg\neg\alpha$.
\end{proof}

\begin{theorem}[Law of Contraposition]
  \label{thm:contraposition}
  For any formulas $\alpha$ and $\beta$, we have $\vdash (\alpha \to \beta) \to
  (\neg\beta \to \neg\alpha)$.
\end{theorem}
\begin{proof}
  We have a proof of $(\alpha \to \beta) \to (\neg\beta \to \neg\alpha)$
  as follows.
  \begin{enumerate}[(1)]
    \item $(\beta \to \neg\neg\beta) \to ((\neg\neg\alpha \to \beta) \to
    (\neg\neg\alpha \to \neg\neg\beta))$. \hfill (\Cref{thm:syl})
    \item $\beta \to \neg\neg\beta$. \hfill (\Cref{thm:dn-intro})
    \item $(\neg\neg\alpha \to \beta) \to (\neg\neg\alpha \to \neg\neg\beta)$.
    \hfill (1, 2)
    \item $((\neg\neg\alpha \to \beta) \to (\neg\neg\alpha \to \neg\neg\beta))
    \to ((\alpha \to \beta) \to ((\neg\neg\alpha \to \beta) \to (\neg\neg\alpha
    \to \neg\neg\beta)))$. \hfill (A1)
    \item $(\alpha \to \beta) \to ((\neg\neg\alpha \to \beta) \to
    (\neg\neg\alpha \to \neg\neg\beta))$. \hfill (3, 4)
    \item $(\alpha \to \beta) \to ((\neg\neg\alpha \to \alpha) \to
    (\neg\neg\alpha \to \beta))$. \hfill (\Cref{thm:syl})
    \item $((\alpha \to \beta) \to ((\neg\neg\alpha \to \alpha) \to
    (\neg\neg\alpha \to \beta))) \to (((\alpha \to \beta) \to (\neg\neg\alpha
    \to \alpha)) \to ((\alpha \to \beta) \to (\neg\neg\alpha \to \beta)))$.
    \hfill (A2)
    \item $((\alpha \to \beta) \to (\neg\neg\alpha \to \alpha)) \to ((\alpha
    \to \beta) \to (\neg\neg\alpha \to \beta))$. \hfill (6, 7)
    \item $(\neg\neg\alpha \to \alpha) \to ((\alpha \to \beta) \to
    (\neg\neg\alpha \to \alpha))$. \hfill (A1)
    \item $\neg\neg\alpha \to \alpha$. \hfill (\Cref{thm:dn-elim})
    \item $(\alpha \to \beta) \to (\neg\neg\alpha \to \alpha)$. \hfill (9, 10)
    \item $(\alpha \to \beta) \to (\neg\neg\alpha \to \beta)$. \hfill (8, 11)
    \item $((\alpha \to \beta) \to ((\neg\neg\alpha \to \beta) \to
    (\neg\neg\alpha \to \neg\neg\beta))) \to (((\alpha \to \beta) \to
    (\neg\neg\alpha \to \beta)) \to ((\alpha \to \beta) \to (\neg\neg\alpha \to
    \neg\neg\beta)))$. \hfill (A2)
    \item $((\alpha \to \beta) \to (\neg\neg\alpha \to \beta)) \to ((\alpha \to
    \beta) \to (\neg\neg\alpha \to \neg\neg\beta))$. \hfill (5, 13)
    \item $(\alpha \to \beta) \to (\neg\neg\alpha \to \neg\neg\beta)$.
    \hfill (12, 14)
    \item $((\neg\neg\alpha \to \neg\neg\beta) \to (\neg\beta \to \neg\alpha))
    \to (((\alpha \to \beta) \to (\neg\neg\alpha \to \neg\neg\beta)) \to
    ((\alpha \to \beta) \to (\neg\beta \to \neg\alpha)))$.
    \hfill (\Cref{thm:syl})
    \item $(\neg\neg\alpha \to \neg\neg\beta) \to (\neg\beta \to \neg\alpha)$.
    \hfill (A3)
    \item $((\alpha \to \beta) \to (\neg\neg\alpha \to \neg\neg\beta)) \to
    ((\alpha \to \beta) \to (\neg\beta \to \neg\alpha))$. \hfill (16, 17)
    \item $(\alpha \to \beta) \to (\neg\beta \to \neg\alpha)$. \hfill (15, 18)
  \end{enumerate}
  Thus, we can conclude that
  $\vdash (\alpha \to \beta) \to (\neg\beta \to \neg\alpha)$.
\end{proof}

\begin{theorem}
  For any formulas $\alpha$ and $\beta$, we have $\vdash \alpha \to (\neg\beta
  \to \neg(\alpha \to \beta))$.
\end{theorem}
\begin{proof}
  We have a proof of $\alpha \to (\neg\beta \to \neg(\alpha \to \beta))$
  as follows.
  \begin{enumerate}[(1)]
    \item $((\alpha \to \beta) \to \beta) \to (\neg\beta \to \neg(\alpha \to
    \beta))$. \hfill (\Cref{thm:contraposition})
    \item $(((\alpha \to \beta) \to \beta) \to (\neg\beta \to \neg(\alpha \to
    \beta))) \to (\alpha \to (((\alpha \to \beta) \to \beta) \to (\neg\beta \to
    \neg(\alpha \to \beta))))$. \hfill (A1)
    \item $\alpha \to (((\alpha \to \beta) \to \beta) \to (\neg\beta \to
    \neg(\alpha \to \beta)))$. \hfill (1, 2)
    \item $(\alpha \to (((\alpha \to \beta) \to \beta) \to (\neg\beta \to
    (\neg(\alpha \to \beta))))) \to ((\alpha \to ((\alpha \to \beta) \to
    \beta)) \to (\alpha \to (\neg\beta \to (\neg(\alpha \to \beta)))))$.
    \hfill (A2)
    \item $(\alpha \to ((\alpha \to \beta) \to \beta)) \to (\alpha \to
    (\neg\beta \to (\neg(\alpha \to \beta))))$. \hfill (3, 4)
    \item $\alpha \to ((\alpha \to \beta) \to \beta)$. \hfill
    (\Cref{thm:modus-ponens})
    \item $\alpha \to (\neg\beta \to \neg(\alpha \to \beta))$. \hfill (5, 6)
  \end{enumerate}
  Thus, we can conclude that $\vdash \alpha \to (\neg\beta \to \neg(\alpha \to
  \beta))$.
\end{proof}

\begin{theorem}[Deduction Theorem]
  \label{thm:deduction}
  Let $\Gamma$ be a set of formulas and let $\alpha$ and $\beta$ be formulas.
  If $\Gamma \cup \{\alpha\} \vdash \beta$, then
  $\Gamma \vdash \alpha \to \beta$.
\end{theorem}
\begin{proof}
  If $\beta \in \Lambda \cup \Gamma$, then we have
  $\Gamma \vdash \alpha \to \beta_k$ since
  $\vdash \beta_k \to (\alpha \to \beta_k)$.
  Furthermore, if $\beta = \alpha$, then we also have
  $\Gamma \vdash \alpha \to \beta$ since $\vdash \beta \to \beta$ by
  \Cref{thm:identity}.
  Thus, one only needs to consider the case that
  $\beta \notin \Lambda \cup \Gamma \cup \{\alpha\}$.

  Suppose that $(\beta_1, \beta_2, \dots, \beta_n)$ is a proof of $\beta$ from
  $\Gamma \cup \{\alpha\}$.
  For $1 \leq k \leq n$, we prove that $\Gamma \vdash \alpha \to \beta_k$ by
  induction on $k$.
  The induction basis holds for $k = 1$ since
  $\beta_1 \in \Lambda \cup \Gamma \cup \{\alpha\}$.
  For the induction step, let $k \geq 2$ and assume that
  $\Gamma \vdash \alpha \to \beta_\ell$ for $1 \leq \ell < k$.
  We have proved for the case that
  $\beta \in \Lambda \cup \Gamma \cup \{\alpha\}$, and thus we assume without
  loss of generality that there exist $1 \leq i < k$ and $1 \leq j < k$
  such that $\beta_j = \beta_i \to \beta_k$.
  Note that $\Gamma \vdash \alpha \to \beta_i$ and
  $\Gamma \vdash \alpha \to (\beta_i \to \beta_k)$
  hold by induction hypothesis.
  Therefore, since
  \begin{equation*}
    \vdash (\alpha \to (\beta_i \to \beta_k)) \to
    ((\alpha \to \beta_i) \to (\alpha \to \beta_k)),
  \end{equation*}
  we can conclude that $\Gamma \vdash \alpha \to \beta_k$,
  which completes the proof.
\end{proof}

\section{Completeness}
\begin{lemma}
  \label{thm:completeness-lemma}
  Let $\tau$ be a truth assignment.
  For each formula $\phi$, we define
  \begin{equation*}
    \phi^{(\tau)} =
    \begin{cases}
      \phi, & \text{if $\tau(\phi) = 1$} \\
      \neg \phi, & \text{if $\tau(\phi) = 0$}.
    \end{cases}
  \end{equation*}
  Then for any formula $\alpha$ that consists of only the propositional
  variables $p_1, \dots, p_k$, we have
  \begin{equation*}
    \Bigl\{p_1^{(\tau)}, \dots, p_k^{(\tau)}\Bigr\} \vdash \alpha^{(\tau)}.
  \end{equation*}
\end{lemma}
\begin{proof}
  Let
  \begin{equation*}
    \Pi = \Bigl\{p_1^{(\tau)}, \dots, p_k^{(\tau)}\Bigr\}.
  \end{equation*}
  The proof is by induction.
  If $\alpha$ is atomic, i.e., $\alpha = p_i$ for some $i \in \{1, \dots, k\}$,
  then we have $\alpha^{(\tau)} \in \Pi$, and thus $\Pi \vdash
  \alpha^{(\tau)}$.

  Now suppose that $\Pi \vdash \alpha^{(\tau)}$, and we prove that $\Pi \vdash
  \beta^{(\tau)}$ with $\beta = \neg\alpha$.

  \textbf{Case 1.}
  If $\tau(\alpha) = 0$, then $\tau(\beta) = 1$, and it follows that
  $\alpha^{(\tau)} = \neg\alpha = \beta^{(\tau)}$, implying
  $\Pi \vdash \beta^{(\tau)}$.

  \textbf{Case 2.}
  If $\tau(\alpha) = 1$, then $\tau(\beta) = 0$, and it follows that
  $\alpha^{(\tau)} = \alpha$ and $\beta^{(\tau)} = \neg\neg\alpha$.
  Since $\Pi \vdash \alpha$ and $\vdash \alpha \to \neg\neg\alpha$, we have
  $\Pi \vdash \neg\neg\alpha$, implying $\Pi \vdash \beta^{(\tau)}$.

  Now suppose that $\Pi \vdash \alpha^{(\tau)}$ and $\Pi \vdash
  \beta^{(\tau)}$, and we prove that $\Pi \vdash \gamma^{(\tau)}$ with
  $\gamma = \alpha \to \beta$.

  \textbf{Case 1.}
  If $\tau(\alpha) = 0$, then $\tau(\gamma) = 1$, and it follows that
  $\alpha^{(\tau)} = \neg\alpha$ and $\gamma^{(\tau)} = \alpha \to \beta$.
  Since $\Pi \vdash \neg\alpha$, and $\vdash \neg\alpha \to (\alpha \to
  \beta)$, we have $\Pi \vdash \alpha \to \beta$, implying $\Pi \vdash
  \gamma^{(\tau)}$.

  \textbf{Case 2.}
  If $\tau(\beta) = 1$, then $\tau(\gamma) = 1$, and it follows that
  $\beta^{(\tau)} = \beta$ and $\gamma^{(\tau)} = \alpha \to \beta$.
  Since $\Pi \vdash \beta$ and $\beta \vdash (\alpha \to \beta)$, we have
  $\Pi \vdash \alpha \to \beta$, implying $\Pi \vdash \gamma^{(\tau)}$.
  
  \textbf{Case 3.}
  If $\tau(\alpha) = 1$ and $\tau(\beta) = 0$, then $\tau(\gamma) = 0$, and it
  follows that $\alpha^{(\tau)} = \alpha$, $\beta^{(\tau)} = \neg\beta$, and
  $\gamma^{(\tau)} = \neg(\alpha \to \beta)$.
  Since $\Pi \vdash \alpha$, $\Pi \vdash \neg\beta$ and $\vdash \alpha \to
  (\neg\beta \to \neg(\alpha \to \beta))$, we have $\Pi \vdash \neg(\alpha \to
  \beta)$, implying $\Pi \vdash \gamma^{(\tau)}$.
\end{proof}

\begin{theorem}[Completness Theorem]
  For each formula $\alpha$, $\vDash \alpha$ implies $\vdash \alpha$.
\end{theorem}
\begin{proof}
  By \Cref{thm:completeness-lemma},
  \begin{equation*}
    p_1^{(\tau)}, p_2^{(\tau)}, \dots, p_k^{(\tau)} \vdash \alpha
  \end{equation*}
  holds for any truth assignment $\tau$, where $p_1, p_2, \dots, p_k$ are the
  propositional variables that appears in $\alpha$.

  Now suppose that $\tau$ is a truth assignment and $p_1, \dots, p_j$ are
  propositional variables such that
  \begin{equation*}
    p_1^{(\tau)}, \dots, p_{j-1}^{(\tau)}, p_j \vdash \alpha
    \quad \text{and} \quad
    p_1^{(\tau)}, \dots, p_{j-1}^{(\tau)}, \neg p_j \vdash \alpha.
  \end{equation*}
  Then we have
  \begin{equation*}
    p_1^{(\tau)}, \dots, p_{j-1}^{(\tau)} \vdash p_j \to \alpha
    \quad \text{and} \quad
    p_1^{(\tau)}, \dots, p_{j-1}^{(\tau)} \vdash \neg p_j \to \alpha.
  \end{equation*}
  Since $\vdash (p_j \to \alpha) \to ((\neg p_j \to \alpha) \to \alpha)$,
  it follows that
  \begin{equation*}
    p_1^{(\tau)}, \dots, p_{j-1}^{(\tau)} \vdash \alpha.
  \end{equation*}
  This process can be performed continually such that all the premises are
  eliminated.
  Thus, we conclude that $\vdash \alpha$.
\end{proof}