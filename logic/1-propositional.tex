\chapter{Propositional Logic}
\section{The Language of Propositional Logic}
\begin{definition}
  An \emph{alphabet} for propositional logic is a pair
  $\mathcal{A} = (\mathcal{V}, \mathcal{C})$, where each component is as
  follows.
  \begin{itemize}
    \item $\mathcal{V}$ is a countably infinite set of
    \emph{propositional variables}.
    \item $\mathcal{C}$ is a finite set of \emph{connectives} with
    \begin{equation*}
      \mathcal{C} = \bigcup_{i \geq 0} \mathcal{C}_i,
    \end{equation*}
    where $\mathcal{C}_i$ is the set of connectives of arity $i$.
  \end{itemize}
\end{definition}

\begin{remark}
  In the default setting, we usually let
  \begin{align*}
    \mathcal{C}_0 &= \{\bot, \top\} \\
    \mathcal{C}_1 &= \{\neg\} \\
    \mathcal{C}_2 &= \{\wedge, \vee, \to, \leftrightarrow\}
  \end{align*}
  and $\mathcal{C}_j = \varnothing$ for $j \geq 3$.
\end{remark}

\begin{definition}
  The language $\mathcal{L}$ of \emph{formulas} over alphabet
  $\mathcal{A} = (\mathcal{V}, \mathcal{C})$ is the minimal set that satisfies
  the following statements.
  \begin{itemize}
    \item Each propositional variable in $\mathcal{V}$ is a formula.
    \item If $\star$ is a connective in $\mathcal{C}_k$ and
    $\alpha_1, \alpha_2, \dots, \alpha_k$ are formulas,
    then $\star\alpha_1\alpha_2\cdots\alpha_k$ is a formula.
  \end{itemize}
\end{definition}

\section{Truth Assignment}
\begin{definition}
  A \emph{truth assignment} is a function $\tau: \mathcal{V} \to \{0, 1\}$.
  It can be extended to $\bar\tau: \mathcal{L} \to \{0, 1\}$ by assigning each
  connective with arity $k$ to a boolean function from $\{0, 1\}^k$ to
  $\{0, 1\}$.
\end{definition}

\begin{remark}
  By convention, we use the truth table as follows.
  \begin{table}[h!]
    \centering
    \begin{tabular}{cc}
      $\bar\tau(\bot)$ & $\bar\tau(\top)$ \\
      \hline
      0 & 1
    \end{tabular}
    \qquad
    \begin{tabular}{c|c}
      $\bar\tau(\alpha)$ & $\bar\tau(\neg\alpha)$ \\
      \hline
      0 & 1 \\
      1 & 0
    \end{tabular}
    \\[1em]
    \begin{tabular}{cc|cccc}
      $\bar\tau(\alpha)$ & $\bar\tau(\beta)$
        & $\bar\tau(\alpha \wedge \beta)$
        & $\bar\tau(\alpha \vee \beta)$
        & $\bar\tau(\alpha \to \beta)$
        & $\bar\tau(\alpha \leftrightarrow \beta)$ \\
      \hline
      0 & 0 & 0 & 0 & 1 & 1 \\
      0 & 1 & 0 & 1 & 1 & 0 \\
      1 & 0 & 0 & 1 & 0 & 0 \\
      1 & 1 & 1 & 1 & 1 & 1
    \end{tabular}
    \caption{Truth Table}
    \label{table:truth-table}
  \end{table}
\end{remark}

\begin{definition}
  We say that a truth assignment $\tau$ \emph{satisfies} a formula $\alpha$
  if $\bar\tau(\alpha) = 1$.
  Also, we say that $\tau$ satisfies a set $\Sigma$ of formulas if it satisfies
  each formula in $\Sigma$.
\end{definition}

\begin{definition}
  Let $\Sigma$ be a set of formulas and let $\alpha$ be a formula.
  We say that $\Sigma$ \emph{tautologically implies} $\alpha$, denoted by
  $\Sigma \models \alpha$, if every truth assignment satisfying $\Sigma$
  also satisfies $\alpha$.
\end{definition}

\section{Proof System}
\begin{definition}
  The collection $\Lambda$ of \emph{axioms} consists of the formulas listed
  below, where $\alpha, \beta, \gamma$ are formulas.
  \begin{enumerate}[leftmargin=3.5em]
    \item[(A1)] $\alpha \to (\beta \to \alpha)$.
    \item[(A2)] $(\alpha \to (\beta \to \gamma)) \to
    ((\alpha \to \beta) \to (\alpha \to \gamma))$.
    \item[(A3)] $(\neg \beta \to \neg \alpha) \to (\alpha \to \beta)$.
  \end{enumerate}
\end{definition}

\begin{definition}
  A \emph{proof} of a formula $\alpha$ from a collection $\Gamma$ of formulas
  is a sequence of formulas
  \begin{equation*}
    (\alpha_1, \alpha_2, \dots, \alpha_n)
  \end{equation*}
  satisfying the following properties.
  \begin{enumerate}
    \item $\alpha_n = \alpha$.
    \item For $k \in \{1, 2, \dots, n\}$, either
    $\alpha_k \in \Lambda \cup \Gamma$ or there exist
    $i, j \in \{1, 2, \dots, k-1\}$ with $\alpha_j = \alpha_i \to \alpha_k$.
  \end{enumerate}
  If there exists a proof of $\varphi$ from $\Gamma$, we write
  $\Gamma \vdash \varphi$.
  If $\varnothing \vdash \varphi$, we write $\vdash \varphi$ for short.
\end{definition}

\begin{theorem}[Law of Identity]
  \label{thm:identity}
  For any formula $\alpha$, we have $\vdash \alpha \to \alpha$.
\end{theorem}
\begin{proof}
  We have a proof of $\alpha \to \alpha$ as follows.
  \begin{enumerate}[(1)]
    \item $(\alpha \to ((\alpha \to \alpha) \to \alpha)) \to
    ((\alpha \to (\alpha \to \alpha)) \to (\alpha \to \alpha))$.
    \hfill (A2)
    \item $\alpha \to ((\alpha \to \alpha) \to \alpha)$. \hfill (A1)
    \item $(\alpha \to (\alpha \to \alpha)) \to (\alpha \to \alpha)$.
    \hfill (1, 2)
    \item $\alpha \to (\alpha \to \alpha)$. \hfill (A1)
    \item $\alpha \to \alpha$. \hfill (3, 4)
  \end{enumerate}
  Thus, we can conclude that $\vdash \alpha \to \alpha$.
\end{proof}

\begin{theorem}[Duns Scotus Law]
  \label{thm:contradiction}
  For any formula $\alpha$ and $\beta$, we have
  $\vdash \neg\alpha \to (\alpha \to \beta)$.
\end{theorem}
\begin{proof}
  We have a proof of $\neg\alpha \to (\alpha \to \beta)$ as follows.
  \begin{enumerate}[(1)]
    \item $((\neg\beta \to \neg\alpha) \to (\alpha \to \beta))
    \to (\neg\alpha \to ((\neg\beta \to \neg\alpha) \to (\alpha \to \beta)))$.
    \hfill (A1)
    \item $(\neg\beta \to \neg\alpha) \to (\alpha \to \beta)$. \hfill (A3)
    \item $\neg\alpha \to ((\neg\beta \to \neg\alpha) \to (\alpha \to \beta))$.
    \hfill (1, 2)
    \item $(\neg\alpha \to ((\neg\beta \to \neg\alpha) \to (\alpha \to \beta)))
    \to ((\neg\alpha \to (\neg\beta \to \neg\alpha))
    \to (\neg\alpha \to (\alpha \to \beta)))$. \\ \hbox{} \hfill (A2)
    \item $(\neg\alpha \to (\neg\beta \to \neg\alpha))
    \to (\neg\alpha \to (\alpha \to \beta))$. \hfill (3, 4)
    \item $\neg\alpha \to (\neg\beta \to \neg\alpha)$. \hfill (A1)
    \item $\neg\alpha \to (\alpha \to \beta)$. \hfill (5, 6)
  \end{enumerate}
  Thus, we can conclude that $\vdash \neg\alpha \to (\alpha \to \beta)$.
\end{proof}

\begin{theorem}[Modus Ponens]
  \label{thm:modus-ponens}
  For any formula $\alpha$ and $\beta$, we have
  $\vdash \alpha \to ((\alpha \to \beta) \to \beta)$.
\end{theorem}
\begin{proof}
  We have a proof of $\alpha \to ((\alpha \to \beta) \to \beta)$ as follows.
  \begin{enumerate}[(1)]
    \item $(\alpha \to \beta) \to (\alpha \to \beta)$.
    \hfill (\Cref{thm:identity})
    \item $((\alpha \to \beta) \to (\alpha \to \beta)) \to
    (((\alpha \to \beta) \to \alpha) \to ((\alpha \to \beta) \to \beta))$.
    \hfill (A2)
    \item $((\alpha \to \beta) \to \alpha) \to ((\alpha \to \beta) \to \beta)$.
    \hfill (1, 2)
    \item $(((\alpha \to \beta) \to \alpha) \to ((\alpha \to \beta) \to \beta))
    \to (\alpha \to (((\alpha \to \beta) \to \alpha) \to ((\alpha \to \beta)
    \to \beta)))$. \hfill (A1)
    \item $\alpha \to (((\alpha \to \beta) \to \alpha) \to ((\alpha \to \beta)
    \to \beta))$. \hfill (3, 4)
    \item $(\alpha \to (((\alpha \to \beta) \to \alpha) \to ((\alpha \to \beta)
    \to \beta))) \to ((\alpha \to ((\alpha \to \beta) \to \alpha)) \to (\alpha
    \to ((\alpha \to \beta) \to \beta)))$. \hfill (A2)
    \item $(\alpha \to ((\alpha \to \beta) \to \alpha)) \to (\alpha \to
    ((\alpha \to \beta) \to \beta))$. \hfill (5, 6)
    \item $\alpha \to ((\alpha \to \beta) \to \alpha)$. \hfill (A1)
    \item $\alpha \to ((\alpha \to \beta) \to \beta)$. \hfill (7, 8)
  \end{enumerate}
  Thus, we can conclude that
  $\vdash \alpha \to ((\alpha \to \beta) \to \beta)$.
\end{proof}

\begin{theorem}[Hypothetical Syllogism]
  \label{thm:syl}
  For any formulas $\alpha$, $\beta$ and $\gamma$, we have
  $\vdash (\beta \to \gamma) \to ((\alpha \to \beta) \to (\alpha \to \gamma))$.
\end{theorem}
\begin{proof}
  We have a proof of $(\beta \to \gamma) \to ((\alpha \to \beta) \to (\alpha
  \to \gamma))$ as follows.
  \begin{enumerate}[(1)]
    \item $(\alpha \to (\beta \to \gamma)) \to ((\alpha \to \beta) \to
    (\alpha \to \gamma))$. \hfill (A2)
    \item $((\alpha \to (\beta \to \gamma)) \to ((\alpha \to \beta) \to
    (\alpha \to \gamma))) \to ((\beta \to \gamma) \to ((\alpha \to (\beta \to
    \gamma)) \to ((\alpha \to \beta) \to (\alpha \to \gamma))))$. \hfill (A1)
    \item $(\beta \to \gamma) \to ((\alpha \to (\beta \to \gamma)) \to
    ((\alpha \to \beta) \to (\alpha \to \gamma)))$. \hfill (1, 2)
    \item $((\beta \to \gamma) \to ((\alpha \to (\beta \to \gamma)) \to
    ((\alpha \to \beta) \to (\alpha \to \gamma)))) \to (((\beta \to \gamma) \to
    (\alpha \to (\beta \to \gamma))) \to ((\beta \to \gamma) \to
    ((\alpha \to \beta) \to (\alpha \to \gamma))))$. \hfill (A2)
    \item $((\beta \to \gamma) \to (\alpha \to (\beta \to \gamma))) \to ((\beta
    \to \gamma) \to ((\alpha \to \beta) \to (\alpha \to \gamma)))$.
    \hfill (3, 4)
    \item $(\beta \to \gamma) \to (\alpha \to (\beta \to \gamma))$. \hfill (A1)
    \item $(\beta \to \gamma) \to ((\alpha \to \beta) \to
    (\alpha \to \gamma))$. \hfill (5, 6)
  \end{enumerate}
  Thus, we can conclude that
  $\vdash (\beta \to \gamma) \to ((\alpha \to \beta) \to (\alpha \to \gamma))$.
\end{proof}

\begin{theorem}[Clavius's Law]
  \label{thm:clavius}
  For any formula $\alpha$, we have $\vdash (\neg\alpha \to \alpha) \to
  \alpha$.
\end{theorem}
\begin{proof}
  We have a proof of $(\neg\alpha \to \alpha) \to \alpha$ as follows.
  \begin{enumerate}[(1)]
    \item $(\neg\alpha \to (\alpha \to \neg(\neg\alpha \to \alpha))) \to
    ((\neg\alpha \to \alpha) \to (\neg\alpha \to \neg(\neg\alpha \to
    \alpha)))$. \hfill (A2)
    \item $\neg\alpha \to (\alpha \to \neg(\neg\alpha \to \alpha))$.
    \hfill (\Cref{thm:contradiction})
    \item $(\neg\alpha \to \alpha) \to (\neg\alpha \to \neg(\neg\alpha \to
    \alpha))$. \hfill (1, 2)
    \item $(\neg\alpha \to \neg(\neg\alpha \to \alpha)) \to ((\neg\alpha \to
    \alpha) \to \alpha)$. \hfill (A3)
    \item $((\neg\alpha \to \neg(\neg\alpha \to \alpha)) \to ((\neg\alpha \to
    \alpha) \to \alpha)) \to (((\neg\alpha \to \alpha) \to (\neg\alpha \to
    \neg(\neg\alpha \to \alpha))) \to ((\neg\alpha \to \alpha) \to ((\neg\alpha
    \to \alpha) \to \alpha)))$. \hfill (\Cref{thm:syl})
    \item $((\neg\alpha \to \alpha) \to (\neg\alpha \to \neg(\neg\alpha \to
    \alpha))) \to ((\neg\alpha \to \alpha) \to ((\neg\alpha \to \alpha) \to
    \alpha))$. \hfill (4, 5)
    \item $(\neg\alpha \to \alpha) \to ((\neg\alpha \to \alpha) \to \alpha)$.
    \hfill (3, 6)
    \item $((\neg\alpha \to \alpha) \to ((\neg\alpha \to \alpha) \to \alpha))
    \to (((\neg\alpha \to \alpha) \to (\neg\alpha \to \alpha)) \to
    ((\neg\alpha \to \alpha) \to \alpha))$. \hfill (A2)
    \item $((\neg\alpha \to \alpha) \to (\neg\alpha \to \alpha)) \to
    ((\neg\alpha \to \alpha) \to \alpha)$ \hfill (7, 8)
    \item $(\neg\alpha \to \alpha) \to (\neg\alpha \to \alpha)$.
    \hfill (\Cref{thm:identity})
    \item $(\neg\alpha \to \alpha) \to \alpha$. \hfill (9, 10)
  \end{enumerate}
  Thus, we can conclude that $\vdash (\neg\alpha \to \alpha) \to \alpha$.
\end{proof}

\begin{theorem}[Elimination of Double Negation]
  \label{thm:dn-elim}
  For any formula $\alpha$, we have $\vdash \neg\neg\alpha \to \alpha$.
\end{theorem}
\begin{proof}
  We have a proof of $\neg\neg\alpha \to \alpha$ as follows.
  \begin{enumerate}[(1)]
    \item $((\neg\alpha \to \alpha) \to \alpha) \to ((\neg\neg\alpha \to
    (\neg\alpha \to \alpha)) \to (\neg\neg\alpha \to \alpha))$.
    \hfill (\Cref{thm:syl})
    \item $(\neg\alpha \to \alpha) \to \alpha$. \hfill (\Cref{thm:clavius})
    \item $(\neg\neg\alpha \to (\neg\alpha \to \alpha)) \to (\neg\neg\alpha \to
    \alpha)$. \hfill (1, 2)
    \item $\neg\neg\alpha \to (\neg\alpha \to \alpha)$.
    \hfill (\Cref{thm:contradiction})
    \item $\neg\neg\alpha \to \alpha$. \hfill (3, 4)
  \end{enumerate}
  Thus, we can conclude that $\vdash \neg\neg\alpha \to \alpha$.
\end{proof}

\begin{theorem}[Introduction of Double Negation]
  \label{thm:dn-intro}
  For any formula $\alpha$, we have $\vdash \alpha \to \neg\neg\alpha$.
\end{theorem}
\begin{proof}
  We have a proof of $\alpha \to \neg\neg\alpha$ as follows.
  \begin{enumerate}[(1)]
    \item $(\neg\neg\neg\alpha \to \neg\alpha) \to (\alpha \to
    \neg\neg\alpha)$. \hfill (A3)
    \item $\neg\neg\neg\alpha \to \neg\alpha$. \hfill (\Cref{thm:dn-elim})
    \item $\alpha \to \neg\neg\alpha$. \hfill (1, 2)
  \end{enumerate}
  Thus, we can conclude that $\vdash \alpha \to \neg\neg\alpha$.
\end{proof}

\begin{theorem}[Law of Contraposition]
  \label{thm:contraposition}
  For any formulas $\alpha$ and $\beta$, we have $\vdash (\alpha \to \beta) \to
  (\neg\beta \to \neg\alpha)$.
\end{theorem}
\begin{proof}
  We have a proof of $(\alpha \to \beta) \to (\neg\beta \to \neg\alpha)$
  as follows.
  \begin{enumerate}[(1)]
    \item $(\beta \to \neg\neg\beta) \to ((\neg\neg\alpha \to \beta) \to
    (\neg\neg\alpha \to \neg\neg\beta))$. \hfill (\Cref{thm:syl})
    \item $\beta \to \neg\neg\beta$. \hfill (\Cref{thm:dn-intro})
    \item $(\neg\neg\alpha \to \beta) \to (\neg\neg\alpha \to \neg\neg\beta)$.
    \hfill (1, 2)
    \item $((\neg\neg\alpha \to \beta) \to (\neg\neg\alpha \to \neg\neg\beta))
    \to ((\alpha \to \beta) \to ((\neg\neg\alpha \to \beta) \to (\neg\neg\alpha
    \to \neg\neg\beta)))$. \hfill (A1)
    \item $(\alpha \to \beta) \to ((\neg\neg\alpha \to \beta) \to
    (\neg\neg\alpha \to \neg\neg\beta))$. \hfill (3, 4)
    \item $(\alpha \to \beta) \to ((\neg\neg\alpha \to \alpha) \to
    (\neg\neg\alpha \to \beta))$. \hfill (\Cref{thm:syl})
    \item $((\alpha \to \beta) \to ((\neg\neg\alpha \to \alpha) \to
    (\neg\neg\alpha \to \beta))) \to (((\alpha \to \beta) \to (\neg\neg\alpha
    \to \alpha)) \to ((\alpha \to \beta) \to (\neg\neg\alpha \to \beta)))$.
    \hfill (A2)
    \item $((\alpha \to \beta) \to (\neg\neg\alpha \to \alpha)) \to ((\alpha
    \to \beta) \to (\neg\neg\alpha \to \beta))$. \hfill (6, 7)
    \item $(\neg\neg\alpha \to \alpha) \to ((\alpha \to \beta) \to
    (\neg\neg\alpha \to \alpha))$. \hfill (A1)
    \item $\neg\neg\alpha \to \alpha$. \hfill (\Cref{thm:dn-elim})
    \item $(\alpha \to \beta) \to (\neg\neg\alpha \to \alpha)$. \hfill (9, 10)
    \item $(\alpha \to \beta) \to (\neg\neg\alpha \to \beta)$. \hfill (8, 11)
    \item $((\alpha \to \beta) \to ((\neg\neg\alpha \to \beta) \to
    (\neg\neg\alpha \to \neg\neg\beta))) \to (((\alpha \to \beta) \to
    (\neg\neg\alpha \to \beta)) \to ((\alpha \to \beta) \to (\neg\neg\alpha \to
    \neg\neg\beta)))$. \hfill (A2)
    \item $((\alpha \to \beta) \to (\neg\neg\alpha \to \beta)) \to ((\alpha \to
    \beta) \to (\neg\neg\alpha \to \neg\neg\beta))$. \hfill (5, 13)
    \item $(\alpha \to \beta) \to (\neg\neg\alpha \to \neg\neg\beta)$.
    \hfill (12, 14)
    \item $((\neg\neg\alpha \to \neg\neg\beta) \to (\neg\beta \to \neg\alpha))
    \to (((\alpha \to \beta) \to (\neg\neg\alpha \to \neg\neg\beta)) \to
    ((\alpha \to \beta) \to (\neg\beta \to \neg\alpha)))$.
    \hfill (\Cref{thm:syl})
    \item $(\neg\neg\alpha \to \neg\neg\beta) \to (\neg\beta \to \neg\alpha)$.
    \hfill (A3)
    \item $((\alpha \to \beta) \to (\neg\neg\alpha \to \neg\neg\beta)) \to
    ((\alpha \to \beta) \to (\neg\beta \to \neg\alpha))$. \hfill (16, 17)
    \item $(\alpha \to \beta) \to (\neg\beta \to \neg\alpha)$. \hfill (15, 18)
  \end{enumerate}
  Thus, we can conclude that
  $\vdash (\alpha \to \beta) \to (\neg\beta \to \neg\alpha)$.
\end{proof}

\begin{theorem}
  For any formulas $\alpha$ and $\beta$, we have $\vdash \alpha \to (\neg\beta
  \to \neg(\alpha \to \beta))$.
\end{theorem}
\begin{proof}
  We have a proof of $\alpha \to (\neg\beta \to \neg(\alpha \to \beta))$
  as follows.
  \begin{enumerate}[(1)]
    \item $((\alpha \to \beta) \to \beta) \to (\neg\beta \to \neg(\alpha \to
    \beta))$. \hfill (\Cref{thm:contraposition})
    \item $(((\alpha \to \beta) \to \beta) \to (\neg\beta \to \neg(\alpha \to
    \beta))) \to (\alpha \to (((\alpha \to \beta) \to \beta) \to (\neg\beta \to
    \neg(\alpha \to \beta))))$. \hfill (A1)
    \item $\alpha \to (((\alpha \to \beta) \to \beta) \to (\neg\beta \to
    \neg(\alpha \to \beta)))$. \hfill (1, 2)
    \item $(\alpha \to (((\alpha \to \beta) \to \beta) \to (\neg\beta \to
    (\neg(\alpha \to \beta))))) \to ((\alpha \to ((\alpha \to \beta) \to
    \beta)) \to (\alpha \to (\neg\beta \to (\neg(\alpha \to \beta)))))$.
    \hfill (A2)
    \item $(\alpha \to ((\alpha \to \beta) \to \beta)) \to (\alpha \to
    (\neg\beta \to (\neg(\alpha \to \beta))))$. \hfill (3, 4)
    \item $\alpha \to ((\alpha \to \beta) \to \beta)$. \hfill
    (\Cref{thm:modus-ponens})
    \item $\alpha \to (\neg\beta \to \neg(\alpha \to \beta))$. \hfill (5, 6)
  \end{enumerate}
  Thus, we can conclude that $\vdash \alpha \to (\neg\beta \to \neg(\alpha \to
  \beta))$.
\end{proof}

\begin{theorem}[Deduction Theorem]
  \label{thm:deduction}
  Let $\Gamma$ be a set of formulas and let $\alpha$ and $\beta$ be formulas.
  If $\Gamma \cup \{\alpha\} \vdash \beta$, then
  $\Gamma \vdash \alpha \to \beta$.
\end{theorem}
\begin{proof}
  If $\beta \in \Lambda \cup \Gamma$, then we have
  $\Gamma \vdash \alpha \to \beta_k$ since
  $\vdash \beta_k \to (\alpha \to \beta_k)$.
  Furthermore, if $\beta = \alpha$, then we also have
  $\Gamma \vdash \alpha \to \beta$ since $\vdash \beta \to \beta$ by
  \Cref{thm:identity}.
  Thus, one only needs to consider the case that
  $\beta \notin \Lambda \cup \Gamma \cup \{\alpha\}$.

  Suppose that $(\beta_1, \beta_2, \dots, \beta_n)$ is a proof of $\beta$ from
  $\Gamma \cup \{\alpha\}$.
  For $1 \leq k \leq n$, we prove that $\Gamma \vdash \alpha \to \beta_k$ by
  induction on $k$.
  The induction basis holds for $k = 1$ since
  $\beta_1 \in \Lambda \cup \Gamma \cup \{\alpha\}$.
  For the induction step, let $k \geq 2$ and assume that
  $\Gamma \vdash \alpha \to \beta_\ell$ for $1 \leq \ell < k$.
  We have proved for the case that
  $\beta \in \Lambda \cup \Gamma \cup \{\alpha\}$, and thus we assume without
  loss of generality that there exist $1 \leq i < k$ and $1 \leq j < k$
  such that $\beta_j = \beta_i \to \beta_k$.
  Note that $\Gamma \vdash \alpha \to \beta_i$ and
  $\Gamma \vdash \alpha \to (\beta_i \to \beta_k)$
  hold by induction hypothesis.
  Therefore, since
  \begin{equation*}
    \vdash (\alpha \to (\beta_i \to \beta_k)) \to
    ((\alpha \to \beta_i) \to (\alpha \to \beta_k)),
  \end{equation*}
  we can conclude that $\Gamma \vdash \alpha \to \beta_k$,
  which completes the proof.
\end{proof}

\section{Soundness and Completeness}
\begin{theorem}
  Let $\alpha$ be a formula which consists of only the propositional variables
  $p_1, \dots, p_k$ and let $\tau$ be a truth assignment.
  Let $p_1^*, \dots, p_k^*$ be formulas such that for each
  $i \in \{1, \dots, k\}$,
  \begin{equation*}
    p_i^* =
    \begin{cases}
      p_i, & \text{if $\tau(p_i) = 1$} \\
      \neg p_i, & \text{if $\tau(p_i) = 0$}.
    \end{cases}
  \end{equation*}
  Furthermore, let $\alpha^*$ be the formula defined by
  \begin{equation*}
    \alpha^* =
    \begin{cases}
      \alpha, & \text{if $\bar\tau(\alpha) = 1$} \\
      \neg \alpha, & \text{if $\bar\tau(\alpha) = 0$}.
    \end{cases}
  \end{equation*}
  Then we have
  \begin{equation*}
    \{p_1^*, \dots, p_k^*\} \vdash \alpha^*.
  \end{equation*}
\end{theorem}
\begin{proof}
  The proof is by induction on the complexity of $\alpha$.
  It is straightforward that the theorem holds when $\alpha = p_i$ for some
  $i \in \{1, \dots, k\}$.

  Now suppose that $\{p_1^*, \dots, p_k^*\} \vdash \alpha^*$, and we prove
  that
  \begin{equation*}
    \{p_1^*, \dots, p_k^*\} \vdash \beta^*
  \end{equation*}
  with $\beta = \neg\alpha$.
  If $\bar\tau(\alpha) = 0$, then $\bar\tau(\beta) = 1$, and we have
  $\alpha^* = \neg\alpha = \beta^*$.
  Thus, $\{p_1^*, \dots, p_k^*\} \vdash \beta^*$.
  If $\bar\tau(\alpha) = 1$, then $\bar\tau(\beta) = 0$, and we have
  $\alpha^* = \alpha$ and $\beta^* = \neg\neg\alpha$.
  Since
  \begin{equation*}
    \{p_1^*, \dots, p_k^*\} \vdash \alpha
    \quad \text{and} \quad
    \vdash \alpha \to \neg\neg\alpha,
  \end{equation*}
  we have $\{p_1^*, \dots, p_k^*\} \vdash \beta^*$.

  Now suppose that $\{p_1^*, \dots, p_k^*\} \vdash \alpha^*$ and
  $\{p_1^*, \dots, p_k^*\} \vdash \beta^*$, and we prove that
  \begin{equation*}
    \{p_1^*, \dots, p_k^*\} \vdash \gamma^*
  \end{equation*}
  with $\gamma = \alpha \to \beta$.
  If $\bar\tau(\alpha) = 0$, then $\bar\tau(\gamma) = 1$, and we have
  $\alpha^* = \neg\alpha$ and $\gamma^* = \alpha \to \beta$.
  Since $\{p_1^*, \dots, p_k^*\} \vdash \neg\alpha$
  \begin{equation*}
    \{p_1^*, \dots, p_k^*\} \vdash \neg\alpha
    \quad \text{and} \quad
    \vdash \neg\alpha \to (\alpha \to \beta)
  \end{equation*}
  we have $\{p_1^*, \dots, p_k^*\} \vdash \gamma^*$.
  If $\bar\tau(\beta) = 1$, then $\bar\tau(\gamma) = 1$, and we have
  $\beta^* = \beta$ and $\gamma^* = \alpha \to \beta$.
  Since
  \begin{equation*}
    \{p_1^*, \dots, p_k^*\} \vdash \beta
    \quad \text{and} \quad
    \vdash \beta \to (\alpha \to \beta)
  \end{equation*}
  we have $\{p_1^*, \dots, p_k^*\} \vdash \gamma^*$.
  If $\bar\tau(\alpha) = 1$ and $\bar\tau(\beta) = 0$, then
  $\bar\tau(\gamma) = 0$, and we have $\alpha^* = \alpha$, 
  $\beta^* = \neg\beta$ and $\gamma^* = \neg(\alpha \to \beta)$.
  Since
  \begin{equation*}
    \{p_1^*, \dots, p_k^*\} \vdash \alpha,
    \quad
    \{p_1^*, \dots, p_k^*\} \vdash \neg\beta,
    \quad \text{and} \quad
    \vdash \alpha \to (\neg\beta \to \neg(\alpha \to \beta)),
  \end{equation*}
  we have $\{p_1^*, \dots, p_k^*\} \vdash \gamma^*$, completing the proof.
\end{proof}