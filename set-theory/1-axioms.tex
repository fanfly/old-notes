\chapter{Axioms and Operations}
\section{Basic Axioms}
For sets $x$ and $y$, we write $x \in y$ to say that $x$ is an element of
$y$, and we write $x = y$ to say that $x$ and $y$ are equal.
Furthermore, we define
\begin{align*}
  x \notin y \quad &\Leftrightarrow \quad \neg(x \in y) \\
  x \neq y \quad &\Leftrightarrow \quad \neg(x = y).
\end{align*}

\begin{axiom}[Extensionality]
  Two sets are equal if they have exactly the same elements.
  Formally,
  \begin{equation*}
    \forall x \forall y (\forall z (z \in x \leftrightarrow z \in y) \to
    x = y).
  \end{equation*}
\end{axiom}

\begin{definition}
  Let $x$ and $y$ be sets.
  We say that $x$ is a \emph{subset} of $y$, denoted by $x \subseteq y$, if
  every element of $x$ belongs to $y$.
  Formally,
  \begin{equation*}
    x \subseteq y
    \quad \Leftrightarrow \quad
    \forall z (z \in x \to z \in y).
  \end{equation*}
  Furthermore, $x$ is a \emph{proper subset} of $y$, denoted by
  $x \subsetneq y$, if both $x \subseteq y$ and $x \neq y$ hold.
\end{definition}

\begin{definition}
  The \emph{empty set}, denoted by $\varnothing$, is the set that has no
  elements.
\end{definition}

\begin{axiom}[Pairing]
  For any two sets $x$ and $y$, there is a set that consists of exactly $x$ and
  $y$.
  Formally,
  \begin{equation*}
    \forall x \forall y \exists z \forall w (w \in z \leftrightarrow
    (w = x \vee w = y)).
  \end{equation*}
\end{axiom}

\begin{definition}
  The \emph{pair set} of two sets $x$ and $y$, denoted by $\{x, y\}$, is the
  set that consists of exactly $x$ and $y$.
\end{definition}

\begin{axiom}[Power Set]
  For any set $x$, there is a set whose memebers are exactly the subsets
  of $x$.
  Formally,
  \begin{equation*}
    \forall x \exists y \forall z (z \subseteq x \leftrightarrow z \in y).
  \end{equation*}
\end{axiom}

\begin{definition}
  The \emph{power set} of a set $x$, denoted by $\mathcal{P}(x)$, is the set
  that consists of exactly the subsets of $x$.
\end{definition}

\begin{axiom}[Separation Scheme]
  Let $\phi(z)$ be a formula.
  For any set $x$, there exists a set $y$ such that for any set $z$, we have
  $z \in y$ if and only if both $z \in x$ and $\phi(z)$ hold.
  Formally,
  \begin{equation*}
    \forall x \exists y \forall z (z \in y \leftrightarrow z \in x \wedge
    \phi(z)).
  \end{equation*}
\end{axiom}

\begin{definition}
  Let $x, y$ be sets and let $\phi(z)$ be a formula.
  If for any set $z$, we have $z \in y$ if and only if $z \in x$ and $\phi(z)$,
  then we write
  \begin{equation*}
    y = \{z \in x: \phi(z)\}.
  \end{equation*}
\end{definition}

\begin{theorem}
  There is no set to which every set belongs.
  Formally,
  \begin{equation*}
    \forall x \exists y (y \notin x).
  \end{equation*}
\end{theorem}
\begin{proof}
  Let $x$ be a set and let $y = \{z \in x: z \notin z\}$.
  Then
  \begin{equation*}
    y \in y
    \quad \Leftrightarrow \quad
    y \in x \wedge y \notin y.
  \end{equation*}
  If $y \in x$, then
  \begin{equation*}
    y \in y \quad \Leftrightarrow \quad y \notin y,
  \end{equation*}
  contradiction.
  Thus $y \notin x$, which completes the proof.
\end{proof}

\begin{axiom}[Union]
  For any set $x$, there exists a set whose elements are exactly the
  elements of the elements of $x$.
  Formally,
  \begin{equation*}
    \forall x \exists y \forall z
    (z \in y \leftrightarrow \exists w (w \in x \wedge z \in w)).
  \end{equation*}
\end{axiom}

\begin{definition}
  Let $x$ be a set.
  \begin{itemize}
    \item We define the \emph{union} of $x$, denoted by $\bigcup x$, to be the
    set that consists of the sets that belongs to at least one element of $x$.
    Formally, for any set $z$ we have
    \begin{equation*}
      z \in \bigcup x
      \quad \Leftrightarrow \quad
      \exists w (w \in x \wedge z \in w).
    \end{equation*}
    \item If $x$ is nonempty, we define the \emph{intersection} of $x$, denoted
    by $\bigcap x$, to be the set that consists of the sets that belongs to all
    elements of $x$.
    Formally, for any set $z$ we have
    \begin{equation*}
      z \in \bigcap x
      \quad \Leftrightarrow \quad
      \forall w (w \in x \to z \in w).
    \end{equation*}
  \end{itemize}
\end{definition}

\begin{definition}
  For any sets $x$ and $y$, we define
  \begin{equation*}
    x \cup y = \bigcup \{x, y\}, \qquad x \cap y = \bigcap \{x, y\}.
  \end{equation*}
\end{definition}