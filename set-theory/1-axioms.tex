\chapter{Axioms and Operations}
\section{Basic Axioms}
\begin{axiom}[Extensionality]
  For any sets $x$ and $y$, if for any set $z$, we have $z \in x$ if and only
  if $z \in y$, then we say that $x$ and $y$ are \emph{equal}, denoted $x = y$.
\end{axiom}

\begin{axiom}[Empty Set]
  There is a set $x$ such that $y \notin x$ for each set $y$.
  The set $x$ is called the \emph{empty set} and is denoted by $\varnothing$.
\end{axiom}

\begin{axiom}[Pairing]
  For any sets $x$ and $y$, there is a set $w$ such that for each set
  $z \in w$, either $z = x$ or $z = y$ holds.
  The set $w$ is called the \emph{pair set} of $x$ and $y$ and is denoted by
  $\{x, y\}$.
  If $x = y$, then we write $\{x\}$ for short.
\end{axiom}

\begin{example}
  By axiom of pairing, $\{\varnothing\}$ and $\{\varnothing, \{\varnothing\}\}$
  are sets.
\end{example}

\begin{axiom}[Power Set]
  For any set $x$, there exists a set $y$ such that for any set $z$, $z \in y$
  if and only if $z \subseteq x$.
  The set $y$ is called the \emph{power set} of $x$ and is denoted by
  $\mathcal{P}(x)$.
\end{axiom}

\begin{axiom}[Subset]
  Let $\phi(z)$ be a first-order formula such that $z$ is the only free
  variable in $\phi$.
  For any set $x$, there exists a set $y$ such that for any set $z$, $z \in y$
  if and only if both $z \in x$ and $\phi(z)$ holds.
  The set $y$ will be denoted by
  \begin{equation*}
    y = \{z \in x: \phi(z)\}.
  \end{equation*}
\end{axiom}

\begin{theorem}
  There is no set to which every set belongs.
  That is, for any set $x$, there exists a set $y$ such that $y \notin x$.
\end{theorem}
\begin{proof}
  Let $y = \{z \in x: z \notin z\}$.
  Then we have $y \in y$ if and only if $y \in x$ and $y \notin y$.
  If $y \in x$, then $y \in y$ if and only if $y \notin y$, contradiction.
  Thus, $y \notin x$.
\end{proof}

\section{Arbitrary Unions and Intersections}
\begin{axiom}[Union]
  For any set $x$, there exists a set $y$ whose elements are exactly the
  members of the members of $x$.
  That is,
  \begin{equation*}
    \forall x \exists y \forall z
    (z \in y \leftrightarrow \exists w (w \in x \wedge z \in w)).
  \end{equation*}
  The set $y$ is called the \emph{union} of $x$, denoted by $\bigcup x$.
\end{axiom}

\begin{theorem}
  For any nonempty set $x$, there exists a unique set $y$ such that for any
  set $z$, $z \in y$ if and only if $z$ belongs to every member of $x$.
\end{theorem}
\begin{proof}
  Since $x$ is nonempty, there is a member $w_0$ of $x$.
  Then by a subset axiom there exists a set $y$ such that
  \begin{equation*}
    y = \{z \in w_0: \forall w (w \in x \to z \in w)\},
  \end{equation*}
  and uniqueness of $y$ follows from extensionality.
\end{proof}

\begin{definition}
  For any nonempty set $x$, we define the \emph{intersection} of $x$ as the set
  $y$ such that for any set $z$, $z \in y$ if and only if $z$ belongs to every
  member of $x$.
  Let $\bigcap x$ denote the intersection of $x$.
\end{definition}

\begin{definition}
  For any sets $x$ and $y$, we define
  \begin{align*}
    x \cup y &= \bigcup \{x, y\} \\
    x \cap y &= \bigcap \{x, y\}.
  \end{align*}
\end{definition}