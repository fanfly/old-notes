\chapter{Axioms and Operations}
\section{Basic Axioms}
\begin{axiom}[Extensionality]
  For any sets $x$ and $y$, if for any set $z$, we have $z \in x$ if and only
  if $z \in y$, then we say that $x$ and $y$ are \emph{equal}, denoted $x = y$.
\end{axiom}

\begin{axiom}[Empty Set]
  There is a set $x$ such that $y \notin x$ for each set $y$.
  The set $x$ is called the \emph{empty set} and is denoted by $\varnothing$.
\end{axiom}

\begin{axiom}[Pairing]
  For any sets $x$ and $y$, there is a set $w$ such that for each set
  $z \in w$, either $z = x$ or $z = y$ holds.
  The set $w$ is called the \emph{pair set} of $x$ and $y$ and is denoted by
  $\{x, y\}$.
  If $x = y$, then we write $\{x\}$ for short.
\end{axiom}

\begin{axiom}[Power Set]
  For any set $x$, there exists a set $y$ such that for any set $z$, $z \in y$
  if and only if $z \subseteq x$.
  The set $y$ is called the \emph{power set} of $x$ and is denoted by
  $\mathcal{P}(x)$.
\end{axiom}

\begin{axiom}[Subset]
  Let $\phi(z)$ be a first-order formula such that $z$ is the only free
  variable in $\phi$.
  For any set $x$, there exists a set $y$ such that for any set $z$, $z \in y$
  if and only if both $z \in x$ and $\phi(z)$ holds.
  The set $y$ will be denoted by
  \begin{equation*}
    y = \{z \in x: \phi(z)\}.
  \end{equation*}
\end{axiom}

\section{Arbitrary Unions and Intersections}