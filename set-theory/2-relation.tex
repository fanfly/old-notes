\chapter{Relations and Functions}
\section{Ordered Pairs}
\begin{definition}
  For sets $x$ and $y$, we define
  \begin{equation*}
    \langle x, y \rangle = \{\{x\}, \{x, y\}\}.
  \end{equation*}
\end{definition}

\begin{lemma}
  Let $x, y, y'$ be sets.
  If $\{x, y\} = \{x, y'\}$, then $y = y'$.
\end{lemma}
\begin{proof}
  Suppose that $y \neq y'$.
  Since $y \in \{x, y\} = \{x, y'\}$ and $y \neq y'$, we have $y = x$.
  Then we have $y' \in \{x, y'\} = \{x, y\} = \{x\}$, implying $y' = x = y$,
  contradiction.
  Thus, $y = y'$.
\end{proof}

\begin{theorem}
  For sets $x, x', y, y'$, we have
  \begin{equation*}
    \langle x, y \rangle = \langle x', y' \rangle
  \end{equation*}
  if and only if $x = x'$ and $y = y'$.
\end{theorem}
\begin{proof}
  ($\Leftarrow$)
  Straightforward.
  ($\Rightarrow$)
  Suppose that $x \neq x'$.
  Since
  \begin{equation*}
    \{\{x\}, \{x, y\}\} = \{\{x'\}, \{x', y'\}\},
  \end{equation*}
  either $\{x\} = \{x', y'\}$ or $\{x\} = \{x'\}$ holds.
  For both cases we all have $x' \in \{x\}$, implying $x' = x$, contradiction.
  Hence we have $x = x'$, and it follows that $\{x\} = \{x'\}$, implying
  $\{x, y\} = \{x', y'\}$, and thus $y = y'$.
\end{proof}

\begin{lemma}
  If $x, y \in C$, then $\langle x, y \rangle \in \mathcal{P}(\mathcal{P}(C))$.
\end{lemma}
\begin{proof}
  Since $\{x\}$ and $\{y\}$ are subsets of $C$, we have
  $\{x\}, \{x, y\} \in \mathcal{P}(C)$.
  It follows that $\{\{x\}, \{x, y\}\}$ is a subset of $\mathcal{P}(C)$,
  implying
  \begin{equation*}
    \langle x, y \rangle = \{\{x\}, \{x, y\}\} \in \mathcal{P}(\mathcal{P}(C)).
    \qedhere
  \end{equation*}
\end{proof}

\begin{theorem}
  For any sets $A$ and $B$, there is a set whose members are exactly the
  pairs $(x, y)$ with $x \in A$ and $y \in B$.
\end{theorem}
\begin{proof}
  Since $x, y \in A \cup B$, the set of pairs $\langle x, y \rangle$ with
  $x \in A$ and $y \in B$ can be given by
  \begin{equation*}
    \{z \in \mathcal{P}(\mathcal{P}(A \cup B)):
    \text{$z = \langle x, y \rangle$ for some $x \in A$ and $y \in B$}\}.
    \qedhere
  \end{equation*}
\end{proof}

\begin{definition}
  For any sets $A$ and $B$, the \emph{Cartesian product} of $A$ and $B$,
  denoted by $A \times B$, is the set whose memebers are exactly the pairs
  $\langle x, y \rangle$ with $x \in A$ and $y \in B$.
\end{definition}

\section{Relations}
\begin{definition}
  A \emph{relation} is a set of ordered pairs.
  For any relation $R$, the \emph{domain} and the \emph{range} of $R$, denoted
  by $\dom(R)$ and $\ran(R)$, respectively, are defined as follows.
  \begin{itemize}
    \item $\dom(R)$ is the collection of sets $x$ with
    $\langle x, y \rangle \in R$ for some $y$.
    \item $\ran(R)$ is the collection of sets $y$ with
    $\langle x, y \rangle \in R$ for some $x$.
  \end{itemize}
\end{definition}

\begin{definition}
  Let $R$ and $S$ be relations and let $X$ be a set.
  \begin{itemize}
    \item The \emph{inverse} of $R$, denoted by $R^{-1}$, is the set of all
    pairs $\langle y, x \rangle$ with $\langle x, y \rangle \in R$.
    \item The \emph{restriction} of $R$ to $X$, denoted by
    $R \upharpoonright X$, is the set of all pairs $\langle x, y \rangle \in R$
    with $x \in X$.
    \item The \emph{composition} of $R$ and $S$, denoted by $R \circ S$, is the
    set of all pairs $\langle x, z \rangle$ with $\langle x, y \rangle \in R$
    and $\langle y, z \rangle \in S$.
  \end{itemize}
\end{definition}

\begin{definition}
  A \emph{function} is a relation $f$ such that for any set $x \in \dom(f)$,
  there exists a unique set $y$ such that $\langle x, y \rangle \in f$.
  The unique set $y$ with respect to $x$ is called the \emph{value} of $f$
  at $x$ and is denoted $f(x)$.
  \begin{itemize}
    \item We say that $f$ is a function from $A$ to $B$, denoted by
    $f: A \to B$, if $\dom(f) = A$ and $\ran(f) \subseteq B$.
    \item $f$ is \emph{one-to-one} if for any $y \in \ran(f)$, there exists a
    unique set $x \in \dom(f)$ with $f(x) = y$.
  \end{itemize}
\end{definition}

\begin{definition}
  For any sets $A$ and $B$, the set of functions from $A$ to $B$ is denoted by
  $B^A$.
\end{definition}

\section{Equivalence Relations and Ordering Relations}
\begin{definition}
  Let $A$ be a set.
  An \emph{equivalence relation} on $A$ is a relation $R \subseteq A \times A$
  that satisfies the following three conditions.
  \begin{itemize}
    \item Reflexivity: $\langle x, x \rangle \in R$ for any $x \in A$.
    \item Symmetry: $\langle x, y \rangle \in R$ implies
    $\langle y, x \rangle \in R$ for any $x, y \in A$.
    \item Transitivity: $\langle x, y \rangle \in R$ and
    $\langle y, z \rangle \in R$ implies $\langle x, z \rangle \in R$ for any
    $x, y, z \in A$.
  \end{itemize}
\end{definition}