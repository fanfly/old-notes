\chapter{Equinumerosity}
\section{Equinumerosity}
\begin{definition}
  We say that $A$ is \emph{equinumerous} to $B$, denoted by $A \approx B$,
  if there exists a one-to-one function from $A$ onto $B$.
\end{definition}

\begin{theorem}
  The following statements hold for any sets $A, B, C$.
  \begin{enumerate}
    \item $A \approx A$.
    \item $A \approx B$ implies $B \approx A$.
    \item $A \approx B$ and $B \approx C$ implies $A \approx C$.
  \end{enumerate}
\end{theorem}
\begin{proof}
  To be completed.
\end{proof}

\begin{theorem}
  For any set $A$, we have $A \not \approx \mathcal{P}(A)$.
\end{theorem}

\section{Finite Sets}
\begin{definition}
  A set is \emph{finite} if it is equinumerous to a natural number.
  A set is \emph{infinite} if it is not finite.
\end{definition}

\begin{theorem}[Pigeonhole Principle]
  If $A$ is finite and $B \subsetneq A$, then $A \prec B$.
\end{theorem}
