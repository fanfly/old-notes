\chapter{Natural Numbers}
\section{Inductive Sets}
\begin{definition}
  The \emph{successor} of a set $x$, denoted $x^+$, is defined by
  \begin{equation*}
    x^+ = x \cup \{x\}.
  \end{equation*}
  We say that a set $A$ is \emph{inductive} if $\varnothing \in A$ and for any
  $x \in A$, we have $x^+ \in A$.
\end{definition}

\begin{axiom}[Infinity]
  There exists an inductive set.
\end{axiom}

\begin{definition}
  A \emph{natural number} is a set belonging to all inductive sets.
  The set of natural numbers is denoted by $\omega$.
\end{definition}

\begin{theorem}
  $\omega$ is inductive.
\end{theorem}
\begin{proof}
  First, $\varnothing \in \omega$ since $\varnothing$ belongs to all inductive
  sets by definition.
  For any set $x \in \omega$, $x$ belongs to all inductive sets, implying that
  $x^+$ belongs to all inductive sets, and thus $x^+ \in \omega$.
  Thus, $\omega$ is inductive.
\end{proof}

\section{Recursion}
\begin{theorem}[Recursion Theorem]
  Let $S$ be a set, let $s \in S$, and let $\sigma: S \to S$.
  Then there exists a unique function $f: \omega \to S$ such that $f(0) = s$
  and $f(n^+) = \sigma(f(n))$ for all $n \in \omega$.
\end{theorem}

\section{Arithmetic}

\section{Ordering}