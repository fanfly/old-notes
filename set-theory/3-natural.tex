\chapter{Natural Numbers}
\section{Inductive Sets}
\begin{definition}
  The \emph{successor} of a set $x$, denoted $x^+$, is defined by
  \begin{equation*}
    x^+ = x \cup \{x\}.
  \end{equation*}
\end{definition}

\begin{definition}
  A set $A$ is \emph{inductive} if $\varnothing \in A$, and for any $a \in A$,
  we have $a^+ \in A$.
\end{definition}

\begin{axiom}[Infinity]
  There exists an inductive set.
\end{axiom}

\begin{definition}
  A \emph{natural number} is a set belonging to all inductive sets.
  The set of natural numbers is denoted by $\omega$.
\end{definition}

\begin{theorem}
  $\omega$ is inductive.
\end{theorem}
\begin{proof}
  First, $\varnothing \in \omega$ since $\varnothing$ belongs to all inductive
  sets by definition.
  For any set $x \in \omega$, $x$ belongs to all inductive sets, implying that
  $x^+$ belongs to all inductive sets, and thus $x^+ \in \omega$.
  Thus, $\omega$ is inductive.
\end{proof}

\begin{definition}
  Let
  \begin{equation*}
    0 = \varnothing, \quad
    1 = \varnothing^+, \quad
    2 = (\varnothing^+)^+, \quad
    3 = ((\varnothing^+)^+)^+, \quad
    4 = (((\varnothing^+)^+)^+)^+, \quad
    \dots
  \end{equation*}
  denote the natural numbers.
\end{definition}

\section{Recursion}
\begin{theorem}[Recursion Theorem]
  Let $A$ be a set.
  Let $a \in A$ and $G: A \to A$.
  Then there is a unique function $f: \omega \to A$ such that $f(0) = a$
  and $f(n^+) = G(f(n))$ for all $n \in \omega$.
\end{theorem}
\begin{proof}
  Let $H$ be the set of functions $h \subseteq \omega \times X$ satisfying the
  following conditions.
  \begin{enumerate}[1.]
    \item If $0 \in \dom(h)$, then $h(0) = a$.
    \item For any $n \in \omega$, if $n^+ \in \dom(h)$, then $n \in \dom(h)$
    and $h(n^+) = G(h(n))$.
  \end{enumerate}
  Let $f = \bigcup H$.
  The rest of this proof is to be completed.
\end{proof}

\section{Arithmetic}
\begin{definition}
  For $n, m \in \omega$, we define
  \begin{equation*}
    n + 0 = n
    \qquad \text{and} \qquad
    n + m^+ = (n + m)^+
  \end{equation*}
  for all $n, m \in \omega$.
\end{definition}

\begin{definition}
  For $n, m \in \omega$, we define
  \begin{equation*}
    n \cdot 0 = 0
    \qquad \text{and} \qquad
    n \cdot m^+ = n \cdot m + n
  \end{equation*}
  for all $n, m \in \omega$.
\end{definition}

\section{Ordering}
\begin{definition}
  We define binary relations $<$ and $\leq$ over the set $\omega$ of natural
  numbers such that
  \begin{equation*}
    n < m
    \quad \Leftrightarrow \quad
    n \in m
  \end{equation*}
  and
  \begin{equation*}
    n \leq m
    \quad \Leftrightarrow \quad
    \text{$n \in m$ or $n = m$}
  \end{equation*}
  for $n, m \in \omega$.
\end{definition}