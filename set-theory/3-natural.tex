\chapter{Natural Numbers}
\section{Inductive Sets}
\begin{definition}
  The \emph{successor} of a set $x$, denoted $x^+$, is defined by
  \begin{equation*}
    x^+ = x \cup \{x\}.
  \end{equation*}
\end{definition}

\begin{definition}
  A set $x$ is \emph{inductive} if it has the empty set as member and is closed
  under successor.
  That is,
  \begin{equation*}
    \text{$x$ is inductive}
    \quad \Leftrightarrow \quad
    \varnothing \in x \wedge \forall y (y \in x \to y^+ \in x).
  \end{equation*}
\end{definition}

\begin{axiom}[Infinity]
  There exists an inductive set.
  That is,
  \begin{equation*}
    \exists x (\varnothing \in x \wedge \forall y (y \in x \to y^+ \in x)).
  \end{equation*}
\end{axiom}

\begin{definition}
  A \emph{natural number} is a set $x$ that belongs to all inductive sets.
  The set of natural numbers is denoted by $\omega$.
  That is,
  \begin{equation*}
    x \in \omega
    \quad \Leftrightarrow \quad
    \forall A (\text{$A$ is inductive} \to x \in A)
  \end{equation*}
\end{definition}

\begin{theorem}
  $\omega$ is inductive.
\end{theorem}
\begin{proof}
  First, $\varnothing \in \omega$ since $\varnothing$ belongs to all inductive
  sets by definition.
  For any set $x \in \omega$, $x$ belongs to all inductive sets, implying that
  $x^+$ belongs to all inductive sets, and thus $x^+ \in \omega$.
  Thus, $\omega$ is inductive.
\end{proof}

\begin{definition}
  Let
  \begin{equation*}
    0 = \varnothing, \quad
    1 = \varnothing^+, \quad
    2 = (\varnothing^+)^+, \quad
    3 = ((\varnothing^+)^+)^+, \quad
    4 = (((\varnothing^+)^+)^+)^+, \quad
    \dots
  \end{equation*}
  denote the natural numbers.
\end{definition}

\section{Recursion}
\begin{theorem}[Recursion Theorem]
  For any sets $A$ and $e$ with $e \in A$ and any function $\Phi: A \to A$,
  there is a unique function $f: \omega \to A$ such that
  \begin{equation*}
    f(\varnothing) = e
    \qquad \text{and} \qquad
    f(n^+) = \Phi(f(n))
  \end{equation*}
  for all $n \in \omega$.
\end{theorem}
\begin{proof}
  We call a function $h \in \mathcal{P}(\omega \times A)$ a candidate function
  if it satisfies the following properties.
  \begin{enumerate}[1.]
    \item If $\varnothing \in \dom(h)$, then $h(\varnothing) = e$.
    \item For every $n \in \omega$, if $n^+ \in \dom(h)$, then $n \in \dom(h)$
    and $f(n^+) = \Phi(f(n))$.
  \end{enumerate}
  Let $H$ denote the set of all candidate functions and let $f = \bigcup H$.
  First we show that $f \in \mathcal{P}(\omega \times A)$ is a function,
  i.e., the set
  \begin{equation*}
    I = \{n \in \omega: \text{$\langle n, a \rangle, \langle n, a' \rangle$
    implies $a = a'$ for all $a, a' \in A$}\}
  \end{equation*}
  is inductive.
  We have $\varnothing \in I$ by definition of candidate function.
  Now suppose that $n \in I$ and we prove that $n^+ \in I$.
  For any $h, h' \in H$ with $n^+ \in \dom(h)$ and $n^+ \in \dom(h')$, we have
  $h(n) = h'(n)$ by $n \in I$, implying
  \begin{equation*}
    h(n^+) = \Phi(h(n)) = \Phi(h'(n)) = h'(n^+).
  \end{equation*}
  Thus $n^+ \in I$, and we conclude that $f$ is a function.
  Now we show that $\dom(f) = \omega$.
  We have $\varnothing \in \dom(f)$ since $\{\langle \varnothing, e \rangle\}
  \in H$.
  For any $n \in \dom(f)$, let $h \in H$ with $n \in \dom(h)$.
  If $n^+ \in \dom(h)$, then $n^+ \in \dom(f)$.
  Otherwise, since $h' = h \cup \{\langle n^+, \Phi(h(n)) \rangle\} \in H$, we
  also have $n^+ \in \dom(f)$.

  Now we show that $f \in H$.
  We have $f(\varnothing) = e$ since
  $\{\langle \varnothing, e \rangle\} \in H$.
  For any $n \in \dom(f)$, let $h \in H$ with $n \in \dom(h)$.
  If $n^+ \in \dom(h)$, then we have
  \begin{equation*}
    f(n^+) = h(n^+) = \Phi(h(n)) = \Phi(f(n)).
  \end{equation*}
  Otherwise, let $h' = h \cup \{\langle n^+, \Phi(h(n)) \rangle\} \in H$, and
  then we have
  \begin{equation*}
    f(n^+) = h'(n^+) = \Phi(h(n)) = \Phi(f(n)).
  \end{equation*}
  Thus $f \in H$.
  For the uniqueness of $f$, let $g \in H$ with $\dom(g) = \omega$, and then
  since $g \subseteq f$, we have $g(n) = f(n)$ for all $n \in \omega$, implying
  $g = f$.
  This completes the proof.
\end{proof}

\section{Arithmetic}
\begin{definition}
  For $n, m \in \omega$, we define
  \begin{equation*}
    n + 0 = n
    \qquad \text{and} \qquad
    n + m^+ = (n + m)^+
  \end{equation*}
  for all $n, m \in \omega$.
\end{definition}

\begin{lemma}
  For $n, m \in \omega$, we have the following properties.
  \begin{enumerate}
    \item $0 + n = n$.
    \item $n^+ + m = (n + m)^+$.
  \end{enumerate}
\end{lemma}
\begin{proof}
  \leavevmode
  \begin{enumerate}
    \item The proof is by induction on $n$.
    We have $0 + 0 = 0$.
    Now let $n \in \omega$.
    If $0 + n = n$, then
    \begin{equation*}
      0 + n^+ = (0 + n)^+ = n^+.
    \end{equation*}
    \item The proof is by induction on $m$.
    We have $n^+ + 0 = n^+ = (n + 0)^+$ for all $n \in \omega$.
    Now let $m \in \omega$.
    If $n^+ + m = (n + m)^+$ for all $n \in \omega$, then
    \begin{equation*}
      n^+ + m^+ = (n^+ + m)^+ = ((n + m)^+)^+ = (n + m^+)^+.
    \end{equation*}
    for all $n \in \omega$.
    \qedhere
  \end{enumerate}
\end{proof}

\begin{theorem}
  For $n, m, \ell \in \omega$, we have the following properties.
  \begin{enumerate}
    \item $n + m = m + n$.
    \item $(n + m) + \ell = n + (m + \ell)$.
  \end{enumerate}
\end{theorem}
\begin{proof}
  \leavevmode
  \begin{enumerate}
    \item The proof is by induction on $m$.
    We have $n + 0 = n = 0 + n$ for all $n \in \omega$.
    Now let $m \in \omega$.
    If $n + m = m + n$ for all $n \in \omega$, then
    \begin{equation*}
      n + m^+ = (n + m)^+ = (m + n)^+ = m^+ + n.
    \end{equation*}
    for all $n \in \omega$.
    \item The proof is by induction on $\ell$.
    We have $(n + m) + 0 = n + m = n + (m + 0)$.
    Now let $\ell \in \omega$.
    If $(n + m) + \ell = n + (m + \ell)$ for all $n, m \in \omega$, then
    \begin{align*}
      (n + m) + \ell^+
      &= ((n + m) + \ell)^+ \\
      &= (n + (m + \ell))^+ \\
      &= n + (m + \ell)^+ \\
      &= n + (m + \ell^+)
    \end{align*}
    for all $n, m \in \omega$.
  \end{enumerate}
\end{proof}

\begin{definition}
  For $n, m \in \omega$, we define
  \begin{equation*}
    n \cdot 0 = 0
    \qquad \text{and} \qquad
    n \cdot m^+ = n \cdot m + n
  \end{equation*}
  for all $n, m \in \omega$.
\end{definition}

\begin{lemma}
  For $n, m \in \omega$, we have the following properties.
  \begin{enumerate}
    \item $0 \cdot n = 0$.
    \item $n^+ \cdot m = n \cdot m + m$.
  \end{enumerate}
\end{lemma}
\begin{proof}
  \leavevmode
  \begin{enumerate}
    \item The proof is by induction on $n$.
    We have $0 \cdot 0 = 0$.
    Now let $n \in \omega$.
    If $0 \cdot n = 0$, then
    \begin{equation*}
      0 + n^+ = 0 \cdot n^+ + 0 = 0.
    \end{equation*}
    \item The proof is by induction on $m$.
    We have $n^+ \cdot 0 = 0 = n \cdot 0 + 0$ for all $n \in \omega$.
    Now let $m \in \omega$.
    If $n^+ \cdot m = n \cdot m + m$ for all $n \in \omega$, then
    \begin{align*}
      n^+ \cdot m^+
      &= n^+ \cdot m + n^+ \\
      &= n \cdot m + m + n^+ \\
      &= n \cdot m + n + m^+ \\
      &= n \cdot m^+ + m^+
    \end{align*}
    for all $n \in \omega$.
    \qedhere
  \end{enumerate}
\end{proof}

\begin{theorem}
  For $n, m, \ell \in \omega$, we have the following properties.
  \begin{enumerate}
    \item $n \cdot (m + \ell) = n \cdot m + n \cdot \ell$.
    \item $n \cdot m = m \cdot n$.
    \item $(n \cdot m) \cdot \ell = n \cdot (m \cdot \ell)$.
  \end{enumerate}
\end{theorem}
\begin{proof}
  \leavevmode
  \begin{enumerate}
    \item The proof is by induction on $\ell$.
    We have
    \begin{equation*}
      n \cdot (m + 0)
      = n \cdot m
      = n \cdot m + 0
      = n \cdot m + n \cdot 0
    \end{equation*}
    for all $n, m \in \omega$.
    Now let $\ell \in \omega$.
    If $n \cdot (m + \ell) = n \cdot m + n \cdot \ell$ for all $n, m \in
    \omega$, then
    \begin{align*}
      n \cdot (m + \ell^+)
      &= n \cdot (m + \ell)^+ \\
      &= n \cdot (m + \ell) + n \\
      &= (n \cdot m + n \cdot \ell) + n \\
      &= n \cdot m + (n \cdot \ell + n) \\
      &= n \cdot m + n \cdot \ell^+
    \end{align*}
    for all $n, m \in \omega$.
    \item The proof is by induction on $m$.
    We have $n \cdot 0 = 0 = 0 \cdot n$ for all $n \in \omega$.
    Now let $m \in \omega$.
    If $n \cdot m = m \cdot n$ for all $n \in \omega$, then
    \begin{equation*}
      n \cdot m^+
      = n \cdot m + n
      = m \cdot n + n
      = m^+ \cdot n
    \end{equation*}
    for all $n \in \omega$.
    \item The proof is by induction on $\ell$.
    We have $(n \cdot m) \cdot 0 = 0 = n \cdot (m + 0)$.
    Now let $\ell \in \omega$.
    If $(n \cdot m) \cdot \ell = n \cdot (m \cdot \ell)$ for all $n, m \in
    \omega$, then
    \begin{align*}
      (n \cdot m) \cdot \ell^+
      &= (n \cdot m) \cdot \ell + n \cdot m \\
      &= n \cdot (m \cdot \ell) + n \cdot m \\
      &= n \cdot (m \cdot \ell + m) \\
      &= n \cdot (m \cdot \ell^+)
    \end{align*}
    for all $n, m \in \omega$.
    \qedhere
  \end{enumerate}
\end{proof}

\section{Ordering}
\begin{definition}
  We define binary relations $<$ and $\leq$ over the set $\omega$ of natural
  numbers such that
  \begin{equation*}
    n < m
    \quad \Leftrightarrow \quad
    n \in m
  \end{equation*}
  and
  \begin{equation*}
    n \leq m
    \quad \Leftrightarrow \quad
    \text{$n \in m$ or $n = m$}
  \end{equation*}
  for $n, m \in \omega$.
\end{definition}