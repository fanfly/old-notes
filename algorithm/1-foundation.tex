\chapter{Foundations}
\section{Computational Problems and Algorithms}
\begin{definition}
  A \emph{computational problem} is a relation
  \begin{equation*}
    P \subseteq X \times Y,
  \end{equation*}
  where $X$ is called the set of \emph{instances} and $Y$ is called the sets of
  \emph{solutions}.
\end{definition}

\begin{example}
  Let $X$ be the set of nonempty sequence of distinct integers and let $Y$
  be the set of positive integers.
  Then we can define $P \subseteq X \times Y$ such that
  \begin{equation*}
    ((a_1, a_2, \dots, a_n), i) \in P
  \end{equation*}
  if and only if $1 \leq i \leq n$ and $a_i \geq a_j$ for all
  $j \in \{1, \dots, n\}$.
  We can write this problem as follows.
\end{example}
\begin{problem}[Champion Problem]
  \label{prob:champion}
  \leavevmode
  \begin{itemize}
    \item Input: A sequence $A$ of $n$ distinct integers $A[1], \dots, A[n]$.
    \item Output: The index of the maximum element of $A$.
  \end{itemize}
\end{problem}

\begin{definition}
  We will assume the \emph{random-access machine (RAM)} model of computation as
  our implementaion technology for most of this note.
  In this model, we have an infinite sequence of $w$-bit words, and we assume
  $w = \lceil c \lg n \rceil$ for some constant $c \geq 1$, where $n$ is the
  input size.
  We can perform some basic operations on these words, including
  \begin{itemize}
    \item arithmetic operations (e.g., addition, subtraction, multiplication,
    division),
    \item data movement operations (e.g., load, store, copy), and
    \item control operations (e.g., branch, subroutine call, return).
  \end{itemize}
\end{definition}

\begin{definition}
  Given a computational model, an \emph{algorithm} is defined as a finite
  sequence of basic operations that transforms a given input into a unique
  output.
  \begin{itemize}
    \item We say that an algorithm \emph{solves} a computational problem
    $P \subseteq X \times Y$ if it transforms every instance $x \in X$ into a
    solution $y \in Y$ such that $(x, y) \in P$.
    \item The \emph{running time} of an algorithm on a specific input is
    defined as the number of basic operations performed.
  \end{itemize}
\end{definition}

\begin{example}
  We can find the index of maximum of a sequence by the following algorithm
  \textsc{Max-Index}.
  \begin{codebox}
    \Procname{$\proc{Max-Index}(A)$}
    \li $n \gets \attrib{A}{length}$
    \li $i \gets 1$
    \li \For $j \gets 2$ \To $n$ \Do
    \li     \If $A[i] < A[j]$ \Then
    \li         $i \gets j$
            \End
        \End
    \li \Return $i$
  \end{codebox}
  Note that we will describe algorithm is pseudocode so that implementaion
  details can be hidden.
\end{example}

\begin{theorem}
  The algorithm \textsc{Max-Index} solves the champion problem using the least
  number of comparisons.
\end{theorem}
\begin{proof}
  We show that \textsc{Max-Index} solves the champion problem as follows.
  We focus on the loop invariant that at the start of each iteration of the
  \For loop of lines 3 -- 5, $A[i]$ is the maximum of the subarray
  $A[1 \twodots j-1]$.
  The loop invariant is obviously true when entering the loop, and it remains
  true between iterations since $A[i] \geq A[j]$ must hold at the end of each
  iteration due to lines 4 and 5.
  Thus, when the \For loop terminates, $A[i]$ should be the maximum of
  $A[1 \twodots n]$.

  Furthremore, the algorithm \textsc{Max-Index} uses $n-1$ comparisons, and it
  is optimal with respect to the number of comparisons performed, since at
  least $n-1$ comparisons are necessary to determine the maximum.
\end{proof}