\chapter{Hardness of Problems}
\section{Computational Problems and Algorithms}
\begin{definition}
  A \emph{computational problem} is a relation between two sets, i.e.,
  \begin{equation*}
    R \subseteq X \times Y,
  \end{equation*}
  where $X$ is called the set of \emph{instances} and $Y$ is called the sets of
  \emph{solutions}. 
\end{definition}

\begin{definition}
  A \emph{computational model} is a model which defines a set of basic
  operations, each transforming a state of computation into another.
\end{definition}

\begin{example}
  A popular computational model is called the \emph{random-access machine}
  (RAM) model.
  In this model, we have an infinite sequence of cells (the $i$th cell is
  denoted $M[i]$), and we allow operations of the following forms, where
  $i, j, k$ are positive integers.
  \begin{itemize}
    \item $M[i] \gets j$.
    \item $M[i] \gets M[j] + M[k]$.
    \item $M[i] \gets M[j] - M[k]$.
    \item $M[i] \gets M[M[j]]$.
    \item $M[M[i]] \gets M[j]$.
    \item If $M[i] > 0$, jump to operation numbered with $j$.
  \end{itemize}
\end{example}

\begin{definition}
  Given a computational model, an \emph{algorithm} is defined as a finite
  sequence of basic operations (defined in that computational model) that
  transforms a given input into a unique output.
\end{definition}