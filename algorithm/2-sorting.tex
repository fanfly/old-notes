\chapter{Sorting}
\section{Insertion Sort}
In this chapter, we consider algorithms that solves the sorting problem, which
is stated as follows.
\begin{problem}[Sorting Problem]
  \leavevmode
  \begin{itemize}
    \item Input: An array $A$ of $n$ integers.
    \item Output: A permutation of $A$ that is non-decreasing.
  \end{itemize}
\end{problem}

\noindent
Let us begin with \emph{insertion sort}, which is a simple sorting algorithm.
\begin{codebox}
  \Procname{$\proc{Insertion-Sort}(A, n)$}
  \li \For $i \gets 2$ \To $n$ \Do
  \li     $k \gets A[i]$
  \li     $j \gets i$
  \li     \While $j > 1$ and $k < A[j - 1]$ \Do
  \li         $A[j] \gets A[j - 1]$
  \li         $j \gets j - 1$
          \End
  \li     $A[j] \gets k$
      \End
\end{codebox}

\begin{theorem}
  $\proc{Insertion-Sort}$ solves the sorting problem in $\Theta(n^2)$ time in
  the worst case, where $n$ is the length of the input array.
\end{theorem}
\begin{proof}
  We prove the loop invariant that at the start of each iteration of the \For
  loop, the subarray $A[1 \twodots i-1]$ is a non-decreasing permutation of the
  elements originally in $A[1 \twodots i-1]$.

  The loop invariant is trivially true for $i = 2$, and is maintained by each
  iteration as follows.
  First, $A[i]$ is copied into $k$.
  The \While loop of lines 4 -- 6 moves the elements in $A[1 \twodots i-1]$
  that are greater than $k$ by one position to the right.
  Thus, at the end of itertaion, $A[1 \twodots j-1]$ stores the elements
  originally in $A[1 \twodots i-1]$ that are less than or equal to $k$, and
  $A[j+1 \twodots i]$ stores the elements originally in $A[1 \twodots i-1]$
  that are greater than $k$.
  After storing $k$ into $A[j]$, $A[1 \twodots i]$ is a nondecreasing
  permutation of the elements originally in $A[1 \twodots i]$, and incrementing
  $i$ preserves the loop invariant.

  When the \For loop terminates, we have $j = n + 1$.
  Hence, the entire array $A[1 \twodots n]$ is sorted due to the loop
  invariant, implying that $\proc{Insertion-Sort}$ correctly solves the sorting
  problem.

  Now we analyze the running time of $\proc{Insertion-Sort}$.
  Let $t_i$ denote the number of times the \While loop test in line 4 is
  executed for that value of $i$.
  Then the running time $T(n)$ is given by
  \begin{equation*}
    T(n) = \Theta\left(\sum_{i=2}^n t_i\right).
  \end{equation*}
  We have $T(n) = O(n^2)$ since $t_i \leq i$ for each $i \in \{2, \dots, n\}$.
  If the original array is strictly decreasing, then $t_i = i$ for each $i \in
  \{2, \dots, n\}$, and we have $T(n) = \Omega(n^2)$ in this case.
  Thus, $T(n)  = \Theta(n^2)$ in the worst case, which completes the proof.
\end{proof}

\section{Heap Sort}
This section is under construction.
\begin{codebox}
  \Procname{$\proc{Max-Heapify}(A, i, n)$}
  \li $l \gets 2i$
  \li $r \gets 2i + 1$
  \li $k \gets i$
  \li \If $l \leq n$ and $A[l] > A[i]$ \Then
  \li     $k \gets l$
      \End
  \li \If $r \leq n$ and $A[r] > A[i]$ \Then
  \li     $k \gets r$
      \End
  \li \If $k \neq i$ \Then
  \li     swap $A[i]$ and $A[k]$
  \li     $\proc{Max-Heapify}(A, k)$
      \End
\end{codebox}

\begin{codebox}
  \Procname{$\proc{Build-Max-Heap}(A, n)$}
  \li \For $i \gets \lfloor n / 2 \rfloor$ \Downto $1$ \Do
  \li     $\proc{Max-Heapify}(A, i, n)$
      \End
\end{codebox}

\begin{codebox}
  \Procname{$\proc{Heap-Sort}(A, n)$}
  \li $\proc{Build-Max-Heap}(A, n)$
  \li \For $i \gets n$ \Downto $2$ \Do
  \li     swap $A[1]$ and $A[i]$
  \li     $\proc{Max-Heapify}(A, 1, i - 1)$
      \End
\end{codebox}