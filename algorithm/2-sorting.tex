\chapter{Sorting}
\section{Insertion Sort}
In this chapter, we focus on the sorting problem.
An algorithm that solves the sorting problem is usually called a sorting
algorithm.
\begin{problem}[Sorting Problem]
  \leavevmode
  \begin{itemize}
    \item Input: An array $A[1 \twodots n]$ of numbers.
    \item Output: A permutation of $A$ that is nondecreasing.
  \end{itemize}
\end{problem}

\begin{algorithm}
  $\proc{Insertion-Sort}$ is an efficient sorting algorithm if the size of
  input array is small.
  \leavevmode
  \begin{codebox}
    \Procname{$\proc{Insertion-Sort}(A[1 \twodots n])$}
    \li \For $i \gets 2$ \To $n$ \Do
    \li     $\tau \gets A[i]$
    \li     $j \gets i$
    \li     $\phi \gets \const{true}$
    \li     \While $\phi$ \Do
    \li         \If $j = 1$ or $A[j-1] \leq \tau$ \Then
    \li             $\phi \gets \const{false}$
    \li         \Else
    \li             $A[j] \gets A[j-1]$
    \li             $j \gets j-1$
                \End
            \End
    \li     $A[j] \gets \tau$
        \End
  \end{codebox}
\end{algorithm}

\begin{theorem}
  The algorithm $\proc{Insertion-Sort}$ correctly solves the sorting problem.
\end{theorem}
\begin{proof}
  We prove the loop invariant that at the start of each iteration of the \For
  loop of lines 1 -- 11, the subarray $A[1 \twodots i-1]$ is a nondecreasing
  permutation of the elements originally in $A[1 \twodots i-1]$.
  The loop invariant is trivially true for $i = 2$, and we show that each
  iteration maintains the loop invariant.

  First, we set $\tau \gets A[i]$ and $j \gets i$.
  Then the \While loop of lines 5 -- 10 maintains the loop invariant that
  at the start of each iteration, $A[1 \twodots j-1]$ remains unchanged, and
  the elements in $A[j+1 \twodots i]$ are the elements originally in
  $A[j \twodots i-1]$, each at its corresponding position.
  It can be shown that when the \While loop of lines 5 -- 10 terminates,
  each element in $A[1 \twodots j-1]$ is less than or equal to $A[j]$, and each
  element in $A[j+1 \twodots i]$ is greater than $A[j]$.
  Thus, after we set $A[j] \gets \tau$, the subarray $A[1 \twodots i]$ becomes
  a sorted permutation of the elements originally in $A[1 \twodots i]$,
  implying that the loop invariant holds after the increment of $i$.

  When the \For loop of lines 1 -- 11 terminates, we have $i = n+1$.
  Due to the loop invariant, the entire array is a nondecreasing permutation of
  the original input array, which completes the proof.
\end{proof}

\begin{theorem}
  The worst-case running time of $\proc{Insertion-Sort}$ is $\Theta(n^2)$.
\end{theorem}
\begin{proof}
  It is easy to verify that the \While loop of lines 5 -- 10 takes $O(i)$ time.
  Thus, the overall running time is $O(n^2)$.

  However, if the input array is strictly decreasing, then the \While loop of
  lines 5 -- 10 will takes $\Omega(i)$ time.
  In this case, the overall running time is $\Omega(n^2)$.
  Thus, the worst-case running time of $\proc{Insertion-Sort}$ is
  $\Theta(n^2)$.
\end{proof}

\section{Heapsort}
\begin{definition}
  A \emph{binary heap} is a complete binary tree such that the value of each
  node is not less than the values of its children.

  We can use an array to represent a complete binary tree, such that $A[1]$
  is the root of the tree, and $A[2i]$ and $A[2i+1]$ are the left child and the
  right child of $A[i]$.
\end{definition}

\begin{algorithm}
  Suppose that $A[1 \twodots n]$ is an array representing a complete binary
  tree.
  If the subtrees rooted at $A[2i]$ and $A[2i+1]$ are already heapfied, then we
  can use $\proc{Heapify-Down}$ to heapify the subtree rooted at $A[i]$.
  \begin{codebox}
    \Procname{$\proc{Heapify-Down}(A[1 \twodots n], i)$}
    \li $\phi \gets \const{true}$
    \li \While $\phi$ \Do
    \li     $\ell \gets 2i$
    \li     $r \gets 2i+1$
    \li     $j \gets i$
    \li     \If $\ell \leq n$ and $A[\ell] > A[j]$ \Then
    \li         $j \gets \ell$
            \End
    \li     \If $r \leq n$ and $A[r] > A[j]$ \Then
    \li         $j \gets r$
            \End
    \li     \If $j = i$ \Then
    \li         $\phi \gets \const{false}$
    \li     \Else
    \li         swap $A[i]$ and $A[j]$
    \li         $i \gets j$
            \End
        \End
  \end{codebox}
\end{algorithm}

\begin{codebox}
  \Procname{$\proc{Heapify-Up}(A[1 \twodots n], i)$}
  \li $j \gets \lfloor i/2 \rfloor$
  \li \While $j \geq 1$ and $A[i] > A[j]$ \Do
  \li     swap $A[i]$ and $A[j]$
  \li     $i \gets j$
  \li     $j \gets \lfloor j/2 \rfloor$
      \End
\end{codebox}

\begin{codebox}
  \Procname{$\proc{Heapsort}(A[1 \twodots n])$}
  \li \For $i \gets \lfloor n/2 \rfloor$ \Downto $1$ \Do
  \li     $\proc{Heapify-Down}(A, i)$
      \End
  \li \For $j \gets n$ \Downto $2$ \Do
  \li     swap $A[1]$ and $A[j]$
  \li     $\proc{Heapify-Down}(A[1 \twodots j-1], 1)$
      \End
\end{codebox}