\chapter{Vector Spaces}
\section{Algebraic Strutures}
\begin{definition}
  A nonempty set $G$ and a binary operation $\star: G \times G \to G$
  form a \emph{group} $(G, \star)$, if the following conditions
  (G1), (G2), and (G3) hold:
  \begin{enumerate}[(G1)]
    \item For all $a, b, c \in G$,
      $(a \star b) \star c = a \star (b \star c)$.
    \item There exists an \emph{identity element} $e \in G$ such that
      $a \star e = a$ and $e \star a = a$ hold for all $a \in G$. 
    \item For all $a \in G$, there exists $a' \in G$ such that
      $a \star a' = e$, where $e$ is an identity element of $G$.
  \end{enumerate}
\end{definition}
\begin{definition}
  A group $(G, \star)$ is called an \emph{Abelian group} if the following
  condition (G4) holds:
  \begin{enumerate}[(G1),start=4]
    \item For elements $a, b \in G$, $a \star b = b \star a$.
  \end{enumerate}
\end{definition}
\begin{example}
  $(\mathbb{Z}, +)$ is an Abelian group.
\end{example}