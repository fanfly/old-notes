\chapter{Vector Spaces}
\section{Algebraic Strutures}
\begin{definition}
  A binary operation $\star$ on a set $G$ is called \emph{associative} if
  for all $a, b, c \in G$, $(a \star b) \star c = a \star (b \star c)$ holds.
\end{definition}

\begin{remark}
  The product $a_1 \star a_2 \star \cdots \star a_n$ can be uniquely defined
  if $\star$ is associative.
\end{remark}

\begin{definition}
  Let $G$ be a set and $\star$ be a binary operation on $G$. An
  \emph{identity} of $G$ with respect to $\star$ is an element $e \in G$ such
  that $a \star e = a$ and $e \star a = a$ for all $a \in G$.
\end{definition}

\begin{observation}
  The identity of $G$ with respect to $\star$ is unique if it exists.
\end{observation}
\begin{proof}
  If $e$ and $e'$ are identity of $G$ with respect to $\star$, then
  $e = e \star e' = e'$.
\end{proof}

\begin{definition}
  Let $\star$ be a binary operation on $G$ with identity $e$. Let $a$ be an
  element of $G$. An element $b \in G$ is called an \emph{inverse} of $a$ if
  $a \star b = e$ and $b \star a = e$.
\end{definition}

\begin{observation}
  Let $\star$ be an associative binary operation on $G$ with identity $e$.
  Then for all $a \in G$, the inverse of $a \in G$ is unique if it exists.
\end{observation}
\begin{proof}
  If both $b$ and $b'$ are inverses of $a$, then
  $$
  b = b \star e = b \star (a \star b') = (b \star a) \star b'
    = e \star b' = b',
  $$
  so $b = b'$.
\end{proof}

\begin{definition}
  A set $G$ and a binary operation $\star$ on $G$ form a \emph{group}
  $(G, \star)$ if the following conditions hold.
  \begin{itemize}
    \item $\star$ is associative.
    \item $G$ has an identity with respect to $\star$.
    \item Every element of $G$ has an inverse.
  \end{itemize}
\end{definition}

\begin{example}
  Let $S$ denote the set of permutations of $\{1, 2, 3\}$ and $\circ$
  denote the composition of permutations. Then $(S, \circ)$ is a group.
\end{example}

\begin{definition}
  A binary operation $\star$ on a set $G$ is called \emph{commutative} if
  for all $a, b, \in G$, $a \star b = b \star a$ holds.
\end{definition}

\begin{definition}
  A group $(G, \star)$ is called an \emph{Abelian group} if $\star$ is
  commutative.
\end{definition}

\begin{example}
  $(\mathbb{Z}, +)$ and $(\mathbb{Q} \setminus \{0\}, \cdot)$ are Abelian
  groups.
\end{example}