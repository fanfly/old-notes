\chapter{Vector Spaces}
\section{Algebraic Strutures}
\subsection{Groups and Abelian Groups}
\begin{definition}
  A binary operation $\star$ on a set $G$ is called \emph{associative} if
  for all $a, b, c \in G$, $(a \star b) \star c = a \star (b \star c)$ holds.
\end{definition}

\begin{definition}
  Let $G$ be a set and $\star$ be a binary operation on $G$. An
  \emph{identity} of $G$ with respect to $\star$ is an element $e \in G$ such
  that $a \star e = a$ and $e \star a = a$ for all $a \in G$.
\end{definition}

\begin{theorem}
  The identity of $G$ with respect to $\star$ is unique if it exists.
\end{theorem}
\begin{proof}
  If $e$ and $e'$ are identity of $G$ with respect to $\star$, then
  $e = e \star e' = e'$.
\end{proof}

\begin{notation}
  The identity of $G$ is denoted by $1_G$.
  However, if the binary operation is written additively, the identity is
  denoted by $0_G$ instead.
\end{notation}

\begin{definition}
  Let $\star$ be a binary operation on $G$ with identity $e$. Let $a$ be an
  element of $G$. An element $b \in G$ is called an \emph{inverse} of $a$ if
  $a \star b = e$ and $b \star a = e$.
\end{definition}

\begin{theorem}
  For all $a \in G$, the inverse of $a \in G$ is unique if it exists.
\end{theorem}
\begin{proof}
  If both $b$ and $b'$ are inverses of $a$, then
  \begin{align*}
  b
  = b \star 1_G
  = b \star (a \star b')
  = (b \star a) \star b'
  = 1_G \star b'
  = b'. &\qedhere
  \end{align*}
\end{proof}

\begin{notation}
  The inverse of $a$ in $G$ is denoted by $a^{-1}$.
  However, if the binary operation is written additively, the inverse of $a$
  is denoted by $-a$ instead.
\end{notation}

\begin{definition}
  A set $G$ and a binary operation $\star$ on $G$ form a \emph{group}
  $(G, \star)$ if the following conditions hold.
  \begin{enumerate}[(a)]
    \item The binary operation $\star$ is associative.
    \item $1_G$ exists.
    \item For all $a \in G$, $a^{-1}$ exists.
  \end{enumerate}
\end{definition}

\begin{example}
  Let $S$ denote the set of permutations of $\{1, 2, 3\}$ and $\circ$
  denote the composition of permutations. Then $(S, \circ)$ is a group.
\end{example}

\begin{definition}
  A binary operation $\star$ on a set $G$ is called \emph{commutative} if
  for all $a, b, \in G$, $a \star b = b \star a$ holds.
\end{definition}

\begin{definition}
  A group $(G, \star)$ is called an \emph{Abelian group} if $\star$ is
  commutative.
\end{definition}

\begin{example}
  $(\mathbb{Z}, +)$ and $(\mathbb{Q} \setminus \{0\}, \cdot)$ are Abelian
  groups.
\end{example}

\begin{theorem}
  Let $(G, \star)$ be a group. Then for all $a \in G$, $(a^{-1})^{-1} = a$.
\end{theorem}
\begin{proof}
  Since $a \star a^{-1} = 1_G$, $a$ is the inverse of $a^{-1}$ in $G$.
  Thus, $(a^{-1})^{-1} = a$.
\end{proof}

\begin{theorem}{(Cancellation Law)}
  Let $(G, \star)$ be a group. Then the following statements are true.
  \begin{enumerate}[(a)]
    \item For all $a, b, c \in G$, if $c \star a = c \star b$, then $a = b$.
    \item For all $a, b, c \in G$, if $a \star c = b \star c$, then $a = b$.
  \end{enumerate}
\end{theorem}
\begin{proof} \leavevmode
  \begin{enumerate}[(a)]
    \item We have
      $$
      a = 1_G \star a = (c^{-1} \star c) \star a = c^{-1} \star (c \star a)
      $$
      and
      $$
      b = 1_G \star b = (c^{-1} \star c) \star b = c^{-1} \star (c \star b).
      $$
      Because $c \star a = c \star b$, we have $a = b$.
    \item The proof is similar to (a). \qedhere
  \end{enumerate}
\end{proof}

\subsection{Fields}
\begin{definition}
  Let $F$ be a set. Let $+$ and $\cdot$ be binary operations on $F$.
  The operation $\cdot$ is called \emph{distributive} over the operation $+$
  if
  $$
  a \cdot (b + c) = a \cdot b + a \cdot c
  $$
  and
  $$
  (b + c) \cdot a = b \cdot a + c \cdot a
  $$
  hold for all $a, b, c \in F$.
\end{definition}

\begin{definition}
  A set $F$ and two binary operations $+$ and $\cdot$ on $F$ form a
  \emph{field} $(F, +, \cdot)$ if the following conditions hold.
  \begin{itemize}
    \item $(F, +)$ is an Abelian group.
    \item $(F \setminus \{0_F\}, \cdot)$ is an Abelian group.
    \item The operation $\cdot$ is distributive over the operation $+$.
  \end{itemize}
\end{definition}

\begin{example}
  $(\mathbb{Q}, +, \cdot)$, $(\mathbb{R}, +, \cdot)$,
  and $(\mathbb{C}, +, \cdot)$ are fields.
\end{example}

\begin{example}
  $(\mathbb{Q}[\sqrt{2}], +, \cdot)$ is a field, where
  $$
  \mathbb{Q}[\sqrt{2}] = \{a + b\sqrt{2} : a, b \in \mathbb{Q}\}.
  $$
\end{example}

\begin{theorem}\label{thm:field}
  Let $(F, +, \cdot)$ be a field. Then the following statements are true.
  \begin{enumerate}[(a)]
    \item For all $a \in F$, $a \cdot 0_F = 0_F = 0_F \cdot a$.
    \item For all $a, b \in F$, $(-a) \cdot b = -(a \cdot b) = a \cdot (-b)$.
    \item For all $a, b \in F$, $(-a) \cdot (-b) = a \cdot b$.
  \end{enumerate}
\end{theorem}
\begin{proof} \leavevmode
  \begin{enumerate}[(a)]
    \item Since
      $$
      a \cdot 0_F + a \cdot 0_F
      = a \cdot (0_F + 0_F)
      = a \cdot 0_F
      = a \cdot 0_F + 0_F,
      $$
      we have $a \cdot 0_F = 0_F$ by cancellation law.
      The proof of $0_F \cdot a = 0_F$ is similar.
    \item By (a), we have
      $$
      a \cdot b + (-a) \cdot b
      = (a + (-a)) \cdot b
      = 0_F \cdot b
      = 0_F.
      $$
      Thus, $(-a) \cdot b = -(a \cdot b)$.
      The proof of $a \cdot (-b) = -(a \cdot b)$ is similar.
    \item We have
      $$
      (-a) \cdot (-b) = -(a \cdot (-b)) = -(-(a \cdot b)) = a \cdot b
      $$
      by applying (b) twice. \qedhere
  \end{enumerate}
\end{proof}

\begin{remark}
  Let $G = F \setminus \{0_F\}$ and $1_F$ be the multiplicative identity of
  $G$.
  By Theorem \ref{thm:field} (a), we have
  $1_G \cdot 0_F = 0_F = 0_F \cdot 1_G$.
  Therefore, $1_G$ is also the multiplicative identity of $F$, and thus we
  denote it by $1_F$.
\end{remark}

\begin{remark}
  Subtraction and division are defined in terms of addition and
  multiplication.
  That is, for all $a, b \in F$, we have $a - b = a + (-b)$ and
  $a / b = a \cdot b^{-1}$.
\end{remark}