\chapter{Linear Transformations}
\section{Linear Transformations}
\begin{definition}
  \label{def:bijection}
  Let $X$ and $Y$ be sets.
  Let $f: X \to Y$ be a function.
  \begin{itemize}
    \item $f$ is \emph{injective} if $T(x) = T(x')$ implies $x = x'$
    for all $x, x' \in X$.
    \item $f$ is \emph{surjective} if there exists $x \in X$ with $T(x) = y$
    for each $y \in Y$.
    \item $f$ is \emph{bijective} if $f$ is injective and surjective.
  \end{itemize}
\end{definition}

\begin{definition}
  \label{def:linear-transformation}
  Let $V$ and $W$ be vector spaces over a field $F$.
  A transformation $T: V \to W$ is said to be \emph{linear} if
  \begin{equation*}
    T(ax + y) = aT(x) + T(y)
  \end{equation*}
  holds for any scalar $a \in F$ and any vectors $x, y \in V$.
  The set of all linear transformations from $V$ to $W$ is denoted by
  $\mathcal{L}(V, W)$, and in the case that $V = W$ we write $\mathcal{L}(V)$
  for short.
\end{definition}

\begin{theorem}
  \label{thm:linear-transformation-space}
  If $V$ and $W$ are vector spaces over a field $F$, then $\mathcal{L}(V, W)$
  is also a vector space over $F$.
\end{theorem}
\begin{proof}
  For any $c \in F$ and $T_1, T_2 \in \mathcal{L}(V, W)$, since
  \begin{align*}
    (cT_1 + T_2)(ax + y)
    &= cT_1(ax + y) + T_2(ax + y) \tag{linearity of $cT_1 + T_2$} \\
    &= c(aT_1(x) + T_1(y)) + (aT_2(x) + T_2(y)) \tag{linearity of $T_1$ and $T_2$} \\
    &= acT_1(x) + cT_1(y) + aT_2(x) + T_2(y) \\
    &= a(cT_1(x) + T_2(x)) + (cT_1(y) + T_2(y)) \\
    &= a(cT_1 + T_2)(x) + (cT_1 + T_2)(y) \tag{linearity of $cT_1 + T_2$}
  \end{align*}
  holds for each $a \in F$ and $x, y \in V$,
  we have $cT_1 + T_2 \in \mathcal{L}(V, W)$.
  Furthermore, $0_{\mathcal{F}(V, W)} \in \mathcal{L}(V, W)$.
  Thus, $\mathcal{L}(V, W)$ is a subspace of $\mathcal{F}(V, W)$.
\end{proof}

\begin{theorem}\label{thm:linear-span}
  Let $V$ and $W$ be vector spaces over $F$, and let $T: V \to W$ be linear.
  Let $S$ be a subset of $V$ and let $U$ be a subspace of $V$.
  Then the following statements are true.
  \begin{enumerate}
    \item If $n$ is a nonnegative integer, then for $a_1, \dots, a_n \in F$
      and $x_1, \dots, x_n \in V$, we have
      \begin{equation*}
        T\left(\sum_{i=1}^n a_ix_i\right) = a_i\sum_{i=1}^n T(x_i).
      \end{equation*}
    \item If $S$ spans $U$, then $T(S)$ spans $T(U)$.
  \end{enumerate}
\end{theorem}
\begin{proof} \leavevmode
  \begin{enumerate}
    \item The proof is by induction on $n$. For $n = 0$, it holds trivially.
      If the statement is true for some $n \geq 0$, then we have
      \begin{align*}
        T(a_1x_1 + \cdots + a_nx_n + a_{n+1}x_{n+1})
        &= T(a_1x_1 + \cdots + a_nx_n) + T(a_{n+1}x_{n+1}) \\
        &= a_1T(x_1) + \cdots + a_nT(x_n) + a_{n+1}T(x_{n+1}).
      \end{align*}
      Thus, the statement is true for nonnegative integer $n$.
    \item We prove that $\mathrm{span}(T(S)) = T(U)$.
      If $y \in \mathrm{span}(T(S))$, then there exist
      $a_i \in F$, $x_i \in S$ for $i \in \{1, \dots, n\}$ such that
      \begin{equation*}
        y = \sum_{i=1}^n a_iT(x_i)
          = T\left(\sum_{i=1}^n a_ix_i\right) \in T(U),
      \end{equation*}
      so $\mathrm{span}(T(S)) \subseteq T(U)$.

      If $y \in T(U)$, then there exist
      $a_i \in F$, $x_i \in S$ for $i \in \{1, \dots, n\}$ such that
      \begin{equation*}
        y = T\left(\sum_{i=1}^n a_ix_i\right)
          = \sum_{i=1}^n a_iT(x_i) \in \mathrm{span}(T(S)),
      \end{equation*}
      so $T(U) \subseteq \mathrm{span}(T(S))$.
      Thus, $\mathrm{span}(T(S)) = T(U)$. \qedhere
  \end{enumerate}
\end{proof}

\section{Rank and Nullity}
\begin{definition}
  \label{def:range}
  Let $V$ and $W$ be vector spaces.
  The \emph{range} of a transformation $T: V \to W$, denoted by
  $\mathcal{R}(T)$, is defined by
  \begin{equation*}
    \mathcal{R}(T) = \{y \in W : \text{$y = T(x)$ for some $x \in V$}\}.
  \end{equation*}
\end{definition}

\begin{proposition}
  \label{prop:range}
  Let $V$ and $W$ be vector spaces over a field $F$.
  If $T: V \to W$ is linear, then $\mathcal{R}(T)$ is a subspace of $W$.
\end{proposition}
\begin{proof}
  For each $a \in F$ and $y, y' \in \mathcal{R}(T)$, there exist $x, x' \in V$
  such that $y = T(x)$ and $y' = T(x')$.
  Since
  \begin{equation*}
    ay + y' = aT(x) + T(x') = T(ax + x'),
  \end{equation*}
  we have $ay + y' \in \mathcal{R}(T)$.
  Furthermore, $0_W = T(0_V) \in \mathcal{R}(T)$.
  Thus, $\mathcal{R}(T)$ is a subspace of $W$.
\end{proof}

\begin{definition}
  \label{def:null-space}
  Let $V$ and $W$ be vector spaces.
  The \emph{null space} of a transformation $T: V \to W$, denoted by
  $\mathcal{N}(T)$, is defined by
  \begin{equation*}
    \mathcal{N}(T) = \{x \in V : T(x) = 0_W\}.
  \end{equation*}
\end{definition}

\begin{proposition}
  \label{prop:null-space}
  Let $V$ and $W$ be vector spaces over a field $F$.
  If $T: V \to W$ is linear, then $\mathcal{N}(T)$ is a subspace of $V$.
\end{proposition}
\begin{proof}
  \label{prop:null-space}
  For each $a \in F$ and $x, x' \in \mathcal{N}(T)$, we have
  \begin{equation*}
    T(ax + x') = aT(x) + T(x') = a0_W + 0_W = 0_W.
  \end{equation*}
  Thus, $ax + x' \in \mathcal{N}(T)$.
  Furthermore, $0_V \in \mathcal{N}(T)$ since $T(0_V) = 0_W$.
  Thus, $\mathcal{N}(T)$ is a subspace of $V$.
\end{proof}

\begin{definition}
  \label{def:rank-nullity}
  Let $V$ and $W$ be vector spaces. Let $T: V \to W$ be linear.
  \begin{itemize}
    \item The \emph{rank} of $T$, denoted by $\rank(T)$,
      is the dimension of $\mathcal{R}(T)$.
    \item The \emph{nullity} of $T$, denoted by $\nullity(T)$,
      is the dimension of $\mathcal{N}(T)$.
  \end{itemize}
\end{definition}

\begin{theorem}[Rank-nullity Theorem]
  \label{thm:rank-nullity}
  Let $V$ and $W$ be finite-dimensional vector spaces over $F$.
  Let $T: V \to W$ be linear.
  Then we have
  \begin{equation*}
    \nullity(T) + \rank(T) = \dim(V).
  \end{equation*}
\end{theorem}
\begin{proof}
  Let $S$ be a basis for $V$ and $Q$ a basis for $\mathcal{N}(T)$.
  By corollary to replacement theorem (\Cref{thm:replacement}), there is
  $R \subseteq S \setminus Q$ such that $Q \cup R$ is a basis for $V$.
  Since $|R| = |Q \cup R| - |Q| = \mathrm{dim}(V) - \mathrm{nullity}(T)$,
  the theorem holds if $|R| = \mathrm{dim}(\mathcal{R}(T))$.
  
  If there exist different $x, x' \in R$ with $T(x) = T(x')$, then we have
  $T(x - x') = T(x) - T(x') = 0_W$, and thus
  $x - x' \in \mathcal{N}(T) = \mathrm{span}(Q)$.
  It follows that $x \in \mathrm{span}(Q \cup \{x'\})$,
  contradiction to the fact that $S$ is linearly independent.
  Thus, $|R| = |T(R)|$. We claim that $T(R)$ is a basis for $\mathcal{R}(T)$.

  First we prove that $T(R)$ spans $\mathcal{R}(T)$.
  By \Cref{thm:linear-span} (b) and the fact that $T(Q) = \{0_V\}$, we have
  \begin{align*}
    \mathcal{R}(T) &= T(\mathrm{span}(Q \cup R)) \\
                   &= \mathrm{span}(T(Q \cup R)) \\
                   &= \mathrm{span}(T(Q)) + \mathrm{span}(T(R)) \\
                   &= \mathrm{span}(T(R)).
  \end{align*}

  Then we prove that $T(R)$ is linearly independent.
  Suppose that
  \begin{equation*}
    a_1T(x_1) + \cdots + a_nT(x_n) = 0_W
  \end{equation*}
  holds for some $a_1, \dots, a_n \in F$ and some different
  $x_1, \dots, x_n \in R$ with $n \geq 1$.
  Then by \Cref{thm:linear-span} we have $T(a_1x_1 + \cdots + a_nx_n) = 0_W$,
  and thus $a_1x_1 + \cdots a_nx_n \in \mathcal{N}(T)$.
  Hence, there exist some $b_1, \dots, b_m \in F$ and some different
  $y_1, \dots, y_m \in Q$ such that
  \begin{equation*}
    a_1x_1 + \cdots + a_nx_n = b_1y_1 + \cdots + b_my_m.
  \end{equation*}
  That is,
  \begin{equation*}
    a_1x_1 + \cdots + a_nx_n + (-b_1)y_1 + \cdots + (-b_m)y_m = 0_V.
  \end{equation*}
  Since $Q \cup R$ is linearly independent, we have
  $a_1 = \cdots = a_n = b_1 = \cdots = b_m = 0_F$, implying that $T(R)$ is
  linearly independent.

  Thus, $T(R)$ is a basis for $\mathcal{R}(T)$, and we can conclude that
  $\mathrm{rank}(T) = |T(R)| = |R| = |Q \cup R| - |Q|$,
  which completes the proof.
\end{proof}

\section{Isomorphisms}
\begin{definition}
  \label{def:composition}
  Let $X, Y, Z$ be sets.
  Let $f: X \to Y$ and $g: Y \to Z$ be functions.
  Then the \emph{composition} of $f$ and $g$ is the function $gf: X \to Z$ such
  that
  \begin{equation*}
    (gf)(x) = g(f(x))
  \end{equation*}
  for all $x \in X$.
\end{definition}

\begin{proposition}
  \label{prop:associativity-of-composition}
  Let $W, X, Y, Z$ be sets.
  Let
  \begin{equation*}
    f: W \to X, \quad g: X \to Y, \quad h: Y \to Z
  \end{equation*}
  be functions.
  Then we have
  \begin{equation*}
    (hg)f = h(gf).
  \end{equation*}
\end{proposition}
\begin{proof}
  Since
  \begin{align*}
    ((hg)f)(w)
    &= (hg)(f(w)) \tag{composition of $f$ and $hg$} \\
    &= h(g(f(w))) \tag{composition of $g$ and $h$} \\
    &= h((gf)(w)) \tag{composition of $f$ and $g$} \\
    &= (h(gf))(w) \tag{composition of $gf$ and $h$}
  \end{align*}
  holds for all $w \in W$, we have $(hg)f = h(gf)$.
\end{proof}

\begin{definition}
  \label{def:identity-function}
  The \emph{identity function} over a set $X$ is a function $I_X: X \to X$
  with $I_X(x) = x$ for all $x \in X$.
\end{definition}

\begin{definition}
  \label{def:invertibility}
  Let $X$ and $Y$ be sets.
  A function $f: X \to Y$ is said to be \emph{invertible} if there exists a
  function $f^{-1}: Y \to X$, called the \emph{inverse} of $f$, such that
  \begin{equation*}
    f^{-1}f = I_X
    \quad \text{and} \quad
    ff^{-1} = I_Y.
  \end{equation*}
\end{definition}

\begin{proposition}
  \label{prop:inverse}
  Let $X$ and $Y$ be sets.
  Let $f: X \to Y$ and $g: Y \to X$ be functions.
  \begin{enumerate}
    \item If $f$ is invertible, then $f^{-1}$ is invertible.
    \item If $f$ is invertible, then $f^{-1}$ is linear.
    \item If $f$ is invertible, then either $gf = I_X$ or $fg = I_Y$ implies
    $g = f^{-1}$.
    \item $f$ is invertible if and only if $f$ is bijective.
  \end{enumerate}
\end{proposition}
\begin{proof}
  \leavevmode
  \begin{enumerate}
    \item Straightforward from \Cref{def:invertibility}.
    
    \item For $a \in F$ and $y, y' \in Y$, we have
    \begin{align*}
      f^{-1}(ay + y')
      &= f^{-1}(af(f^{-1}(y)) + f(f^{-1}(y'))) \tag{$ff^{-1} = I_Y$} \\
      &= f^{-1}(f(af^{-1}(y) + f^{-1}(y'))) \tag{linearity of $f$} \\
      &= af^{-1}(y) + f^{-1}(y'). \tag{$f^{-1}f = I_X$}
    \end{align*}
    Thus, $f^{-1}$ is linear.
    
    \item
    If $gf = I_X$, then
    \begin{equation*}
      g = gI_Y = g(ff^{-1}) = (gf)f^{-1} = I_Xf^{-1} = f^{-1}.
    \end{equation*}
    If $fg = I_Y$, then
    \begin{equation*}
      g = I_Xg = (f^{-1}f)g = f^{-1}(fg) = f^{-1}I_Y = f^{-1}.
    \end{equation*}
    
    \item
    ($\Rightarrow$)
    Suppose that $f$ is invertible.
    Then $f$ is injective since for each $x, x' \in X$ with $f(x) = f(x')$,
    we have
    \begin{equation*}
      x = (f^{-1}f)(x) = f^{-1}(f(x)) = f^{-1}(f(x')) = (f^{-1}f)(x') = x'.
    \end{equation*}
    Also, $f$ is surjective since for each $y \in Y$, we have
    \begin{equation*}
      y = (ff^{-1})y = f(f^{-1}(y)).
    \end{equation*}

    ($\Leftarrow$)
    If $f$ is bijective, then for each $y \in Y$ there exists a unique element
    $x \in X$ with $f(x) = y$.
    Thus, there exists a function $g: Y \to X$ such that
    \begin{equation*}
      g(f(x)) = x
    \end{equation*}
    for each $x \in X$.
    For any $y \in Y$, if $x \in X$ is the element such that $f(x) = y$,
    then we have
    \begin{equation*}
      f(g(y)) = f(g(f(x))) = f(x) = y.
    \end{equation*}
    Thus, $f$ is invertible since $gf = I_X$ and $fg = I_Y$. \qedhere
  \end{enumerate}
\end{proof}

\begin{definition}
  \label{def:isomorphism}
  Let $V$ and $W$ be vector spaces.
  An \emph{isomorphism} from $V$ onto $W$ is a invertible linear transformation
  from $V$ to $W$.
  If there is an isomorphism from $V$ onto $W$, then $V$ and $W$ are said to be
  \emph{isomorphic}, denoted by $V \cong W$.
\end{definition}

\begin{lemma}
  \label{lem:same-dimension}
  Let $V$ and $W$ be finite-dimensional vector spaces with $\dim(V) = \dim(W)$.
  Let $T: V \to W$ be linear.
  Then $T$ is injective if and only if $T$ is surjective.
\end{lemma}
\begin{proof}
  ($\Rightarrow$)
  If $T$ is injective, then $\mathcal{N}(T) = \{0_V\}$, implying
  $\nullity(T) = 0$.
  Then we have
  \begin{equation*}
    \dim(\mathcal{R}(T))
    = \rank(T)
    = \dim(V) - \nullity(T)
    = \dim(W) - 0
    = \dim(W).
  \end{equation*}
  Since $\mathcal{R}(T)$ is a subspace of $W$ with
  $\dim(\mathcal{R}(T)) = \dim(W)$, we can conclude that $\mathcal{R}(T) = W$
  by \Cref{prop:subspace-dimension}.

  ($\Leftarrow$)
  If $T$ is surjective, then $\mathcal{R}(T) = W$.
  Thus,
  \begin{equation*}
    \nullity(T)
    = \dim(V) - \rank(T)
    = \dim(W) - \dim(W)
    = 0,
  \end{equation*}
  implying $\mathcal{N}(T) = \{0_V\}$.
  It follows that $T$ is injective.
\end{proof}

\begin{lemma}
  \label{lem:linear-uniqueness}
  Let $V$ and $W$ be finite-dimensional vector spaces over a field $F$.
  Let $S = \{x_1, x_2, \dots, x_n\}$ be a basis for $V$ and let
  $y_1, y_2, \dots, y_n$ be vectors in $W$.
  Then there exists a unique $T \in \mathcal{L}(V, W)$ with $T(x_i) = y_i$
  for each $i \in \{1, \dots, n\}$.
\end{lemma}
\begin{proof}
  Let $T$ be the transformation that satisfies
  \begin{equation*}
    T(a_1x_1 + a_2x_2 + \cdots + a_nx_n) = a_1y_1 + a_2y_2 + \cdots + a_ny_n
  \end{equation*}
  for any $a_1, a_2, \dots, a_n \in F$.
  It is obvious that $T(x_i) = y_i$ for each $i \in \{1, \dots, n\}$,
  and $T$ is linear since
  \begin{align*}
    T\left(c\sum_{i=1}^na_ix_i + \sum_{i=1}^nb_ix_i\right)
    &= T\left(\sum_{i=1}^n(ca_i + b_i)x_i\right) \\
    &= \sum_{i=1}^n(ca_i + b_i)y_i \\
    &= c\sum_{i=1}^na_iy_i + \sum_{i=1}^nb_iy_i \\
    &= cT\left(\sum_{i=1}^na_ix_i\right) + T\left(\sum_{i=1}^nb_ix_i\right)
  \end{align*}
  holds for any scalars $a_1, a_2, \dots, a_n, b_1, b_2, \dots, b_n, c \in F$.
  To see the uniqueness, if $T' \in \mathcal{L}(V, W)$ satisfies
  $T'(x_i) = y_i$ for each $i \in \{1, \dots, n\}$, then we have
  \begin{align*}
    T'(a_1x_1 + \cdots + a_nx_n)
    &= a_1T'(x_1) + \cdots + a_nT'(x_n) \\
    &= a_1T(x_1) + \cdots + a_nT(x_n) \\
    &= T(a_1x_1 + \cdots + a_nx_n). \\
  \end{align*}
  for any $a_1, \dots, a_n \in F$.
  Thus, $T' = T$.
\end{proof}

\begin{theorem}
  \label{thm:isomorphism}
  Let $V$ and $W$ be finite-dimensional vector spaces over a field $F$.
  Then $V \cong W$ if and only if $\dim(V) = \dim(W)$.
\end{theorem}
\begin{proof}
  $(\Rightarrow)$
  Let $T$ be an isomorphism from $V$ onto $W$.
  Since $T$ is invertible, $T$ is bijective.
  Then we have $\rank(T) = \dim(W)$ since $\mathcal{R}(T) = W$.
  Furthermore, since $T$ is injective, we have $\nullity(T) = 0$,
  and it follows that $\rank(T) = \dim(V)$ by rank-nullity theorem
  (\Cref{thm:rank-nullity}).
  Thus, $\dim(V) = \rank(T) = \dim(W)$.

  $(\Leftarrow)$
  Suppose that $S = \{x_1, x_2, \dots, x_n\}$ is a basis for $V$ and
  $R = \{y_1, y_2, \dots, y_n\}$ is a basis for $W$.
  Then by \Cref{lem:linear-uniqueness} there exists $T \in \mathcal{L}(V, W)$
  such that $T(x_i) = y_i$ for each $i \in \{1, \dots, n\}$.
  Since $R$ is a basis for $W$, for each $y \in W$ there exist
  scalars $a_1, \dots, a_n \in F$ such that
  \begin{equation*}
    y
    = \sum_{i=1}^n a_iy_i
    = \sum_{i=1}^n a_iT(x_i)
    = T\left(\sum_{i=1}^n a_ix_i\right).
  \end{equation*}
  It follows that $T$ is surjective, and we can conclude that $T$ is bijective
  by \Cref{lem:same-dimension}.
  Thus, $T$ is an isomorphism from $V$ onto $W$, implying $V \cong W$.
\end{proof}

\section{Coordinates and Matrix Representations}
\begin{definition}
  \label{def:ordered-basis}
  Let $V$ be an finite-dimensional vector space over a field $F$ with
  $\dim(V) = n$.
  An \emph{ordered basis} for $V$ is a finite sequence
  \begin{equation*}
    \beta = (x_1, x_2, \dots, x_n)
  \end{equation*}
  of vectors in $V$ such that the set $S = \{x_1, x_2, \dots, x_n\}$ is a basis
  for $V$.
\end{definition}

\begin{examples}
  \leavevmode
  \begin{itemize}
    \item The \emph{standard ordered basis} for $F^n$ is $(e_1, \dots, e_n)$,
    where $e_i$ is the $n$-tuple whose $i$-th component is $1_F$ and the other
    components are all $0_F$.
    \item The \emph{standard ordered basis} for $\mathcal{P}_n(F)$ is
    $(t^0, t^1, \dots, t^n)$.
  \end{itemize}
\end{examples}

\begin{definition}
  \label{def:coordinate}
  Let $V$ be a finite-dimensional vector space over a field $F$.
  Let $\beta = (x_1, \dots, x_n)$ be an ordered basis for $V$.
  Then we define $\phi_\beta: V \to F^n$ by
  \begin{equation*}
    \phi_\beta(a_1x_1 + a_2x_2 + \cdots + a_nx_n)
    = \begin{pmatrix} a_1 \\ a_2 \\ \vdots \\ a_n \end{pmatrix}
  \end{equation*}
  for any $a_1, a_2, \dots, a_n \in F$.
  For each vector $x$ in $V$, $\phi_\beta(x)$ is called the \emph{coordinate}
  of $x$ with respect to $\beta$, denoted by $[x]_\beta$.
\end{definition}

\begin{proposition}
  \label{prop:coordinate}
  Let $\beta = (x_1, \dots, x_n)$ be an ordered basis for a vector space $V$
  over $F$.
  Then $\phi_\beta$ is an isomorphism from $V$ onto $F^n$.
\end{proposition}
\begin{proof}
  $\phi_\beta$ is linear since
  \begin{align*}
    \phi_\beta\left(c\sum_{i=1}^na_ix_i + \sum_{i=1}^nb_ix_i\right)
    &= \phi_\beta\left(\sum_{i=1}^n(ca_i + b_i)x_i\right)
     = \begin{pmatrix} ca_1 + b_1 \\ \vdots \\ ca_n + b_n \end{pmatrix}
     = c\begin{pmatrix} a_1 \\ \vdots \\ a_n \end{pmatrix}
       + \begin{pmatrix} b_1 \\ \vdots \\ b_n \end{pmatrix} \\
    &= c \cdot \phi_\beta\left(\sum_{i=1}^na_ix_i\right)
       + \phi_\beta\left(\sum_{i=1}^nb_ix_i\right)
  \end{align*}
  holds for any $a_1, \dots, a_n, b_1, \dots, b_n, c \in F$.
  Also, $\phi_\beta$ is invertible since there exists
  $\phi_\beta^{-1}: F^n \to V$ with
  \begin{equation*}
    \phi_\beta^{-1} \left(
      \begin{pmatrix} a_1 \\ a_2 \\ \vdots \\ a_n \end{pmatrix}
    \right)
    = a_1x_1 + a_2x_2 + \cdots + a_nx_n
  \end{equation*}
  for any $a_1, a_2, \dots, a_n \in F$.
  Thus, $\phi_\beta$ is an isomorphism.
\end{proof}

\begin{definition}
  \label{def:matrix-representation}
  Let $V$ and $W$ be finite-dimensional vector spaces over a field $F$.
  Let
  \begin{equation*}
    \beta = (x_1, \dots, x_n)
    \quad \text{and} \quad
    \gamma = (y_1, \dots, y_m)
  \end{equation*}
  be ordered basis for $V$ and $W$, respectively.
  Then we define $\Phi_\beta^\gamma: \mathcal{L}(V, W) \to F^{m \times n}$
  by
  \begin{equation*}
    \Phi_\beta^\gamma(T) =
    \begin{pmatrix}
      a_{11} & a_{12} & \cdots & a_{1n} \\
      a_{21} & a_{22} & \cdots & a_{2n} \\
      \vdots & \vdots & \ddots & \vdots \\
      a_{m1} & a_{m2} & \cdots & a_{mn}
    \end{pmatrix},
  \end{equation*}
  for each $T \in \mathcal{L}(V, W)$, where
  \begin{equation*}
    \begin{gathered}
      T(x_1) = a_{11}y_1 + a_{21}y_2 + \cdots + a_{m1}y_m \\
      T(x_2) = a_{12}y_1 + a_{22}y_2 + \cdots + a_{m2}y_m \\
      \vdots \\
      T(x_n) = a_{1n}y_1 + a_{2n}y_2 + \cdots + a_{mn}y_m
    \end{gathered}
  \end{equation*}
  hold.
  For each linear $T: V \to W$, the matrix $\Phi_\beta^\gamma(T)$ is called the
  \emph{matrix representation} of $T$ with respect to $\beta$ and $\gamma$,
  denoted by $[T]_\beta^\gamma$.
\end{definition}

\begin{proposition}
  \label{prop:matrix-representation}
  Let $\beta = (x_1, \dots, x_n)$ and $\gamma = (y_1, \dots, y_m)$ be ordered
  bases for a vector spaces $V$ and $W$ over $F$, respectively.
  Then for any $T \in \mathcal{L}(V, W)$, we have
  \begin{equation*}
    \Bigl([T]_\beta^\gamma\Bigr)_{ij} = \Bigl([T(x_j)]_\gamma\Bigr)_i
  \end{equation*}
  for any $i \in \{1, \dots, m\}$ and $j \in \{1, \dots, n\}$.
\end{proposition}
\begin{proof}
  Let
  \begin{equation*}
    [T]_\beta^\gamma =
    \begin{pmatrix}
      a_{11} & a_{12} & \cdots & a_{1n} \\
      a_{21} & a_{22} & \cdots & a_{2n} \\
      \vdots & \vdots & \ddots & \vdots \\
      a_{m1} & a_{m2} & \cdots & a_{mn}
    \end{pmatrix}.
  \end{equation*}
  Since $T(x_j) = a_{1j}y_1 + a_{2j}y_2 + \cdots + a_{mj}y_m$, we have
  \begin{equation*}
    [T(x_j)]_\gamma
    = \begin{pmatrix} a_{1j} \\ a_{2j} \\ \vdots \\ a_{mj} \end{pmatrix}.
  \end{equation*}
  Thus,
  \begin{equation*}
    \Bigl([T(x_j)]_\gamma\Bigr)_i = a_{ij}
  \end{equation*}
  holds, which completes the proof.
\end{proof}

\begin{theorem}
  \label{thm:matrix-representation}
  Let $\beta = (x_1, \dots, x_n)$ and $\gamma = (y_1, \dots, y_m)$ be ordered
  bases for a vector spaces $V$ and $W$ over $F$, respectively.
  Then $\Phi_\beta^\gamma$ is an isomorphism from $\mathcal{L}(V, W)$ onto
  $F^{m \times n}$.
\end{theorem}
\begin{proof}
  $\Phi_\beta^\gamma$ is linear since for any $c \in F$ and
  $T_1, T_2 \in \mathcal{L}(V, W)$, we have
  \begin{align*}
    \Bigl([cT_1 + T_2]_\beta^\gamma\Bigr)_{ij}
    &= \Bigl([(cT_1 + T_2)(x_j)]_\gamma\Bigr)_i
       \tag{\Cref{prop:matrix-representation}} \\
    &= \Bigl([cT_1(x_j) + T_2(x_j)]_\gamma\Bigr)_i \\
    &= \Bigl(c[T_1(x_j)]_\gamma + [T_2(x_j)]_\gamma\Bigr)_i
       \tag{$\phi_\gamma$ is linear} \\
    &= c\Bigl([T_1(x_j)]_\gamma\Bigr)_i + \Bigl([T_2(x_j)]_\gamma\Bigr)_i \\
    &= c\Bigl([T_1]_\beta^\gamma\Bigr)_{ij}
       + \Bigl([T_2]_\beta^\gamma\Bigr)_{ij}
       \tag{\Cref{prop:matrix-representation}} \\
    &= \Bigl(c[T_1]_\beta^\gamma + [T_2]_\beta^\gamma\Bigr)_{ij}
  \end{align*}
  for any $i \in \{1, \dots, m\}$ and $j \in \{1, \dots, n\}$.
  The following is to be completed.
\end{proof}