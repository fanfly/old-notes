\chapter{Diagonalization}
\section{Eigenvalues and Eigenvectors}
\begin{definition}
  Let $T: V \to V$ be a linear operator on a vector space $V$ over a field $F$.
  If
  \begin{equation*}
    T(x) = \lambda x
  \end{equation*}
  holds for some scalar $\lambda \in F$ and some vector
  $x \in V \setminus \{0_V\}$, then $(\lambda, x)$ is called an
  \emph{eigenpair} of $T$, with $\lambda$ and $x$ called an
  \emph{eigenvalue} and an \emph{eigenvector} of $T$, respectively.
\end{definition}

\begin{definition}
  Let $V$ be a finite-dimensional vector space over a field $F$.
  Let $T \in \mathcal{L}(V)$.
  An \emph{eigenbasis} of $V$ for $T$ is an ordered basis of $V$ in which every
  vector is an eigenvector of $T$.
\end{definition}

\begin{theorem}
  \label{thm:diagonalization}
  Let $V$ be a vector space over a field $F$ and let $T: V \to V$ be linear.
  Let $\beta = (x_1, x_2, \dots, x_n)$ be an ordered basis for $T$.
  Let $\lambda_1, \lambda_2, \dots, \lambda_n \in F$ be scalars.
  Then
  \begin{equation*}
    [T]_\beta^\beta =
    \begin{pmatrix}
      \lambda_1 & 0 & \cdots & 0 \\
      0 & \lambda_2 & \cdots & 0 \\
      \vdots & \vdots & \ddots & \vdots \\
      0 & 0 & \cdots & \lambda_n
    \end{pmatrix}
  \end{equation*}
  if and only if
  $T(x_i) = \lambda_ix_i$ for each $i \in \{1, 2, \dots, n\}$.
\end{theorem}
\begin{proof}
  ($\Rightarrow$)
  For each $i \in \{1, 2, \dots, n\}$, we have
  \begin{equation*}
    [T(x_i)]_\beta = \lambda_ie_i = [\lambda_ix_i]_\beta.
  \end{equation*}
  Thus, $T(x_i) = \lambda_ix_i$.
  ($\Leftarrow$)
  For each $i \in \{1, 2, \dots, n\}$, we have
  $[T(x_i)]_\beta = [\lambda_ix_i]_\beta = \lambda_ie_i$,
  and it follows that
  \begin{equation*}
    [T]_\beta^\beta =
    \begin{pmatrix}
      \lambda_1 & 0 & \cdots & 0 \\
      0 & \lambda_2 & \cdots & 0 \\
      \vdots & \vdots & \ddots & \vdots \\
      0 & 0 & \cdots & \lambda_n
    \end{pmatrix}.
    \qedhere
  \end{equation*}
\end{proof}

\begin{corollary}
  Let $V$ be a finite-dimensional vector space and let $T: V \to V$ be linear.
  Let $\beta$ be an ordered basis of $T$.
  Then $[T]_\beta^\beta$ is diagonal if and only if $\beta$ is an eigenbasis
  of $V$ for $T$.
\end{corollary}
\begin{proof}
  Straightforward from \Cref{thm:diagonalization}.
\end{proof}

\begin{definition}
  Let $A \in F^{n \times n}$.
  If
  \begin{equation*}
    Ax = \lambda x
  \end{equation*}
  holds for some scalar $\lambda \in F$ and some vector
  $x \in V \setminus \{0_V\}$, then $(\lambda, x)$ is called an
  \emph{eigenpair} of $A$, with $\lambda$ and $x$ called an
  \emph{eigenvalue} and an \emph{eigenvector} of $A$, respectively.
\end{definition}

\begin{theorem}
  Let $V$ be a finite-dimensional vector space with an ordered basis $\beta$.
  Let $\lambda \in F$ be a scalar and $x \in V$ be a vector.
  Then $(\lambda, x)$ is an eigenpair of $T$ if and only if
  $(\lambda, [x]_\beta)$ is an eigenpair of $[T]_\beta^\beta$.
\end{theorem}
\begin{proof}
  ($\Rightarrow$)
  Suppose that $T(x) = \lambda x$.
  Then we have
  \begin{equation*}
    [T]_\beta^\beta[x]_\beta
    = [T(x)]_\beta
    = [\lambda x]_\beta
    = \lambda [x]_\beta.
  \end{equation*}
  ($\Leftarrow$)
  Since
  \begin{equation*}
    [T(x)]_\beta
    = [T]_\beta^\beta[x]_\beta
    = \lambda[x]_\beta
    = [\lambda x]_\beta,
  \end{equation*}
  we can conclude that $T(x) = \lambda x$.
\end{proof}

\section{Characteristic Polynomials and Eigenspaces}
\begin{definition}
  The \emph{characteristic polynomial} of $A \in F^{n \times n}$ is the
  polynomial $f_A \in \mathcal{P}(F)$ with
  \begin{equation*}
    f_A(t) = \det(A - tI_n).
  \end{equation*}
\end{definition}

\begin{definition}
  Let $V$ be a vector space over a field $F$.
  Let $T: V \to V$ be linear and let $\lambda$ be an eigenvalue of $T$.
  Then
  \begin{equation*}
    E_T(\lambda) = \mathcal{N}(T - \lambda I_V).
  \end{equation*}
  is called the \emph{eigenspace} of $T$ with respect to $\lambda$.
\end{definition}