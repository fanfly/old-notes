\chapter{Diagonalization}
\section{Eigenvalues and Eigenvectors}
\begin{definition}
  Let $T: V \to V$ be a linear operator on a vector space $V$ over a field $F$.
  If
  \begin{equation*}
    T(x) = \lambda x
  \end{equation*}
  holds for some scalar $\lambda \in F$ and some vector
  $x \in V \setminus \{0_V\}$, then $(\lambda, x)$ is called an
  \emph{eigenpair} of $T$, with $\lambda$ and $x$ called an
  \emph{eigenvalue} and an \emph{eigenvector} of $T$, respectively.
\end{definition}

\begin{definition}
  Let $V$ be a finite-dimensional vector space over a field $F$.
  Let $T \in \mathcal{L}(V)$.
  An \emph{eigenbasis} of $V$ for $T$ is an ordered basis of $V$ in which every
  vector is an eigenvector of $T$.
\end{definition}

\begin{theorem}
  \label{thm:diagonalization}
  Let $V$ be a vector space over a field $F$ and let $T: V \to V$ be linear.
  Let $\beta = (x_1, x_2, \dots, x_n)$ be an ordered basis for $T$.
  Let $\lambda_1, \lambda_2, \dots, \lambda_n \in F$ be scalars.
  Then
  \begin{equation*}
    [T]_\beta^\beta =
    \begin{pmatrix}
      \lambda_1 & 0 & \cdots & 0 \\
      0 & \lambda_2 & \cdots & 0 \\
      \vdots & \vdots & \ddots & \vdots \\
      0 & 0 & \cdots & \lambda_n
    \end{pmatrix}
  \end{equation*}
  if and only if
  $T(x_i) = \lambda_ix_i$ for each $i \in \{1, 2, \dots, n\}$.
\end{theorem}
\begin{proof}
  ($\Rightarrow$)
  For each $i \in \{1, 2, \dots, n\}$, we have
  \begin{equation*}
    [T(x_i)]_\beta = \lambda_ie_i = [\lambda_ix_i]_\beta.
  \end{equation*}
  Thus, $T(x_i) = \lambda_ix_i$.
  ($\Leftarrow$)
  For each $i \in \{1, 2, \dots, n\}$, we have
  $[T(x_i)]_\beta = [\lambda_ix_i]_\beta = \lambda_ie_i$,
  and it follows that
  \begin{equation*}
    [T]_\beta^\beta =
    \begin{pmatrix}
      \lambda_1 & 0 & \cdots & 0 \\
      0 & \lambda_2 & \cdots & 0 \\
      \vdots & \vdots & \ddots & \vdots \\
      0 & 0 & \cdots & \lambda_n
    \end{pmatrix}.
    \qedhere
  \end{equation*}
\end{proof}

\begin{corollary}
  Let $V$ be a finite-dimensional vector space and let $T: V \to V$ be linear.
  Let $\beta$ be an ordered basis of $T$.
  Then $[T]_\beta^\beta$ is diagonal if and only if $\beta$ is an eigenbasis
  of $V$ for $T$.
\end{corollary}
\begin{proof}
  Straightforward from \Cref{thm:diagonalization}.
\end{proof}

\begin{definition}
  Let $A \in F^{n \times n}$.
  If
  \begin{equation*}
    Ax = \lambda x
  \end{equation*}
  holds for some scalar $\lambda \in F$ and some vector
  $x \in V \setminus \{0_V\}$, then $(\lambda, x)$ is called an
  \emph{eigenpair} of $A$, with $\lambda$ and $x$ called an
  \emph{eigenvalue} and an \emph{eigenvector} of $A$, respectively.
\end{definition}

\begin{theorem}
  Let $V$ be a finite-dimensional vector space with an ordered basis $\beta$.
  Let $\lambda \in F$ be a scalar and $x \in V$ be a vector.
  Then $(\lambda, x)$ is an eigenpair of $T$ if and only if
  $(\lambda, [x]_\beta)$ is an eigenpair of $[T]_\beta^\beta$.
\end{theorem}
\begin{proof}
  ($\Rightarrow$)
  Suppose that $T(x) = \lambda x$.
  Then we have
  \begin{equation*}
    [T]_\beta^\beta[x]_\beta
    = [T(x)]_\beta
    = [\lambda x]_\beta
    = \lambda [x]_\beta.
  \end{equation*}
  ($\Leftarrow$)
  Since
  \begin{equation*}
    [T(x)]_\beta
    = [T]_\beta^\beta[x]_\beta
    = \lambda[x]_\beta
    = [\lambda x]_\beta,
  \end{equation*}
  we can conclude that $T(x) = \lambda x$.
\end{proof}

\section{Characteristic Polynomials and Eigenspaces}
\begin{theorem}
  \label{thm:eigenvalue-condition}
  Let $A \in F^{n \times n}$ be a matrix and let $\lambda \in F$ be a scalar.
  Then $\lambda$ is an eigenvalue of $A$ if and only if
  $\det(A - \lambda I_n) = 0_F$.
\end{theorem}
\begin{proof}
  The proof is as follows.
  \begin{align*}
    \text{$\lambda$ is an eigenvalue of $A$}
    &\quad \Leftrightarrow \quad
    \text{$Ax = \lambda x$ for some $x \in F^n \setminus \{0_{F^n}\}$} \\
    &\quad \Leftrightarrow \quad
    \text{$(A - \lambda I_n)x$ for some $x \in F^n \setminus \{0_{F^n}\}$} \\
    &\quad \Leftrightarrow \quad
    \text{$(A - \lambda I_n)$ is not invertible} \\
    &\quad \Leftrightarrow \quad \det(A - \lambda I_n) = 0_F.
    \qedhere
  \end{align*}
\end{proof}

\begin{definition}
  The \emph{characteristic polynomial} of $A \in F^{n \times n}$ is the
  polynomial $f_A \in \mathcal{P}(F)$ with
  \begin{equation*}
    f_A(t) = \det(A - tI_n).
  \end{equation*}
\end{definition}

\begin{theorem}
  Let $A, B \in F^{n \times n}$. If $A \sim B$, then $f_A = f_B$.
\end{theorem}
\begin{proof}
  To be completed.
\end{proof}

\begin{definition}
  Let $V$ be a finite-dimensional vector space over $F$ with $\dim(V) = n$.
  For all $T \in \mathcal{L}(V)$, we define the
  \emph{characteristic polynomial} of $T$ as
  \begin{equation*}
    f_T = f_A,
  \end{equation*}
  where $A = [T]_\beta^\beta$ for some arbitrary ordered basis $\beta$ of $V$.
\end{definition}

\begin{theorem}
  Let $V$ be a vector space over a field $F$ and let $T \in \mathcal{L}(V)$.
  For any $\lambda \in F$ and $x \in V \setminus \{0_V\}$, $(\lambda, x)$ is
  an eigenpair of $T$ if and only if $x \in \mathcal{N}(T - \lambda I_V)$.
\end{theorem}
\begin{proof}
  The proof is as follows.
  \begin{align*}
    \text{$(\lambda, x)$ is an eigenpair of $T$}
    &\quad \Leftrightarrow \quad T(x) = \lambda x \\
    &\quad \Leftrightarrow \quad T(x) = (\lambda I_V)(x) \\
    &\quad \Leftrightarrow \quad (T - \lambda I_V)(x) = 0_V \\
    &\quad \Leftrightarrow \quad x \in \mathcal{N}(T - \lambda I_V).
    \qedhere
  \end{align*}
\end{proof}

\begin{definition}
  Let $V$ be a vector space over a field $F$.
  Let $T: V \to V$ be linear and let $\lambda$ be an eigenvalue of $T$.
  Then
  \begin{equation*}
    E_T(\lambda) = \mathcal{N}(T - \lambda I_V).
  \end{equation*}
  is called the \emph{eigenspace} of $T$ with respect to $\lambda$.
\end{definition}

\section{Diagonalizability}

\section{Cayley-Hamilton Theorem}
\begin{definition}
  Let $V$ be a vector space and let $T \in \mathcal{L}(V)$.
  A subspace $W$ of $V$ is a \emph{$T$-invariant subspace} of $V$ if
  \begin{equation*}
    T(W) \subseteq W.
  \end{equation*}
\end{definition}

\begin{theorem}
  Let $V$ be a finite-dimensional vector space and let $T: V \to V$ be linear.
  Let $W$ be a $T$-invariant subspace of $V$ and define $T': W \to W$ as the
  transformation such that $T'(x) = T(x)$ for any $x \in W$.
  Then we have
  \begin{equation*}
    f_{T'}(t) \mid f_T(t).
  \end{equation*}
\end{theorem}
\begin{proof}
  Let $\gamma = (x_1, \dots, x_k)$ be an ordered basis of $W$.
  By replacement theorem (\Cref{thm:replacement}), there is an ordered basis
  $\beta = (x_1, \dots, x_k, x_{k+1}, \dots, x_n)$ of $V$.
  It can be shown that
  \begin{equation*}
    [T]_\beta^\beta =
    \begin{pmatrix}
      [T']_\gamma^\gamma & X \\
      O & Y
    \end{pmatrix}
  \end{equation*}
  for some $X \in F^{k \times (n-k)}$ and $Y \in F^{(n-k) \times (n-k)}$.
  Thus, we have
  \begin{align*}
    f_T(t)
    &= \det([T]_\beta^\beta - tI_n) \\
    &= \det
    \begin{pmatrix}
      [T']_\gamma^\gamma - tI_k & X \\
      O & Y - tI_{n-k}
    \end{pmatrix} \\
    &= \det([T']_\gamma^\gamma - tI_k) \cdot \det(Y - tI_{n-k}) \\
    &= f_{T'}(t) \cdot \det(Y - tI_{n-k}).
    \qedhere
  \end{align*}
\end{proof}

\begin{definition}
  Let $V$ be a vector space and let $T \in \mathcal{L}(V)$.
  The \emph{$T$-cyclic subspace} of $V$ generated by $x \in V$ is defined as
  \begin{equation*}
    C_T(x) = \spn\left(\bigcup_{i=0}^\infty \{T^i(x)\}\right).
  \end{equation*}
\end{definition}

\begin{theorem}
  Let $V$ be a vector space and let $T \in \mathcal{L}(V)$.
  Then the following statements hold for any $x \in V$.
  \begin{enumerate}
    \item $C_T(x)$ is a $T$-invariant subspace of $V$.
    \item If $W$ is a $T$-invariant subspace of $V$ with $x \in W$, then
    $C_T(x) \subseteq W$.
  \end{enumerate}
\end{theorem}
\begin{proof}
  \leavevmode
  \begin{enumerate}
    \item Suppose that $y \in C_T(x)$ with
    \begin{equation*}
      y = \sum_{i=0}^k a_iT^i(x).
    \end{equation*}
    Then we have
    \begin{equation*}
      T(y)
      = T\left(\sum_{i=0}^k a_iT^i(x)\right)
      = \sum_{i=0}^k a_iT^{i+1}(x)
      \in C_T(x).
    \end{equation*}
    It follows that $T(C_T(x)) \subseteq C_T(x)$, and thus $C_T(x)$ is
    $T$-invariant.
    \item Since $x \in U$ and $T(U) \subseteq U$, we can conclude that
    $T^i(x) \in U$ holds for any nonnegative integer $i$.
    Thus, we have
    \begin{equation*}
      \bigcup_{i=0}^\infty \{T^i(x)\} \subseteq U,
    \end{equation*}
    implying
    \begin{equation*}
      C_T(x)
      = \spn\left(\bigcup_{i=0}^\infty \{T^i(x)\}\right)
      \subseteq U.
      \qedhere
    \end{equation*}
  \end{enumerate}
\end{proof}