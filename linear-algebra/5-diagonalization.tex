\chapter{Diagonalization}
\section{Eigenvalues and Eigenvectors}
\begin{definition}
  Let $T: V \to V$ be a linear operator on a vector space $V$ over a field $F$.
  If
  \begin{equation*}
    T(x) = \lambda x
  \end{equation*}
  holds for some scalar $\lambda \in F$ and some vector
  $x \in V \setminus \{0_V\}$, then $(\lambda, x)$ is called an
  \emph{eigenpair} of $T$, with $\lambda$ and $x$ called an
  \emph{eigenvalue} and an \emph{eigenvector} of $T$, respectively.
\end{definition}

\begin{definition}
  Let $V$ be a vector space over a field $F$ with $\dim(V) = n$.
  Let $T \in \mathcal{L}(V)$.
  An \emph{eigenbasis} of $V$ for $T$ is an ordered basis of $V$ in which every
  vector is an eigenvector of $T$.
\end{definition}

\section{Characteristic Polynomials and Eigenspaces}
\begin{definition}
  The \emph{characteristic polynomial} of $A \in F^{n \times n}$ is the
  polynomial $f_A \in \mathcal{P}(F)$ with
  \begin{equation*}
    f_A(t) = \det(A - tI_n).
  \end{equation*}
\end{definition}

\begin{definition}
  Let $V$ be a vector space over a field $F$.
  Let $T: V \to V$ be linear and let $\lambda$ be an eigenvalue of $T$.
  Then
  \begin{equation*}
    E_T(\lambda) = \mathcal{N}(T - \lambda I_V).
  \end{equation*}
  is called the \emph{eigenspace} of $T$ with respect to $\lambda$.
\end{definition}