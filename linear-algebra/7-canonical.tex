\chapter{Canonical Forms}
\section{Generalized Eigenspaces}
\begin{definition}
  Let $V$ be a vector space over $F$ and let $T: V \to V$ be linear.
  We say that $\lambda \in F$ and $x \in V \setminus \{0_V\}$ form a
  \emph{generalized eigenpair} if
  \begin{equation*}
    (T - \lambda I_V)^\ell(x) = 0_V
  \end{equation*}
  holds for some positive integer $\ell$.
\end{definition}

\begin{theorem}
  Let $V$ be a vector space over $F$ and let $T: V \to V$ be linear.
  If $(\lambda, x)$ is a generalized eigenpair of $T$, then $\lambda$ is an
  eigenvalue of $T$.
\end{theorem}
\begin{proof}
  Let $\ell$ be the smallest positive integer such that 
  $(T - \lambda I_V)^\ell(x) = 0_V$.
  Let
  \begin{equation*}
    y = (T - \lambda I_V)^{\ell-1}(x).
  \end{equation*}
  Since $(T - \lambda I_V)(y) = (T - \lambda I_V)^\ell(x) = 0_V$,
  $(\lambda, y)$ is an eigenpair of $T$, and thus $\lambda$ is an eigenvalue of
  $T$.
\end{proof}

\begin{definition}
  Let $V$ be a vector space over $F$ and let $T: V \to V$ be linear.
  For any scalar $\lambda \in F$, we define
  \begin{equation*}
    G_T(\lambda) = \{x \in V : \text{$(T - \lambda I_V)^\ell(x) = 0_V$ for some
    nonnegative integer $\ell$}\}.
  \end{equation*}
  If $\lambda$ is an eigenvalue of $T$, then $G_T(\lambda)$ is called the
  \emph{generalized eigenspace} of $T$ corresponding to $\lambda$.
\end{definition}

\begin{theorem}
  Let $V$ be a vector space over $F$ and let $T: V \to V$ be linear.
  If scalars $\lambda_1, \lambda_2 \in F$ are distinct, then
  \begin{equation*}
    G_T(\lambda_1) \cap G_T(\lambda_2) = \{0_V\}.
  \end{equation*}
\end{theorem}
\begin{proof}
  Assume $x \in (G_T(\lambda_1) \cap G_T(\lambda_2)) \setminus \{0_V\}$
  for contradiction.
  Let $\ell_1$ be the smallest positive integer with
  \begin{equation*}
    (T - \lambda_1I_V)^{\ell_1}(x) = 0_V.
  \end{equation*}
  Let $y = (T - \lambda_1I_V)^{\ell_1-1}(x)$, and we have
  $(T - \lambda_1I_V)(y) = 0_V$.
  Note that there is a positive integer $\ell_2$ such that
  \begin{equation*}
    (T - \lambda_2I_V)^{\ell_2}(x) = 0_V,
  \end{equation*}
  and it follows that
  \begin{align*}
    (T - \lambda_2I_V)^{\ell_2}(y)
    &= (T - \lambda_2I_V)^{\ell_2}(T - \lambda_2I_V)^{\ell_1-1}(x) \\
    &= (T - \lambda_1I_V)^{\ell_1-1}(T - \lambda_2I_V)^{\ell_2}(x) \\
    &= 0_V.
  \end{align*}
  Thus we can define $\ell_2'$ as the smallest positive integer such that
  \begin{equation*}
    (T - \lambda_2I_V)^{\ell_2'}(y) = 0_V.
  \end{equation*}
  Let $z = (T - \lambda_2I_V)^{\ell_2'-1}(y)$, and we have
  $(T - \lambda_2I_V)(z) = 0_V$.
  Furthermore,
  \begin{align*}
    (T - \lambda_1I_V)(z)
    &= (T - \lambda_1I_V)(T - \lambda_2I_V)^{\ell_2'}(y) \\
    &= (T - \lambda_2I_V)^{\ell_2'}(T - \lambda_1I_V)(y) \\
    &= 0_V.
  \end{align*}
  Thus, $z \in (E_T(\lambda_1) \cap E_T(\lambda_2)) \setminus \{0_V\}$,
  contradiction.
\end{proof}

\section{The Jordan Canonical Form}
\begin{definition}
  Let $V$ be a vector space over $F$ and let $T: V \to V$ be linear.
  If $(\lambda, x)$ is a generalized eigenpair and $\ell$ is the smallest
  positive integer such that
  \begin{equation*}
    (T - \lambda I_V)^\ell(x) = 0_V,
  \end{equation*}
  then the ordered set
  \begin{equation*}
    ((T - \lambda I_V)^{\ell-1}(x),
     (T - \lambda I_V)^{\ell-2}(x),
     \dots,
     (T - \lambda I_V)^2(x),
     (T - \lambda I_V)(x),
     x)
  \end{equation*}
  is a \emph{chain} of generalized eigenvectors of $T$ corresponding to
  $\lambda$, where
  \begin{itemize}
    \item $\ell$ is called the \emph{length} of the chain, and
    \item $(T - \lambda I_V)^{\ell-1}(x)$ and $x$ are called the \emph{initial
    vector} and the \emph{end vector} of the chain, respectively.
  \end{itemize}
\end{definition}