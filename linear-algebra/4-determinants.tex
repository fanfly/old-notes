\chapter{Determinants}
\section{Characterization of the Determinant}
\begin{definition}
  \label{def:multilinear}
  A function $\delta: F^{n \times n} \to F$ is \emph{$n$-linear} if for each
  $i \in \{1, \dots, n\}$ and
  $x_1, \dots, x_{i-1}, x_{i+1}, \dots, x_n \in F^n$,
  the function $T: F^n \to F$ with
  \begin{equation*}
    T(y) = \delta
    \begin{pmatrix}
      x_1 \\
      \vdots \\
      x_{i-1} \\
      y \\
      x_{i+1} \\
      \vdots \\
      x_n
    \end{pmatrix}
  \end{equation*}
  for each $y \in F^n$ is linear.
\end{definition}

\begin{definition}
  An $n$-linear function $\delta: F^{n \times n} \to F$ is \emph{alternating}
  if
  \begin{equation*}
    \delta(A) = 0_F
  \end{equation*}
  for each $A \in F^{n \times n}$ that has two identical rows.
\end{definition}

\begin{proposition}
  Let $\delta: F^{n \times n} \to F$ be an alternating $n$-linear function
  and let $A \in F^{n \times n}$.
  Then the following statements are true.
  \begin{enumerate}
    \item If $E_1 \in F^{n \times n}$ is an elementary matrix of type 1,
    then $\delta(E_1A) = -\delta(A)$.
    \item If $E_2 \in F^{n \times n}$ is an elementary matrix of type 2
    obtained by multiplying one row of $I_n$ by scalar $k \in F$,
    then $\delta(E_2A) = k\delta(A)$.
    \item If $E_3 \in F^{n \times n}$ is an elementary matrix of type 3,
    then $\delta(E_3A) = \delta(A)$.
  \end{enumerate}
\end{proposition}
\begin{proof}
  Let $\row(A) = (x_1, \dots, x_n)$.
  \begin{enumerate}
    \item Let $E_1$ be obtained from $I_n$ by interchanging the $p$th row and
    the $q$th row with $p < q$.
    Then we have
    \begin{align*}
      0_F
      &=
      \delta
      \begin{pmatrix}
        x_1 \\ \vdots \\ x_p + x_q \\ \vdots \\ x_p + x_q \\ \vdots \\ x_n
      \end{pmatrix}
      =
      \delta
      \begin{pmatrix}
        x_1 \\ \vdots \\ x_p \\ \vdots \\ x_p + x_q \\ \vdots \\ x_n
      \end{pmatrix}
      +
      \delta
      \begin{pmatrix}
        x_1 \\ \vdots \\ x_q \\ \vdots \\ x_p + x_q \\ \vdots \\ x_n
      \end{pmatrix} \\
      &=
      \delta
      \begin{pmatrix}
        x_1 \\ \vdots \\ x_p \\ \vdots \\ x_p \\ \vdots \\ x_n
      \end{pmatrix}
      +
      \delta
      \begin{pmatrix}
        x_1 \\ \vdots \\ x_p \\ \vdots \\  x_q \\ \vdots \\ x_n
      \end{pmatrix}
      +
      \delta
      \begin{pmatrix}
        x_1 \\ \vdots \\ x_q \\ \vdots \\ x_p \\ \vdots \\ x_n
      \end{pmatrix}
      +
      \delta
      \begin{pmatrix}
        x_1 \\ \vdots \\ x_q \\ \vdots \\ x_q \\ \vdots \\ x_n
      \end{pmatrix} \\
      &= 0_F + \delta(A) + \delta(E_1A) + 0_F.
    \end{align*}
    Thus, $\delta(E_1A) = -\delta(A)$.

    \item Let $E_2$ be obtained from $I_n$ by multiplying the $p$th row
    by some scalar $k$.
    Then we have
    \begin{equation*}
      \delta(E_2A)
      = \delta
      \begin{pmatrix}
        x_1 \\ \vdots \\ kx_p \\ \vdots \\ x_n
      \end{pmatrix}
      = k\delta
      \begin{pmatrix}
        x_1 \\ \vdots \\ x_p \\ \vdots \\ x_n
      \end{pmatrix}
      = k\delta(A).
    \end{equation*}

    \item Let $E_3$ be obtained from $I_n$ by adding the $p$th row multiplied
    by some scalar $k$ to the $q$th row.
    If $p < q$, then we have
    \begin{equation*}
      \delta(E_3A)
      =
      \delta
      \begin{pmatrix}
        x_1 \\ \vdots \\ x_p \\ \vdots \\ kx_p + x_q \\ \vdots \\ x_n
      \end{pmatrix}
      =
      k\delta
      \begin{pmatrix}
        x_1 \\ \vdots \\ x_p \\ \vdots \\ x_p \\ \vdots \\ x_n
      \end{pmatrix}
      +
      \delta
      \begin{pmatrix}
        x_1 \\ \vdots \\ x_p \\ \vdots \\ x_q \\ \vdots \\ x_n
      \end{pmatrix}
      = k0_F + \delta(A)
      = \delta(A).
    \end{equation*}
    The case that $q < p$ can be proved similarly.
    \qedhere
  \end{enumerate}
\end{proof}

\begin{theorem}
  Let $\delta: F^{n \times n} \to F$ be an alternating $n$-linear function
  and let $A \in F^{n \times n}$.
  If $\rank(A) < n$, then $\delta(A) = 0_F$.
\end{theorem}
\begin{proof}
  Since
  \begin{equation*}
    \dim(\spn(\row(A))) = \rank(A^t) = \rank(A) < n,
  \end{equation*}
  the rows of $A$ is not a spanning set of $F^n$, and thus is linearly
  dependent, implying that there exists a row which is a linear combination of
  the other rows.

  Therefore, $A$ can be transformed into a matrix $B$ that has two identical
  rows by a finite number of elementary row operations.
  It follows that
  \begin{equation*}
    \delta(A) = \delta(E_p \cdots E_1A) = \delta(B) = 0_F,
  \end{equation*}
  where $E_1, \dots, E_p \in F^{n \times n}$ are elementary matrices.
\end{proof}

\begin{theorem}
  Let $\delta: F^{n \times n} \to F$ be an alternating $n$-linear function
  such that $\delta(I_n) = 1_F$.
  Then for any $A, B \in F^{m \times n}$, we have
  \begin{equation*}
    \delta(AB) = \delta(A)\delta(B).
  \end{equation*}
\end{theorem}
\begin{proof}
  First, suppose that $\rank(A) < n$.
  Then we have $\rank(AB) < n$.
  Thus,
  \begin{equation*}
    \delta(AB) = 0_F = \delta(A)\delta(B).
  \end{equation*}
  Now suppose that $\rank(A) = n$.
  That is, $A$ is invertible, and thus $A = E_k \cdots E_1$ for some elementary
  matrices $E_1, \dots, E_k \in F^{n \times n}$.
  Then we have
  \begin{align*}
    \delta(AB)
    &= \delta(E_k \cdots E_1B) \\
    &= \delta(E_k) \cdots \delta(E_1)\delta(B) \\
    &= \delta(E_k) \cdots \delta(E_1)\delta(I_n)\delta(B)
       \tag{$\delta(I_n) = 1_F$} \\
    &= \delta(E_k \cdots E_1I_n)\delta(B) \\
    &= \delta(A)\delta(B).
    \qedhere
  \end{align*}
\end{proof}

\begin{theorem}
  For any alternating $n$-linear functions ${\delta: F^{n \times n} \to F}$ and
  ${\delta': F^{n \times n} \to F}$,
  $\delta(I_n) = \delta'(I_n)$ if and only if $\delta = \delta'$.
\end{theorem}
\begin{proof}
  ($\Leftarrow$) Straightforward.

  ($\Rightarrow$)
  We prove that $\delta(A) = \delta(A')$ for any $A \in F^{n \times n}$.
  If $\rank(A) < n$, then
  \begin{equation*}
    \delta(A) = 0_F = \delta'(A).
  \end{equation*}
  If $\rank(A) = n$, then $A$ is invertible, and thus $A = E_p \cdots E_1$
  for some elementary matrices $E_1, \dots, E_p \in F^{n \times n}$.
  Suppose that $k_1, \dots, k_p \in F$ are scalars such that
  $\delta(E_iA) = k_iA$ for each $1 \leq i \leq p$ and $A \in F^{n \times n}$.
  Then we have
  \begin{align*}
    \delta(A)
    &= \delta(E_p \cdots E_1I_n) \\
    &= k_p \cdots k_1\delta(I_n) \\
    &= k_p \cdots k_1\delta'(I_n) \\
    &= \delta'(E_p \cdots E_1I_n) \\
    &= \delta'(A).
    \qedhere
  \end{align*}
\end{proof}

\begin{definition}
  The \emph{determinant} of $A \in F^{n \times n}$ is
  \begin{equation*}
    \det(A) = \delta(A),
  \end{equation*}
  where $\delta: F^{n \times n} \to F$ is the alternating $n$-linear function
  with $\delta(I_n) = 1_F$.
\end{definition}