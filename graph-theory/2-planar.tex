\chapter{Planar Graphs}
\section{The Jordan Polygon Theorem}
\begin{definition}
  A \emph{line segment} is a set of the form
  \begin{equation*}
    \{\lambda x + (1 - \lambda y): 0 \leq \lambda \leq 1\},
  \end{equation*}
  where $x$ and $y$ are distinct points in $\mathbb{R}^2$, called the
  \emph{endpoints} of the line segment.
\end{definition}

\begin{definition}
  A \emph{polygonal curve} $C$ is a union of line segments such that there is
  a continuous bijection $\gamma: [0, 1] \to C$.
  The points $\gamma(0)$ and $\gamma(1)$ are called the \emph{endpoints} of
  $C$.
\end{definition}

\begin{definition}
  A \emph{polygon} $P$ is a union of line segments such that there is
  a continuous bijection $\gamma: S^1 \to P$, where
  \begin{equation*}
    S^1 = \{(x, y): x^2 + y^2 = 1\}.
  \end{equation*}
\end{definition}

\begin{definition}
  We say that $S \subseteq \mathbb{R}^2$ is \emph{connected} if for any
  $x, y \in S$, there exists a polygonal curve $C \subseteq S$ whose endpoints
  are $x$ and $y$.
  Furthermore, we say that $S$ is a \emph{region} of $Q \subseteq \mathbb{R}^2$
  if $S$ is a maximal connected subset of $Q$.
\end{definition}

\begin{theorem}[Jordan Polygon Theorem]
  Let $P$ be a polygon.
  Then $\mathbb{R}^2 \setminus P$ has exactly two regions.
\end{theorem}

\section{Plane Graphs}
\begin{definition}
  A \emph{plane graph} is a pair $G = (V, E)$, where each components are as
  follows.
  \begin{itemize}
    \item $V$ consists of a finite number of points (called \emph{vertices}) in
    $\mathbb{R}^2$.
    \item $E$ consists of polygonal curves (called \emph{edges})
    that connects vertices, such that no two edges have the same set of
    endpoints, and the interior of any edge does not contain any vertex or any
    point of other edges.
  \end{itemize}
  A region of $\mathbb{R}^2 \setminus \bigcup E$ is called a \emph{face} of
  $G$.
  The set of faces of $G$ is denoted by $F(G)$.
\end{definition}
\begin{remark}
  A plane graph defines a graph in a natural way.
  Thus, we usually use the same notation for both a plane graph and its
  corresponding graph.
\end{remark}