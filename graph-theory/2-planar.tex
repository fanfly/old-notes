\chapter{Planar Graphs}
\section{Topological Prerequisites}
\begin{definition}
  Let $x, y \in \mathbb{R}^2$ be different points.
  \begin{itemize}
    \item A \emph{straight line segment} between $x$ and $y$ is a set
    $\ell \subseteq \mathbb{R}^2$ with
    \begin{equation*}
      \ell = \{x + \lambda(y - x): 0 \leq \lambda \leq 1\}.
    \end{equation*}
    \item A \emph{polygonal arc} between $x$ and $y$ is a set $\alpha \subseteq
    \mathbb{R}^2$ which is a union of finitely many straight line segments such
    that there is a homeomorphism $\varphi: [0, 1] \to \alpha$ with $\varphi(0)
    = x$ and $\varphi(1) = y$.
  \end{itemize}
\end{definition}

\begin{definition}
  Let $S \subseteq \mathbb{R}^2$ beopen and let $\sim$ be the equivalence
  relation of being connected by a polygonal arc.
  The members of $S / {\sim}$ are called the \emph{regions} of $S$.
\end{definition}

\begin{definition}
  Let $S \subseteq \mathbb{R}^2$.
  The \emph{boundary} of $S$ is the set of points whose every neighborhood
  consists of both a point in $S$ and a point not in $S$.
\end{definition}

\section{Plane Graphs}
\begin{definition}
  A \emph{plane graph} is a pair $G = (V, E)$ of finite sets such that the
  following properties hold, where the elements of $V$ and those of $E$ are
  called \emph{vertices} and \emph{edges}, respectively.
  \begin{itemize}
    \item $V$ is a finite subset of $\mathbb{R}^2$.
    \item $E$ is a finite set of simple curves between vertices.
    \item Different edges in $E$ have different set of endpoints.
    \item The interior of an edge contains no vertex and no point of any other
    edge.
  \end{itemize}
  The \emph{faces} of $G$ are the regions of $\mathbb{R}^2 \setminus (V \cup
  \bigcup E)$, and we denote the set of faces of $G$ by $F(G)$.
\end{definition}
\begin{remark}
  A plane graph defines a graph in a natural way.
  Thus, we usually use the same notation for both a plane graph and its
  corresponding graph.
\end{remark}