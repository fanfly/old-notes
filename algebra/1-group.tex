\chapter{Groups}
\section{Groups}
\begin{definition}
  Let $G$ be a nonempty set.
  Let $\star$ be a binary operation such that $a \star b \in G$ for all
  $a, b \in G$.
  We say that $(G, \star)$ is a \emph{semigroup} if
  \begin{equation*}
    (a \star b) \star c = a \star (b \star c)
  \end{equation*}
  holds for any $a, b, c \in G$.
\end{definition}

\begin{definition}
  Let $(G, \star)$ be a semigroup.
  We say that $(G, \star)$ is a \emph{monoid} if there exists an \emph{identity
  element} $e \in G$ such that
  \begin{equation*}
    a \star e = a = e \star a
  \end{equation*}
  for any $a \in G$.
\end{definition}

\begin{theorem}
  The identity element of a monoid is unique.
\end{theorem}
\begin{proof}
  If $e$ and $e'$ are both identity elements of monoid $(G, \star)$, then
  \begin{equation*}
    e = e \star e' = e'.
    \qedhere
  \end{equation*}
\end{proof}

\begin{definition}
  Let $(G, \star)$ be a monoid and let $e$ be the identity element of $G$.
  We say that $(G, \star)$ is a \emph{group} if for any $a \in G$, there exists
  an \emph{inverse} $b \in G$ such that
  \begin{equation*}
    b \star a = e.
  \end{equation*}
\end{definition}
\begin{remark}
  In most cases, $\star$ is either multiplication (denoted by $\cdot$) or
  addition (denoted by $+$).
  \begin{itemize}
    \item If $\star$ is multiplication, then we denote the identity element by
    $1_G$ and denote the inverse of $a$ by $a^{-1}$.
    \item If $\star$ is addition, then we denote the identity element by $0_G$
    and denote the inverse of $a$ by $-a$.
  \end{itemize}
  In the default settings, $\star$ is considered to be a multiplicative
  operation.
\end{remark}

\section{Subgroups}
\begin{definition}
  Let $(G, \star)$ be a group.
  A \emph{subgroup} of $(G, \star)$ is a group $(H, \star)$ with
  $H \subseteq G$.
\end{definition}

\begin{theorem}
  Let $(G, \star)$ be a group and let $H \subseteq G$ be nonempty.
  Then $H$ is a subgroup of $G$ if and only if for any $a, b \in H$,
  \begin{equation*}
    a \star b \in H
    \quad \text{and} \quad
    a^{-1} \in H.
  \end{equation*}
\end{theorem}
\begin{proof}
  ($\Rightarrow$)
  Straightforward.
  ($\Leftarrow$)
  It is obvious that $(H, \star)$ is a semigroup, and we have
  \begin{equation*}
    e = a^{-1} \star a \in H.
  \end{equation*}
  where $a \in H$ is an arbitrary element.
  Thus, $(H, \star)$ is a monoid.
  Since every element of $H$ has an inverse, $(H, \star)$ is a group.
\end{proof}

\begin{definition}
  Let $(G, \star)$ be a group.
  The \emph{cyclic subgroup} generated by $a \in G$, denoted by
  $\langle a \rangle$, is defined by
  \begin{equation*}
    \langle a \rangle = \{a^k: k \in \mathbb{Z}\}.
  \end{equation*}
  If $G = \langle a \rangle$ for some $a \in G$, then we say that $(G, \star)$
  is \emph{cyclic}.
\end{definition}

\section{Cosets}
\begin{definition}
  Let $(G, \star)$ be a group and let $H$ be a subgroup of $G$.
  For any $a \in G$, we define
  \begin{equation*}
    a \star H = \{a \star h: h \in H\}.
  \end{equation*}
\end{definition}

\begin{theorem}
  Let $(G, \star)$ be a group and let $H$ be a subgroup of $G$.
  Let $\sim$ be the relation such that for any $a, b \in G$,
  \begin{equation*}
    a \sim b
    \quad \Leftrightarrow \quad
    a^{-1} \star b \in H.
  \end{equation*}
  Then $\sim$ is an equivalence relation on $G$.
\end{theorem}
\begin{proof}
  Assume $a, b, c \in G$.
  We have $a \sim a$ since $a^{-1} \star a = 1_G \in H$.
  If $a \sim b$, then $b \sim a$ since
  \begin{equation*}
    b^{-1} \star a
    = (a^{-1} \star b)^{-1}
    \in H.
  \end{equation*}
  Moreover, if $a \sim b$ and $b \sim c$, then $a \sim c$ since
  \begin{equation*}
    a^{-1} \star c
    = (a^{-1} \star b) \star (b^{-1} \star c)
    \in H.
    \qedhere
  \end{equation*}
\end{proof}

\section{Homomorphisms}
\begin{definition}
  Let $(G, \star_G)$ and $(H, \star_H)$ be groups.
  A \emph{homomorphism} from $G$ to $H$ is a function $\phi: G \to H$ such that
  \begin{equation*}
    \phi(a \star_G b) = \phi(a) \star_H \phi(b)
  \end{equation*}
  holds for all $a, b \in G$.
\end{definition}

\begin{definition}
  Let $(G, \star_G)$ and $(H, \star_H)$ be groups and let $\phi: G \to H$ be a
  homomorphism.
  \begin{itemize}
    \item If $\phi$ is injective, then $\phi$ is called a \emph{monomorphism}.
    \item If $\phi$ is surjective, then $\phi$ is called a \emph{epimorphism}.
    \item If $\phi$ is bijective, then $\phi$ is called a \emph{isomorphism}.
  \end{itemize}
  We say that $G$ is \emph{isomorphic} to $H$, denoted $G \cong H$, if there
  exists an isomorphism from $G$ to $H$.
\end{definition}
