\chapter{Groups}
\section{Groups}
\begin{definition}
  Let $G$ be a nonempty set and let $\star: G \times G \to G$ be a binary
  operation.
  We say that $(G, \star)$ is a \emph{semigroup} if
  \begin{equation*}
    (a \star b) \star c = a \star (b \star c)
  \end{equation*}
  holds for any $a, b, c \in G$.
\end{definition}

\begin{definition}
  Let $(G, \star)$ be a semigroup.
  We say that $(G, \star)$ is a \emph{monoid} if there exists an \emph{identity
  element} $e \in G$ such that
  \begin{equation*}
    a \star e = a = e \star a
  \end{equation*}
  for any $a \in G$.
\end{definition}

\begin{theorem}
  The identity element of a monoid is unique.
\end{theorem}
\begin{proof}
  If $e$ and $e'$ are both identity elements of monoid $(G, \star)$, then
  \begin{equation*}
    e = e \star e' = e'.
    \qedhere
  \end{equation*}
\end{proof}

\begin{definition}
  Let $(G, \star)$ be a monoid and let $e \in G$ be the identity element.
  We say that $(G, \star)$ is a \emph{group} if for any $a \in G$, there exists
  an \emph{inverse} $b \in G$ such that
  \begin{equation*}
    b \star a = e.
  \end{equation*}
\end{definition}
\begin{remark}
  In most cases, $\star$ is either addition or multiplication.
  \begin{itemize}
    \item If $\star$ is addition, then we denote the identity element by $0_G$
    and denote the inverse of $a$ by $-a$.
    \item If $\star$ is multiplication, then we denote the identity element by
    $1_G$ and denote the inverse of $a$ by $a^{-1}$.
  \end{itemize}
\end{remark}

\section{Subgroups}
\begin{definition}
  Let $(G, \star)$ be a group.
  A \emph{subgroup} of $(G, \star)$ is a group $(H, \diamond)$ such that
  $H \subseteq G$ and $\diamond: H \times H \to H$ is a restriction of $\star$.
\end{definition}

\section{Homomorphisms}
\begin{definition}
  Let $G$ and $H$ be groups.
  A \emph{homomorphism} from $G$ to $H$ is a function $\phi: G \to H$ such that
  \begin{equation*}
    \phi(a \cdot b) = \phi(a) \cdot \phi(b)
  \end{equation*}
  holds for all $a, b \in G$.
\end{definition}

\begin{definition}
  Let $G$ and $H$ be groups and let $\phi: G \to H$ be a homomorphism.
  \begin{itemize}
    \item If $\phi$ is injective, then $\phi$ is called a \emph{monomorphism}.
    \item If $\phi$ is surjective, then $\phi$ is called a \emph{epimorphism}.
    \item If $\phi$ is bijective, then $\phi$ is called a \emph{isomorphism}.
  \end{itemize}
  We say that $G$ is \emph{isomorphic} to $H$, denoted $G \cong H$, if there
  exists an isomorphism from $G$ to $H$.
\end{definition}