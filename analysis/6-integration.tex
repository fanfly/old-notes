\chapter{Integration}
\section{Darboux Integrals}
\begin{definition}
  Let $a, b \in \mathbb{R}$ with $a < b$.
  Let $f$ be a bounded real function defined on $[a, b]$.
  A \emph{partition} of $[a, b]$ is a finite set $P = \{x_0, x_1, \dots, x_n\}$
  with
  \begin{equation*}
    a \leq x_0 < x_1 < \cdots < x_{n-1} < x_n \leq b,
  \end{equation*}
  and we define its corresponding \emph{upper sum} $U(f, P)$ and
  \emph{lower sum} $L(f, P)$ by
  \begin{align*}
    U(f, P) &= \sum_{i=1}^n \sup(\{f(x): x_{i-1} \leq x \leq x_i\})
    (x_i - x_{i-1}) \\
    L(f, P) &= \sum_{i=1}^n \inf(\{f(x): x_{i-1} \leq x \leq x_i\})
    (x_i - x_{i-1}).
  \end{align*}
  The \emph{upper integral} and the \emph{lower integral} of $f$ on $[a, b]$
  are
  \begin{equation*}
    \upint_a^b f
    = \upint_a^b f(x) \, dx
    = \inf(\{U(f, P): \text{$P$ is a parition of $[a, b]$}\})
  \end{equation*}
  and
  \begin{equation*}
    \lowint_a^b f
    = \lowint_a^b f(x) \, dx
    = \sup(\{L(f, P): \text{$P$ is a parition of $[a, b]$}\}),
  \end{equation*}
  respectively.
  If they are equal, then we say that $f$ is \emph{Darboux integrable} on
  $[a, b]$, and the common value is called the \emph{Darboux integral} of $f$
  on $[a, b]$, denoted by
  \begin{equation*}
    \int_a^b f = \int_a^b f(x) \, dx.
  \end{equation*}
\end{definition}

\begin{lemma}
  Let $a, b \in \mathbb{R}$ with $a < b$.
  Let $f$ be a bounded real function defined on $[a, b]$.
  If $P$ and $Q$ are partitions of $[a, b]$ with $P \subseteq Q$, then
  \begin{equation*}
    L(f, P) \leq L(f, Q) \leq U(f, Q) \leq U(f, P).
  \end{equation*}
\end{lemma}

\begin{lemma}
  Let $a, b \in \mathbb{R}$ with $a < b$.
  Let $f$ be a bounded real function defined on $[a, b]$.
  If $P$ and $Q$ are partitions of $[a, b]$, then
  \begin{equation*}
    L(f, P) \leq U(f, Q).
  \end{equation*}
\end{lemma}

\begin{theorem}
  Let $a, b \in \mathbb{R}$ with $a < b$.
  If $f$ is a bounded real function defined on $[a, b]$, then
  \begin{equation*}
    \lowint_a^b f \leq \upint_a^b f.
  \end{equation*}
\end{theorem}
