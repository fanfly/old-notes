\chapter{Continuity}
\section{Limits of Functions}
\begin{definition}
  Let $X$ and $Y$ be a metric spaces and let $f: D \to Y$ be a map with
  $D \subseteq X$.
  Let $a \in X$ be a limit point and $b \in Y$.
  Then we say that $b$ is the \emph{limit} of $f$ at $a$, denoted
  \begin{equation*}
    \lim_{x \to a} f(x) = b,
  \end{equation*}
  if for any $\epsilon > 0$, there exists $\delta > 0$ such that for all $x \in
  D$,
  \begin{equation*}
    0 < d_X(x, a) < \delta
    \quad \Rightarrow \quad
    d_Y(f(x), b) < \epsilon.
  \end{equation*}
\end{definition}

\section{Continuous Functions}
\begin{definition}
  Let $X$ and $Y$ be a metric spaces and let $f: D \to Y$ be a map with
  $D \subseteq X$.
  We say that $f$ is \emph{continuous} at $a \in X$ if for any $\epsilon > 0$,
  there exists $\delta > 0$ such that
  \begin{equation*}
    d_Y(f(x), f(a)) < \epsilon
  \end{equation*}
  holds for any $x \in D$ with
  \begin{equation*}
    d_X(x, a) < \delta.
  \end{equation*}
  Also, we say that $f$ is \emph{continuous} on $D$ if $f$ is continuous at
  every point of $D$.
\end{definition}

\begin{theorem}
  Let $X$ and $Y$ be metric spaces.
  Let $f: X \to Y$ be a map.
  Then $f$ is continuous if and only if $f^{-1}(E)$ is open for any open set
  $E$ in $Y$.
\end{theorem}
\begin{proof}
  To be completed.
\end{proof}

\section{Properties of Continuous Maps}
\begin{theorem}
  Let $X$ and $Y$ be metric spaces, and let $f: X \to Y$ be a continuous map.
  If $K \subseteq X$ is compact, then $f(K)$ is compact.
\end{theorem}
\begin{proof}
  For any open cover $\{V_\alpha\}_{\alpha \in A}$ of $f(K)$, we have
  \begin{equation*}
    K
    \subseteq f^{-1}(f(K))
    \subseteq f^{-1}\left(\bigcup_{\alpha \in A} V_\alpha\right)
    = \bigcup_{\alpha \in A} f^{-1}(V_\alpha).
  \end{equation*}
  Since $f$ is continuous, $\{f^{-1}(V_\alpha)\}_{\alpha \in A}$ is an open
  cover of $K$.
  Due to compactness of $K$, there exist $\alpha_1, \dots, \alpha_m \in A$ such
  that
  \begin{equation*}
    K \subseteq \bigcup_{i=1}^m f^{-1}(V_{\alpha_i}),
  \end{equation*}
  and we have
  \begin{equation*}
    f(K)
    \subseteq f\left(\bigcup_{i=1}^m f^{-1}(V_{\alpha_i})\right)
    = f\left(f^{-1}\left(\bigcup_{i=1}^m V_{\alpha_i}\right)\right)
    = \bigcup_{i=1}^m V_{\alpha_i}.
  \end{equation*}
  Thus, $f(K)$ is compact.
\end{proof}

\begin{theorem}
  Let $X$ be a metric space and let $f: X \to \mathbb{R}$ be a continuous map.
  If $K \subseteq X$ is compact, then $\max(f(K))$ and $\min(f(K))$ exist.
\end{theorem}
\begin{proof}
  Since $f$ is continuous and $K$ is compact, $f(K)$ is a compact subset of
  $\mathbb{R}$.
  Thus, $f(K)$ has maximum and minimum.
\end{proof}

\begin{theorem}[Intermediate Value Theorem]
  Let $f: [a, b] \to \mathbb{R}$ be continuous and let $c \in \mathbb{R}$.
  If $f(a) < c < f(b)$, then $f(x) = c$ for some $x \in (a, b)$.
\end{theorem}
\begin{proof}
  To be completed.
\end{proof}