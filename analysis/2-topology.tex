\chapter{Basic Topology}
\section{Metric Spaces}
\begin{definition}
  A set $X$ with a function $d: X \times X \to \mathbb{R}$ is a
  \emph{metric space} if the following statements hold for any $x, y, z \in X$.
  \begin{enumerate}
    \item $d(x, y) \geq 0$.
    \item $d(x, y) = 0$ if and only if $x = y$.
    \item $d(x, y) = d(y, x)$.
    \item $d(x, y) \leq d(x, z) + d(z, y)$.
  \end{enumerate}
\end{definition}

\begin{definition}
  Let $(X, d)$ be a metric space.
  Let $r > 0$ be a real number and let $x_0 \in X$.
  The \emph{open ball} of radius $r$ centered at $x_0$, denoted by $B_r(x_0)$,
  is defined by
  \begin{equation*}
    B_r(x_0) = \{x \in X: d(x, x_0) < r\}.
  \end{equation*}
\end{definition}

\begin{definition}
  Let $(X, d)$ be a metric space and let $S \subseteq X$.
  \begin{itemize}
    \item $S$ is \emph{open} if for any $x \in S$, there is a real number
    $r > 0$ such that $B_r(x) \subseteq S$.
    \item $S$ is \emph{closed} if $X \setminus S$ is open.
  \end{itemize}
\end{definition}

\begin{theorem}
  Let $(X, d)$ be a metric space.
  \begin{enumerate}
    \item $X$ and $\varnothing$ are open.
    \item If $S_1, S_2$ are open subsets of $X$, then $S_1 \cap S_2$ is open.
    \item If $\{S_i: i \in I\}$ is a collection of open subsets of $X$, then
    \begin{equation*}
      \bigcup_{i \in I} S_i
    \end{equation*}
    is open.
  \end{enumerate}
\end{theorem}

\begin{definition}
  Let $(X, d)$ be a metric space and let $S \subseteq X$.
  \begin{itemize}
    \item A point $x \in X$ is a \emph{limit point} of $S$ if there exists
    $y \in S \setminus \{x\}$ with $d(x, y) < \epsilon$ for any $\epsilon > 0$.
    \item A point $x \in X$ is an \emph{isolated point} of $S$ if $x$ belongs
    to $S$ and is not a limit point of $S$.
  \end{itemize}
\end{definition}

\section{Compact Sets}
\begin{definition}
  Let $(X, d)$ be a metric space and let $S \subseteq X$.
  \begin{itemize}
    \item A \emph{cover} of $S$ is a collection of subsets of $X$ whose union
    contains $S$.
    An \emph{open cover} of $S$ is a cover of $S$ whose elements are all open.
    \item We say that $S$ is \emph{compact} if every open cover $\Omega$ of $S$
    contains a finite cover $\Phi$ of $S$.
  \end{itemize}
\end{definition}

\begin{theorem}
  Let $(X, d)$ be a metric space and let $R \subseteq S \subseteq X$.
  If $S$ is compact and $R$ is closed, then $R$ is compact.
\end{theorem}
\begin{proof}
  Suppose that $R$ has an open cover $\Omega$.
  Then $\Omega' = \Omega \cup \{X \setminus R\}$ is an open cover of $S$ since
  $X \setminus R$ is open.
  Let $\Phi' \subseteq \Omega'$ be a finite cover of $S$, and let
  $\Phi = \Phi' \setminus \{X \setminus R\}$.
  Then $\Phi$ is a finite open cover of $R$ with $\Phi \subseteq \Omega$.
  Thus, $R$ is compact.
\end{proof}

\begin{theorem}[Nested Interval Theorem]
  Let $\seq{I_n}$ be a sequence of rectangles in $\mathbb{R}^k$ such that
  $I_{n+1} \subseteq I_n$, then the intersection of $\{I_n: n \in \mathbb{N}\}$
  is nonempty.
\end{theorem}
\begin{proof}
  For each positive integer $n$, let
  \begin{equation*}
    I_n = [a_n^{(1)}, b_n^{(1)}] \times \cdots \times
    [a_n^{(k)}, b_n^{(k)}].
  \end{equation*}
  For each $i \in \{1, \dots, k\}$, we have
  \begin{equation*}
    a_n^{(i)} \leq a_{n+m}^{(i)} \leq b_{n+m}^{(i)} \leq b_m^{(i)}
  \end{equation*}
  for any $n, m \in \mathbb{N}$.
  Thus, $\{a_n^{(i)}: n \in \mathbb{N}\}$ is bounded above, implying the
  existence of
  \begin{equation*}
    x_i = \sup(\{a_n^{(i)}: n \in \mathbb{N}\}).
  \end{equation*}
  We have
  \begin{equation*}
    a_n^{(i)} \leq x_i \leq b_m^{(i)}
  \end{equation*}
  for any $n, m \in \mathbb{N}$.
  Thus,
  \begin{equation*}
    x = (x_1, \dots, x_n) \in \bigcap_{n \geq 1} I_n,
  \end{equation*}
  completing the proof.
\end{proof}

\begin{theorem}
  Every $k$-cell in $\mathbb{R}^k$ is compact.
\end{theorem}
\begin{proof}
  Let $I = [a_1, b_1] \times \cdots \times [a_k, b_k]$.
  We have
  \begin{equation*}
    \Vert x - x' \Vert \leq \sqrt{\textstyle \sum_{i=1}^k (a_i - b_i)^2}
  \end{equation*}
  for any $x, x' \in I$.
  Assume that there is an open cover $\mathcal{O}$ of $I$ that contains no
  finite subcover of $I$.
  Let $c_i = (a_i + b_i) / 2$ for all $i \in \{1, \dots, n\}$, and let
  \begin{equation*}
    \mathcal{C} = \{I^{(1)} \times \dots \times I^{(k)}:
    \text{$I^{(i)} \in \{[a_i, c_i], [c_i, b_i]\}$ for $1 \leq i \leq k$}\}
  \end{equation*}
  be a collection of $2^k$ $k$-cells whose union is $I$.
  Then there must be a $k$-cell $I' \in \mathcal{C}$ cannot be covered by any
  finite subset of $\mathcal{O}$, or $I$ could be covered by that set,
  contradtion.

  Thus, if $I$ is not compact, then we can construct a sequence $\seq{I_n}$ of
  $k$-cells which are not covered by any finite subset of $\mathcal{O}$ such
  that $I_1 = I$, $I_{n+1} \subseteq I_n$ for any integer $n \geq 1$, and
  \begin{equation*}
    \Vert x - x' \Vert \leq \frac{\sqrt{\sum_{i=1}^k (a_i - b_i)^2}}{2^{n-1}}
  \end{equation*}
  holds for any $x, x' \in I_n$.
  It follows that there is a point $y \in \bigcap \{I_n\}$, and we have
  $y \in S$ for some $S \in \mathcal{O}$.
  Since $S$ is open, we have $B_r(y) \subseteq S$ for some $r > 0$.
  Let $N$ be a positive integer such that
  \begin{equation*}
    2^N > \frac{\sqrt{\sum_{i=1}^k (a_i - b_i)^2}}{r/2}.
  \end{equation*}
  Then for any $x \in I_N$,
  \begin{equation*}
    \Vert x - y \Vert
    \leq \frac{\sqrt{\sum_{i=1}^k (a_i - b_i)^2}}{2^{N-1}}
    < r,
  \end{equation*}
  implying $x \in B_r(y) \subseteq S$.
  It follows that $I_N \subseteq S$, and $\{S\}$ is a finite subset of
  $\mathcal{O}$, contradtion.
  Thus, $I$ is compact.
\end{proof}

\begin{theorem}[Heine--Borel Theorem]
  Let $S \subseteq \mathbb{R}^k$.
  Then $S$ is compact if and only if $S$ is closed and bounded.
\end{theorem}
\begin{proof}
  ($\Leftarrow$)
  If $S$ is closed and bounded, then there is a $k$-cell $I$ with $S \subseteq
  I$.
  Since $I$ is compact, and $S$ is closed, we conclude that $S$ is compact.

  ($\Rightarrow$)
  Suppose that $S$ is compact.
  Then $S$ is closed.
  Since $\mathcal{O} = \{B_r(0_{\mathbb{R}^k}): r \in \mathbb{N}\}$ is an open
  cover of $S$, there is $\mathcal{O}' \subseteq \mathcal{O}$ such that $S
  \subseteq \bigcup \mathcal{O}'$.
  It can be shown that $\bigcup \mathcal{O}'$ is bounded, and thus $S$ is
  bounded.
\end{proof}