\chapter{Real Numbers}
\section{Fields}
\begin{definition}
  A nonempty set $F$ and two operations $+$ and $\cdot$ form a \emph{field} if
  the following axioms (A 1) -- (A 5), (M 1) -- (M 5) and (D) are satisfied.
  \begin{enumerate}[leftmargin=3.5em]
    \item[(A 1)] $x + y \in F$ for any $x, y \in F$.
    \item[(A 2)] $x + y  = y + x$ for any $x, y \in F$.
    \item[(A 3)] $(x + y) + z = x + (y + z)$ for any $x, y, z \in F$.
    \item[(A 4)] There is an element $0 \in F$ such that $x + 0 = x$ for any
    $x \in F$.
    \item[(A 5)] For each $x \in F$ there is an element $-x$ in $F$ such that
    $x + (-x) = 0$.
    \item[(M 1)] $x \cdot y \in F$ for any $x, y \in F$.
    \item[(M 1)] $x \cdot y  = y \cdot x$ for any $x, y \in F$.
    \item[(M 2)] $(x \cdot y) \cdot z = x \cdot (y \cdot z)$ for any
    $x, y, z \in F$.
    \item[(M 3)] There is an element $1 \in F \setminus \{0\}$ such that
    $x \cdot 1 = x$ for any $x \in F$.
    \item[(M 4)] For each $x \in F \setminus \{0\}$ there is an element
    $x^{-1}$ in $F$ such that $x \cdot x^{-1} = 0$.
    \item[(D)] $x \cdot (y + z) = x \cdot y + x \cdot z$ for any
    $x, y, z \in F$.
  \end{enumerate}
\end{definition}

\begin{theorem}
  Let $F$ be a field.
  Then the following statements are true for any $x, y, z \in F$.
  \begin{enumerate}
    \item If $x + y = x + z$, then $y = z$.
    \item If $x + y = x$, then $y = 0$.
    \item If $x + y = 0$, then $y = -x$.
    \item $-(-x) = x$.
  \end{enumerate}
\end{theorem}
\begin{proof}
  Note that these statements are consequence of axioms (A 1) -- (A 5).
  \begin{enumerate}
    \item We have
    \begin{align*}
      y
      &= 0 + y \\
      &= (-x + x) + y \\
      &= -x + (x + y) \\
      &= -x + (x + z) \\
      &= (-x + x) + z \\
      &= 0 + z \\
      &= z.
    \end{align*}
    \item Since $x + y = x = x + 0$, we have $y = 0$ by (a).
    \item Since $x + y = 0 = x + (-x)$, we have $y = -x$ by (a).
    \item Since $-x + x = 0$, we have $-(-x) = x$ by (c). \qedhere
  \end{enumerate}
\end{proof}

\begin{theorem}
  Let $F$ be a field.
  Then the following statements are true for any $x \in F \setminus \{0\}$
  and $y, z \in F$.
  \begin{enumerate}
    \item If $x \cdot y = x \cdot z$, then $x = y$.
    \item If $x \cdot y = x$, then $y = 1$.
    \item If $x \cdot y = 1$, then $y = x^{-1}$.
    \item $(x^{-1})^{-1} = x$.
  \end{enumerate}
\end{theorem}
\begin{proof}
  Note that these statements are consequence of axioms (M 1) -- (M 5).
  \begin{enumerate}
    \item We have
    \begin{align*}
      y
      &= 1 \cdot y \\
      &= (x^{-1} \cdot x) \cdot y \\
      &= x^{-1} \cdot (x \cdot y) \\
      &= x^{-1} \cdot (x \cdot z) \\
      &= (x^{-1} \cdot x) \cdot z \\
      &= 1 \cdot z \\
      &= z.
    \end{align*}
    \item Since $x \cdot y = x = x \cdot 1$, we have $y = 1$ by (a).
    \item Since $x \cdot y = 1 = x \cdot x^{-1}$, we have $y = x^{-1}$ by (a).
    \item Since $x^{-1} + x = 1$, we have $(x^{-1})^{-1} = x$ by (c). \qedhere
  \end{enumerate}
\end{proof}

\begin{theorem}
  Let $F$ be a field.
  Then the following statements are true for any $x, y \in F$.
  \begin{enumerate}
    \item $0 \cdot x = 0$.
    \item $(-x) \cdot y = -(x \cdot y) = x \cdot (-y)$.
    \item $(-x) \cdot (-y) = x \cdot y$.
  \end{enumerate}
\end{theorem}
\begin{proof}
  \leavevmode
  \begin{enumerate}
    \item We have
    \begin{equation*}
      0 \cdot x + 0 \cdot x = (0 + 0) \cdot x = 0 \cdot x,
    \end{equation*}
    implying $0 \cdot x = 0$.
    \item Since
    \begin{equation*}
      (-x) \cdot y + x \cdot y = (-x + x) \cdot y = 0 \cdot y = 0,
    \end{equation*}
    we have $(-x) \cdot y = -(x \cdot y)$.
    One can prove $x \cdot (-y) = -(x \cdot y)$ similarly.
    \item We have
    \begin{equation*}
      (-x) \cdot (-y) = -(x \cdot (-y)) = -(-(x \cdot y)) = x \cdot y
    \end{equation*}
    by applying (b) twice. \qedhere
  \end{enumerate}
\end{proof}

\section{Ordered Fields}
\begin{definition}
  An \emph{ordered field} is a field on which relation $<$ is defined such
  that the following axioms (O 1) -- (O 4) hold for any $x, y, z \in F$.
  \begin{enumerate}[leftmargin=3.5em]
    \item[(O 1)] One and only one of the statements $x = y$, $x < y$, $y < x$
    is true.
    \item[(O 2)] If $x < y$ and $y < z$, then $x < z$.
    \item[(O 3)] If $x < y$, then $x + z < y + z$.
    \item[(O 4)] If $0 < x$ and $0 < y$, then $0 < x \cdot y$.
  \end{enumerate}
\end{definition}

\begin{definition}
  Let $F$ be an ordered field.
  The relations $>$, $\leq$ and $\geq$ are defined as follows for any
  $x, y \in F$.
  \begin{align*}
    x > y \quad &\Leftrightarrow \quad y < x \\
    x \leq y \quad &\Leftrightarrow \quad x < y \;\;\text{or}\;\; x = y \\
    x \geq y \quad &\Leftrightarrow \quad x > y \;\;\text{or}\;\; x = y.
  \end{align*}
\end{definition}

\begin{definition}
  Let $F$ be an ordered field and let $S \subseteq F$.
  \begin{itemize}
    \item An \emph{upper bound} of $S$ is an element $x$ in $F$ such that
    $x \geq y$ for any $y \in S$.
    We say that $S$ is \emph{bounded above} if $S$ has an upper bound.
    \item A \emph{lower bound} of $S$ is an element $x$ in $F$ such that
    $x \leq y$ for any $y \in S$.
    We say that $S$ is \emph{bounded below} if $S$ has a lower bound.
  \end{itemize}
\end{definition}

\begin{definition}
  Let $F$ be an ordered field and let $S \subseteq F$.
  \begin{itemize}
    \item An element of $S$ is called the \emph{maximum} of $S$, denoted by
    $\max(S)$, if it is an upper bound of $S$.
    \item An element of $S$ is called the \emph{minimum} of $S$, denoted by
    $\min(S)$, if it is a lower bound of $S$.
    \item The minimum of the set of upper bounds of $S$ is called the
    \emph{supremum} of $S$, denoted by $\sup(S)$.
    \item The maximum of the set of lower bounds of $S$ is called the
    \emph{infimum} of $S$, denoted by $\inf(S)$.
  \end{itemize}
\end{definition}

\section{The Real Field}
\begin{definition}
  $\mathbb{R}$ is an ordered field such that every nonempty subset $S$ of
  $\mathbb{R}$ that is bounded above has a supremum.
  The elements of $\mathbb{R}$ are called the \emph{real numbers}.
\end{definition}

\begin{theorem}[Archimedean Property]
  For any $x, y \in \mathbb{R}$ with $x > 0$, there is a positive integer $n$
  such that
  \begin{equation*}
    n \cdot x > y.
  \end{equation*}
\end{theorem}
\begin{proof}
  Let
  \begin{equation*}
    S = \{nx : \text{$n$ is a positive integer}\}.
  \end{equation*}
  Suppose that $y$ is an upper bound of $S$.
  It follows that $S$ has a supremum $z$.
  Note that $z - x$ is not an upper bound of $S$ since $z - x < z$.
  Thus, $z - x < mx$ for some positive integer $m$, implying $z < (m + 1)x$,
  contradiction to the fact that $z$ is an upper bound of $S$.
  Hence, $y$ is not an upper bound of $S$, completing the proof.
\end{proof}