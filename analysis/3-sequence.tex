\chapter{Sequences and Series}
\section{Convergence}
Let $\mathbb{N} = \{1, 2, 3, \dots\}$ denote the set of positive integers.
\begin{definition}
  Let $\langle x_n \rangle_{n \in \mathbb{N}}$ be a sequence in a metric space
  $X$.
  We say that $\langle x_n \rangle_{n \in \mathbb{N}}$ \emph{converges} to a
  point $x \in X$, denoted by
  \begin{equation*}
    \lim_{n \to \infty} x_n = x,
  \end{equation*}
  if for any $\epsilon > 0$ there exists some $n_0 \in \mathbb{N}$ such that
  \begin{equation*}
    n \geq n_0
    \quad \Rightarrow \quad
    d_X(x_n, x) < \epsilon
  \end{equation*}
  holds for all $n \in \mathbb{N}$.
  \begin{itemize}
    \item We say that $\langle x_n \rangle_{n \in \mathbb{N}}$ is
    \emph{convergent} if it converges to a point in $X$.
    \item We say that $\langle x_n \rangle_{n \in \mathbb{N}}$ is
    \emph{divergent} if it is not convergent.
  \end{itemize}
\end{definition}

\begin{theorem}
  Let $\langle x_n \rangle_{n \in \mathbb{N}}$ be a sequence in a metric space
  $X$.
  If $\langle x_n \rangle_{n \in \mathbb{N}}$ converges to both $x \in X$ and
  $x' \in X$, then $x = x'$.
\end{theorem}
\begin{proof}
  For any $\epsilon > 0$, there exists some $n_0 \in \mathbb{N}$ such that
  \begin{align*}
    d_X(x_n, x) < \frac{\epsilon}{2}
    \quad \text{and} \quad
    d_X(x_n, x') < \frac{\epsilon}{2}
  \end{align*}
  hold for all $n \in \mathbb{N}$ with $n \geq n_0$.
  Then we have
  \begin{equation*}
    d_X(x, x')
    \leq d_X(x_n, x) + d_X(x_n, x')
    < \frac{\epsilon}{2} + \frac{\epsilon}{2}
    = \epsilon
  \end{equation*}
  for all $n \in \mathbb{N}$ with $n \geq n_0$.
  Thus $d_X(x, x') = 0$, implying $x = x'$.
\end{proof}

\begin{theorem}
  Let $\langle a_n \rangle_{n \in \mathbb{N}}$ and
  $\langle b_n \rangle_{n \in \mathbb{N}}$ be sequence in $\mathbb{R}$ with
  \begin{equation*}
    \lim_{n \to \infty} a_n = L
    \quad \text{and} \quad
    \lim_{n \to \infty} b_n = M.
  \end{equation*}
  Then the following statements are true.
  \begin{enumerate}
    \item We have
    \begin{equation*}
      \lim_{n \to \infty} (a_n + b_n) = L + M
      \quad \text{and} \quad
      \lim_{n \to \infty} (a_n - b_n) = L - M.
    \end{equation*}
    \item We have
    \begin{equation*}
      \lim_{n \to \infty} a_nb_n = LM.
    \end{equation*}
    \item If $L \neq 0$ and $a_n \neq 0$ for all $n \in \mathbb{N}$, then we
    have
    \begin{equation*}
      \lim_{n \to \infty} \frac{1}{a_n} = \frac{1}{L}. 
    \end{equation*}
  \end{enumerate}
\end{theorem}
\begin{proof}
  \leavevmode
  \begin{enumerate}
    \item For any $\epsilon > 0$, there exists $n_0 \in \mathbb{N}$ such that
    \begin{equation*}
      |a_n - L| < \frac{\epsilon}{2}
      \quad \text{and} \quad
      |b_n - M| < \frac{\epsilon}{2}
    \end{equation*}
    holds for all $n \in \mathbb{N}$ with $n \geq n_0$.
    Then
    \begin{equation*}
      \lvert (a_n + b_n) - (L + M) \rvert
      \leq \lvert a_n - L \rvert + \lvert b_n - M \rvert
      < \frac{\epsilon}{2} + \frac{\epsilon}{2}
      = \epsilon
    \end{equation*}
    and
    \begin{equation*}
      \lvert (a_n - b_n) - (L - M) \rvert
      \leq \lvert a_n - L \rvert + \lvert -b_n + M \rvert
      < \frac{\epsilon}{2} + \frac{\epsilon}{2}
      = \epsilon
    \end{equation*}
    hold for all $n \in \mathbb{N}$ with $n \geq n_0$, completing the proof.
    \item Choose $C > 0$ such that $|L| \leq C$ and $|b_n| \leq C$ for all
    $n \in \mathbb{N}$.
    For any $\epsilon > 0$, there exists $n_0 \in \mathbb{N}$ such that
    \begin{equation*}
      |a_n - L| < \frac{\epsilon}{2C}
      \quad \text{and} \quad
      |b_n - M| < \frac{\epsilon}{2C}
    \end{equation*}
    hold for all $n \in \mathbb{N}$ with $n \geq n_0$.
    Then
    \begin{align*}
      |a_nb_n - LM|
      &= |(a_n - L)b_n + (b_n - M)L| \\
      &\leq |a_n - L| \cdot |b_n| + |b_n - M| \cdot |L| \\
      &\leq |a_n - L| \cdot C + |b_n - M| \cdot C \\
      &< \epsilon
    \end{align*}
    holds for all $n \in \mathbb{N}$ with $n \geq n_0$, completing the proof.
    \item For any $\epsilon > 0$, there exists $n_0 \in \mathbb{N}$ such that
    \begin{equation*}
      |a_n - L|
      < \min\left(\left\{\frac{|L|^2\epsilon}{2}, \frac{|L|}{2}\right\}\right)
    \end{equation*}
    hold for all $n \in \mathbb{N}$ with $n \geq n_0$.
    Then for all $n \in \mathbb{N}$ with $n \geq n_0$ we have
    \begin{equation*}
      |a_n|
      = |L + (a_n - L)|
      \geq |L| - |a_n - L|
      > \frac{|L|}{2},
    \end{equation*}
    implying
    \begin{equation*}
      \left|\frac{1}{a_n} - \frac{1}{L}\right|
      = \left|\frac{a_n - L}{a_nL}\right|
      = \frac{|a_n - L|}{|a_n| \cdot |L|}
      < \frac{|L|^2\epsilon / 2}{|L| / 2 \cdot |L|}
      = \epsilon,
    \end{equation*}
    completing the proof.
    \qedhere
  \end{enumerate}
\end{proof}

\section{Cauchy Sequences}
\begin{definition}
  Let $\langle x_n \rangle_{n \in \mathbb{N}}$ be a sequence in a metric space
  $X$.
  We say that $\langle x_n \rangle_{n \in \mathbb{N}}$ is a
  \emph{Cauchy sequence}, if for any $\epsilon > 0$ there exists some
  $n_0 \in \mathbb{N}$ such that
  \begin{equation*}
    n \geq n_0
    \quad \text{and} \quad
    m \geq n_0
    \quad \Rightarrow \quad
    d_X(x_n, x_m) < \epsilon
  \end{equation*}
  holds for all $n, m \in \mathbb{N}$.
\end{definition}

\begin{definition}
  Let $\langle a_n \rangle_{n \in \mathbb{N}}$ be a sequence in $\mathbb{R}$.
  \begin{itemize}
    \item We say that $\langle a_n \rangle_{n \in \mathbb{N}}$ is
    \emph{increasing} (resp., \emph{strictly increasing}) if
    \begin{equation*}
      a_n \leq a_{n+1}
      \qquad
      \text{(resp., $a_n < a_{n+1}$)}
    \end{equation*}
    holds for all $n \in \mathbb{N}$.
    \item We say that $\langle a_n \rangle_{n \in \mathbb{N}}$ is
    \emph{decreasing} (resp., \emph{strictly decreasing}) if
    \begin{equation*}
      a_n \geq a_{n+1}
      \qquad
      \text{(resp., $a_n > a_{n+1}$)}
    \end{equation*}
    holds for all $n \in \mathbb{N}$.
  \end{itemize}
\end{definition}

\begin{theorem}
  Let $\langle a_n \rangle_{n \in \mathbb{N}}$ be a sequence in $\mathbb{R}$.
  If $\langle a_n \rangle_{n \in \mathbb{N}}$ is increasing and its range
  $A = \{a_n: n \in \mathbb{N}\}$ is bounded above, then
  $\langle a_n \rangle_{n \in \mathbb{N}}$ converges.
\end{theorem}
\begin{proof}
  Let $L = \sup(A)$.
  For any $\epsilon > 0$, since $L - \epsilon$ is not an upper bound of $A$,
  there exists $n_0 \in \mathbb{N}$ with $a_{n_0} > L - \epsilon$.
  Since $\langle a_n \rangle_{n \in \mathbb{N}}$ is increasing, for all
  $n \in \mathbb{N}$ with $n \geq n_0$ we have
  \begin{equation*}
    L - \epsilon < a_{n_0} \leq a_n \leq L,
  \end{equation*}
  and thus $|a_n - L| < \epsilon$.
  It follows that $\langle a_n \rangle_{n \in \mathbb{N}}$ converges to $L$.
\end{proof}
