\chapter{Sequences and Series}
\begin{definition}
  Let $X$ be a metric space and let $(x_n)_{n \in \mathbb{N}}$ be a sequence in
  $X$.
  We say that $(x_n)_{n \in \mathbb{N}}$ \emph{converges} to a point $x \in X$,
  denoted by
  \begin{equation*}
    \lim_{n \to \infty} x_n = x,
  \end{equation*}
  if for any real number $\epsilon > 0$ there is a positive integer $N$
  such that $d_X(x_n, x) < \epsilon$ for any integer $n > N$.
  \begin{itemize}
    \item We say that $(x_n)_{n \in \mathbb{N}}$ is \emph{convergent} if it
    converges to some point in $X$.
    \item We say that $(x_n)_{n \in \mathbb{N}}$ is \emph{divergent} if it is
    not convergent.
  \end{itemize}
\end{definition}

\begin{theorem}
  Let $(x_n)_{n \in \mathbb{N}}$ be a sequence in a metric space $X$.
  If $(x_n)_{n \in \mathbb{N}}$ converges to both $x \in X$ and $x' \in X$,
  then $x = x'$.
\end{theorem}
\begin{proof}
  For any $\epsilon > 0$, there exists a positive integer $N$ such that
  \begin{align*}
    d_X(x_n, x) < \frac{\epsilon}{2}
    \quad \text{and} \quad
    d_X(x_n, x') < \frac{\epsilon}{2}
  \end{align*}
  hold for any integer $n \geq N$.
  It follows that
  \begin{equation*}
    d_X(x, x')
    \leq d_X(x_n, x) + d_X(x_n, x')
    < \frac{\epsilon}{2} + \frac{\epsilon}{2}
    = \epsilon
  \end{equation*}
  holds for any integer $n \geq N$.
  Thus, $x = x'$.
\end{proof}

\begin{theorem}
  Let $(a_n)_{n \in \mathbb{N}}$ and $(b_n)_{n \in \mathbb{N}}$ be real
  sequences with
  \begin{equation*}
    \lim_{n \to \infty} a_n = L
    \quad \text{and} \quad
    \lim_{n \to \infty} b_n = M.
  \end{equation*}
  Then the following statements are true.
  \begin{enumerate}
    \item $\lim\limits_{n \to \infty} (a_n + b_n) = L + M$, and
    $\lim\limits_{n \to \infty} (a_n - b_n) = L - M$.
    \item $\lim\limits_{n \to \infty} a_nb_n = LM$.
    \item If $L \neq 0$ and $a_n \neq 0$ for all $n \in \mathbb{N}$, then
    $\lim\limits_{n \to \infty} a_n^{-1} = L^{-1}$.
  \end{enumerate}
\end{theorem}
\begin{proof}
  \leavevmode
  \begin{enumerate}
    \item For any $\epsilon > 0$, there exists a positive integer $N$ such that
    for any $n \geq N$, we have
    \begin{equation*}
      |a_n - L| < \frac{\epsilon}{2}
      \quad \text{and} \quad
      |b_n - M| < \frac{\epsilon}{2},
    \end{equation*}
    implying
    \begin{equation*}
      |(a_n + b_n) - (L + M)|
      \leq |a_n - L| + |b_n - M|
      < \frac{\epsilon}{2} + \frac{\epsilon}{2}
      = \epsilon.
    \end{equation*}
    
    \item Let $C > 0$ such that $|L| \leq C$ and $|b_n| \leq C$ for any
    positive integer $n$.
    For any $\epsilon > 0$, there exists a positive integer $N$ such that for
    any $n \geq N$, we have
    \begin{equation*}
      |a_n - L| < \frac{\epsilon}{2C}
      \quad \text{and} \quad
      |b_n - M| < \frac{\epsilon}{2C},
    \end{equation*}
    implying
    \begin{align*}
      |a_nb_n - LM|
      &= |(a_n - L)b_n + (b_n - M)L| \\
      &\leq |a_n - L||b_n| + |b_n - M||L| \\
      &< \frac{\epsilon (|b_n| + L)}{2C} \\
      &\leq \epsilon.
    \end{align*}
  
    \item For any $\epsilon > 0$, there exists a positive integer $N$ such that
    for any $n \geq N$, we have
    \begin{equation*}
      |a_n - L| < \frac{|L|^2\epsilon}{2}
      \quad \text{and} \quad
      |a_n - L| < \frac{|L|}{2}.
    \end{equation*}
    It follows that
    \begin{equation*}
      |a_n|
      = |L + (a_n - L)|
      \geq |L| - |a_n - L|
      > \frac{|L|}{2},
    \end{equation*}
    implying
    \begin{equation*}
      \left|\frac{1}{a_n} - \frac{1}{L}\right|
      = \left|\frac{a_n - L}{a_nL}\right|
      = \frac{|a_n - L|}{|a_n||L|}
      < \frac{2|a_n - L|}{|L|^2}
      < \epsilon.
      \qedhere
    \end{equation*}
  \end{enumerate}
\end{proof}

\begin{definition}
  Let $\seq{a_n}$ be a sequence of real numbers.
  \begin{itemize}
    \item We say that $\seq{a_n}$ is \emph{increasing} (resp., \emph{strictly
    increasing}) if $a_n \leq a_{n+1}$ (resp. $a_n < a_{n+1}$) holds for all
    $n \in \mathbb{N}$.
    \item We say that $\seq{a_n}$ is \emph{decreasing} (resp., \emph{strictly
    decreasing}) if $a_n \geq a_{n+1}$ (resp. $a_n > a_{n+1}$) holds for all
    $n \in \mathbb{N}$.
  \end{itemize}
\end{definition}

\begin{theorem}
  Let $\seq{a_n}$ be a sequence of real numbers.
  If $\seq{a_n}$ is increasing and bounded above, then $\seq{a_n}$ converges.
\end{theorem}
\begin{proof}
  Let $L = \sup(\{a_n: n \in \mathbb{N}\})$.
  For any $\epsilon > 0$, since $L - \epsilon$ is not an upper bound of
  $\{a_n: n \in \mathbb{N}\}$, there exists a positive integer $N$ with
  $a_N > L - \epsilon$.
  Since $\seq{a_n}$ is increasing, we have
  \begin{equation*}
    L - \epsilon < a_N < a_n \leq L
  \end{equation*}
  for all positive integer $n > N$, implying $|a_n - L| \leq \epsilon$ for all
  positive integer $n > N$.
  Thus, $\seq{a_n}$ converges to $L$.
\end{proof}

\begin{definition}
  Let $(X, d)$ be a metric space and let $\seq{x_n}$ be a sequence in $X$.
  We say that $\seq{x_n}$ is a \emph{Cauchy sequence} if for any $\epsilon > 0$
  there is a positive integer $N$, such that $n \geq N$ and $m \geq N$ implies
  $d(x_n, x_m) < \epsilon$.
\end{definition}